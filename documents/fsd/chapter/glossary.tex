\label{ch:glossary}
%%A dot will be automatically added to the description.
%%Keep in mind that only entries with at least one reference will be visible in the compiled PDF
%%To refer to a glossary entry, use \gls{ID} (singular), \glspl (plural) or \Gls and \Glspl to print the first character in uppercase.
%%The glossary will automatically be printed in alphabetical order

\newglossaryentry{glossarydummy} %%this is the ID by which the entry will be referenced
{
	name=Dummy, %%the name displayed in the glossary
	text=dummy, %%the name inserted in text
	plural=dummies, %%the plural inserted when using \glspl
	description={This entry will not be visible in the final document as it should stay unreferenced}, %%the description printed in the glossary
}
\newglossaryentry{user} %%this is the ID by which the entry will be referenced
{
	name=User,
	text=user,
	plural=users,
	description={A doctor conducting cancer or genetic disorder related research on a \gls{device}}
}
\newglossaryentry{session}
{
	name=Session,
	text=session,
	plural=sessions,
	description={A session refers to a continuous time interval in which a recording was made on a \gls{device}}
}
\newglossaryentry{device}
{
	name=Device,
	text = device,
	plural=devices,
	description={A single electronic device intended for direct usage by \glspl{user}. In the scope of this project this always refers to a Windows PC as specified in chapter \ref{ch:environment}}
}
\newglossaryentry{event}
{
	name=Event,
	text=event,
	plural=events,
	description={Actions a user performs on a \gls{device} such as pressing a key or launching a program. The specifics are discussed in chapter \ref{ch:data}}
}
\newglossaryentry{module}
{
	name=Module,
	text=module,
	plural=modules,
	description={In the context of this document, a module refers to an optional component which extends to application's functionality, e.g. by collecting additional data}
}
\newglossaryentry{scientist}
{
	name=Data scientist,
	text=data scientist,
	plural=data scientists,
	description={A person working with the recorded data in order to extract knowledge and insights}
}
\newglossaryentry{videostream}
{
	name=Videostream,
	text=videostream,
	plural=videostreams,
	description={Digital, coherent data encoding the video (and optionally audio-) output of a \gls{device}}
}
\newglossaryentry{converter}
{
	name=Converter,
	text=converter,
	plural=converters,
	description={A piece of software designed to alter the digital representation of information from one form to another, ideally without loss or alteration of information}
}
\newglossaryentry{browser}
{
	name=Browser,
	text=browser,
	plural=browsers,
	description={An application which allows \glspl{user} to view and interact with websites}
}
\newglossaryentry{admin}
{
	name=Administrator,
	text=administrator,
	plural=administrators,
	description={A person responsible for setting up and configuring the software. Usually has permissions exceeding those of \glspl{user}}
}