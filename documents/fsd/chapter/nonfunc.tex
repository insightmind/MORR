\chapter{Non-functional Specifications}
\label{ch:nonfunc}

\requirementscope{NF}{nfspec}

All non-functional requirements are oriented towards a \gls{device} using a Intel Core i5 8250U or better with at least 4GB of memory and a screen resolution of 1080p or smaller.

\begin{itemize}
\nfspec{CPU Performance\\
    The CPU utilization by the application on the \gls{user}'s \gls{device} should not exceed 25\% while running a \gls{session}.
}

\nfspec{Memory Usage\\
    The memory usage by the application on the \gls{user}'s \gls{device} should not exceed 250MB while running a \gls{session}.
}

\nfspec{Interaction Performance\\
	The features described in \specref{FS50} should not exceed a response time of 5 seconds.
}

\nfspec{Postprocessing\\
	After the \gls{user} has stopped a \gls{session},\\
	the postprocessing and saving of the \gls{session} should take no longer than 2 minutes. 
}

\nfspec{Unambiquity of session identifiers\\ 
	After running 100 \glspl{session} the identifier of each session must be unique.
}

\nfspec{Usability\\
	After a 5-minute introduction the \gls{user} does not commit more than 1 mistake on a whole day.
}

\nfspec{Internet Availability\\
	The user can record a \gls{session} and save sessions locally without being connected to the internet.
}

\nfspec{Modularity\\
	\Glspl{module}, defined in chapter \ref{ch:sysmodels}, are interchangeable and exchangeable.\\
	The application therefore should support running at least 10 \glspl{module} per \gls{session}.
}

% This is only for a better layout structure in the pdf document.
\newpage

\nfspec{Reliability\\
	The probability of a failure during a 60-minute \gls{session} is under 2\%.\\
	A failure is an unintended termination of a recording \gls{session} with a possible loss of recorded data.\\
	This does not apply should the application be run in an unsupported environment or configuration (see chapter \ref{ch:environment}).
}
\end{itemize}