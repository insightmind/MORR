\chapter{Functional Specifications}
\label{ch:func}

\newcounter{counterFS}
\newcommand{\internalspec}[5]
{
    \item[FS\arabic{counterFS}\phantomsection\label{FS\arabic{counterFS}}]
    \begin{itemize}[noitemsep]
        \item[]{\textbf{#1}} %title
        \item[]{Actor: #2}
        \item[]{Precondition: #3}
        \item[]{Action: #4}
        \item[]{Postcondition: #5}
    \end{itemize}
}
\newcommand{\fspec}[6][]{\ifthenelse{\equal{#1}{}}{\setcounter{counterFS}{\intcalcAdd{\value{counterFS}}{\intcalcSub{10}{\intcalcMod{\value{counterFS}}{10}}}}}{\setcounter{counterFS}{#1}}\internalspec{#2}{#3}{#4}{#5}{#6}}
\newcommand{\fsubspec}[5]{\stepcounter{counterFS}\internalspec{#1}{#2}{#3}{#4}{#5}}

%%commonly used post-conditions
\newcommand{\sessionactive}{A \gls{session} is currently being recorded.}
\newcommand{\sessioninactive}{No \gls{session} is currently being recorded.}
\newcommand{\sessionsavail}{One or more sessions have been recorded and not deleted.}
\newcommand{\applaunched}{The application is running.}
\newcommand{\configvalid}{The configuration file is valid.}
\newcommand{\configinvalid}{The configuration file is invalid.}
This chapter is meant to organize and describe the specifications that dictate the way the application functions for different actors.
\newcommand{\acceptinfodialog}{The user has confirmed the information dialogue.}

\section{High-level specification}
\label{sec:func_highlevel}
The user interface components mentioned in this section are specified in \ref{sec:uispec}.
\begin{itemize}
    \fspec{Identifiable \glspl{session}}{\Gls{user}}{\applaunched{} \configvalid{}}{When the user starts a \gls{session}-recording by pressing the ``Start Recording''-button, a unique session-ID gets created.}{The recording-managing component holds a session-ID which all \glspl{event} recorded during the current \gls{session} will get associated with.}
    \fspec{Event timings}{\Gls{user}}{\sessionactive}{When the user triggers an event specified in chapter \ref{ch:data}, the respective \gls{module} (see \ref{sec:modules}) generates a data-structure (as defined in \ref{sec:events}) associated with the event and adds the timestamp to it.}{The data structure associated with the event is sent to the data-managing component and contains a timestamp describing its time of occurence.}
    \fspec{Event types}{\Gls{user}}{\sessionactive}{When the user triggers an event specified in chapter \ref{ch:data}, the respective module (see \ref{sec:modules}) generates a data-structure (as defined in \ref{sec:events}) associated with the event and adds the event-type to it.}{The data structure associated with the event is sent to the data-managing component and contains a field describing its \gls{event}-type.}
    \fspec{Event processing}{\Gls{user}}{\sessionactive}{When the user triggers an event specified in chapter \ref{ch:data}, the data structure corresponding to this event is processed and filtered by \glspl{module} (see \ref{sec:modules} and \ref{sec:sysarchitecture}). If the configuration does not specify to discard the event, the event data structure is sent to the data-managing component by the respective module.}{The event data structure is either sent to the data-managing component for serialization or deleted based on a configured rule.}
    \fspec{Configurability}{\Gls{admin}}{\sessioninactive}{An \gls{admin} opens the configuration file and adds/modifies/removes rules affecting some of the configurable behavior.}{The configuration-managing component will use the configuration as specified in the changed configuration file the next time the application is launched.}
    \begin{itemize}
    \fsubspec{Configuring active modules}{\Gls{admin}}{\sessioninactive}{An \gls{admin} opens the configuration file and changes the \glspl{module} the application is allowed to use.}{The module-managing component will manage the \glspl{module} as specified in the changed configuration file the next time the application is launched.}
    \fsubspec{Configuring video recording}{\Gls{admin}}{\sessioninactive}{An \gls{admin} opens the configuration file and changes the framerate, bitrate or resolution of the recorded videostream.}{The video-recording-managing component will record the videostream as specified in the changed configuration file the next time the application is launched.}
    \fsubspec{Configuring audio recording}{\Gls{admin}}{\sessioninactive}{An \gls{admin} opens the configuration file and enables or disables the recording of audio.}{The audio-recording-managing component will record or not record audio as specified in the changed configuration file the next time the application is launched.}
    \fsubspec{Configuring saving path}{\Gls{admin}}{\sessioninactive}{An \gls{admin} opens the configuration file and changes the system path where the recording of the current \gls{user} will be stored.}{The recording-managing component will store the recordings at the location as specified in the changed configuration file the next time the application is launched.}
    \end{itemize}
    \fspec{Error-detection}{\Gls{user}}{\configinvalid}{The \gls{user} starts the application. The configuration-managing component checks the configuration file and detects syntactic errors, therefore the application interrupts the startup procedure.}{The application has not regularly started and is therefore not ready for recording. For UI details, see \hyperref[sec:uispec]{Error dialogue}.}
    \fspec{\Gls{user} controllability (start)}{\Gls{user}}{\applaunched{} \sessioninactive{} \acceptinfodialog}{The \gls{user} clicks on the button labeled ``Start Recording''. The recording-managing component starts a new \gls{session}-recording.}{\sessionactive}
    \fspec{\Gls{user} instruction}{\Gls{user}}{\applaunched{} \sessioninactive{} \configvalid}{The \gls{user} clicks on the button labeled ``Start Recording''. The user-interface-managing component opens an information dialogue.}{A dialogue is being displayed, advising the \gls{user} not to enter any personal or confidential business information.}
    \fspec{\Gls{user} controllability (stop)}{\Gls{user}}{\sessionactive}{The \gls{user} clicks on the button labeled ``Stop Recording''. The recording-managing component will stop the current \gls{session}-recording.}{\sessioninactive{} The user-interface-managing component shows the save-dialogue to the \gls{user}, providing the options to store or discard the recorded session. For UI details, see \hyperref[sec:uispec]{Save dialogue}.} %%TODO: link UI
    \fspec{\Gls{user} controllability (quit)}{\Gls{user}}{\applaunched{} \sessioninactive}{The user chooses the ``Quit'' option. The application terminates.}{The application is no longer running.}
    \fspec{Quit while recording}{\Gls{user}}{\applaunched{} \sessionactive}{The user chooses the ``Quit'' option while a session is being recorded. The user-interface-managing-component opens the save-dialogue.}{The save-dialogue is shown to the \gls{user}. The application terminates immediately if the \gls{user} decided to discard the recording or terminates after the recording has been saved by the recording-managing component if the \gls{user} decided to save the recording.}
    \fspec{Save prompt}{\Gls{user}}{A \gls{session}-recording has been stopped and the save-dialogue is displayed.}{The user chooses whether to save or discard the recorded session.}{If the user chose to save the recording, the recording is stored by the recording-managing component at the location specified in the configuration file. If the user chose the discard-option, the recording is not stored.}
    \fspec{\Gls{user} reviewability}{\Gls{user}}{\sessionsavail}{A \gls{user} clicks on the ``Open recordings folder'' button. The user-interface-managing component opens the recordings folder in the Windows File Explorer, displaying the files in which this \gls{user}'s past recordings have been stored.}{The folder containing the user's recordings is opened in the Windows File Explorer. The \gls{user} can view the videostream contained in these files by opening them with a compatible video-player, e.g. Windows Media Player.}
\end{itemize}

\section{Core application specification}
\label{sec:func_core}
The core application has to cover the following responsibilities:
\begin{itemize}
    \fspec[200]{Videostream recording}{\Gls{user}}{\sessionactive}{While a session is being recorded, the video-recording-managing component constantly captures the \gls{device}'s primary screen's content into a videostream which the recording-managing component then serializes. The \gls{videostream} forms the basis for every recording.}{The \gls{session} recording contains a \gls{videostream} containing the \gls{device}'s primary screen's content.}
    \fspec{\hyperref[sec:system-modules]{System module} invocation}{\Gls{user}}{\applaunched}{The \gls{user} clicks the ``Start recording'' button. The system-module-managing component invokes the necessary \hyperref[sec:system-modules]{system modules} according to the configuration.}{All \hyperref[sec:system-modules]{system modules} that should be active according to the configuration are active and listening for their respective \glspl{event}.}
    \fspec{\hyperref[sec:system-modules]{System module} configuration}{\Gls{user}}{\configvalid}{The \gls{user} starts the application. The system-module-managing component configures the \hyperref[sec:system-modules]{system modules} according to the configuration.}{The \hyperref[sec:system-modules]{system modules} are configured according to the configuration and the application is ready for recording.}
    \fspec{\hyperref[sec:on-demand-modules]{On-demand module} registration}{\Gls{user}}{\sessionactive}{The \gls{user} starts an application (e.g. a \gls{browser}) which contains an \hyperref[sec:on-demand-modules]{on-demand module}. The \hyperref[sec:on-demand-modules]{on-demand module} notifies the on-demand-module-managing component about its availability. The on-demand-module managing component reacts by opening a communication port to the \hyperref[sec:on-demand-modules]{on-demand module} to allow for reception of event-data. It also configures the \hyperref[sec:on-demand-modules]{on-demand module} according to the configuration.}{The \hyperref[sec:on-demand-modules]{on-demand module} is ready to send filled-in \glspl{event} to the data-managing component.}
    \fspec{\hyperref[sec:on-demand-modules]{On-demand module} initial registration}{\Gls{user}}{\applaunched{} An application containing an \hyperref[sec:on-demand-modules]{On-demand module} is already running.}{The \gls{user} clicks on the button labeled ``Start Recording''. The on-demand-module managing component registers the available \hyperref[sec:on-demand-modules]{On-demand module} and reacts by opening a communication port to the \hyperref[sec:on-demand-modules]{on-demand-module} to allow for reception of event-data. It also configures the \hyperref[sec:on-demand-modules]{on-demand module} according to the configuration.}{The \hyperref[sec:on-demand-modules]{on-demand module} is ready to send filled-in \glspl{event} to the data-managing component.}
    \fspec{Data serialization}{\Gls{user}}{\sessionactive}{The \gls{user} triggers an event which is noticed by a \gls{module}. The \gls{module} sends the filled-in event to the data-managing component. The data-managing component collects events from multiple modules and the recording-managing component stores them in the same file (see \ref{sec:session_recordings}).}{The event-data received from all active \glspl{module} is written to the same file.}
    \fspec{Combinability of \glspl{module}}{\Gls{user}}{\configvalid{} All modules required by the configuration are installed.}{The \gls{user} starts the application and starts a \gls{session}-recording. The module-managing component sets up the internal processing-pipeline described in \ref{sec:sysarchitecture} in such a way that modules that receive events as input are connected to those that generate events as output and the modules adhere to the configuration.}{The application is ready to process the \glspl{event} in the pipeline as configured.}
    \fspec{Context management}{\Gls{user}}{\sessionactive}{After interacting with an application \textbf{A}, the \gls{user} starts interacting with another application \textbf{B}. The context-managing component registers this change (see \ref{sec:context}) and informs its modules where necessary.}{The context-managing component holds information on the current context and the modules that need to be informed about changes to the context have been informed.}
\end{itemize}

\section{User interface specification}
\label{sec:uispec}
\begin{itemize}
    \fspec[300]{Recording indicator}{\Gls{user}}{\applaunched{} \sessioninactive{} \configvalid}{The \gls{user} clicks on the button labeled ``Start Recording''. The user-interface-managing component will start showing a yellow border along the screen edges.}{A yellow border is being drawn along the screen edges as long as the recording is active.}
    \fspec{Error dialogue}{\Gls{user}}{\configinvalid}{The \gls{user} starts the application. The configuration-managing component will detect syntactic errors in the configuration file and the user-interface-managing component will display an error-message.}{An error-dialogue is shown, informing that the current configuration is invalid. When the user clicks the ``OK'' button shown in this dialogue, the application will terminate.}
    \fspec{Save dialogue}{\Gls{user}}{\sessionactive}{The user stops the recording by pressing the ``Stop Recording'' button or the ``Quit'' button. The user-interface-managing component opens a new dialogue.}{A dialogue window is shown to the \gls{user}, providing the options ``Save recording'' and ``Discard recording''.}
    \fspec{Event-Data extraction}{\Gls{scientist}}{\sessionsavail}{A \gls{scientist} starts the supplied command-line-tool and supplies the paths of one or multiple recordings as well as an output path as parameters. The tool will start extracting the stored event-data from the recording.}{The event-data has been extracted from the specified recordings and stored in CSV-format (comma separated values) at the location specified as output path. The recordings specified as input files have not been altered.}
    \fspec{Tray-Icon}{\Gls{user}}{\configvalid}{The user starts the application. The user-interface-managing component adds a new icon to the tray, indicating that the software has been launched successfully.}{An additional icon is shown in the system tray.}
    \fspec{Tray-Menu}{\Gls{user}}{\applaunched}{The user clicks on the tray-icon. The user-interface-managing component shows a menu providing the following options:
    \begin{itemize}
    \item ``Start recording'' or ``Stop recording'' (based on whether a session is already being recorded)
    \item ``Open recordings directory''
    \item ``Quit''
    \end{itemize}}{A small menu is shown above the tray icon, providing above options.}
\end{itemize}

\section{Per-module specification}
\label{sec:modules}
This section discusses the default \glspl{module} which will be available and ship with the application at release in order to fulfill the mandatory requirements.

\subsection{System modules}
\label{sec:system-modules}

System modules are \glspl{module} which communicate with the operating system. They are invoked as soon as a \gls{session}-recording is started and only terminate at the end of a recording. All system \glspl{module} are \hyperref[collecting-module]{collecting modules} as specified in \ref{sec:module-types}.

\subsubsection{Window-management module}

\begin{itemize}
    \fspec[400]{Window management events}{\Gls{user}}{\sessionactive}{When the \gls{user} performs one of the following window interactions, a window \gls{event} (as defined in \specref{D310}) gets created by this \gls{module} and sent to the data-managing component.}{The data-managing component received a filled-in window \gls{event}.}
    \begin{itemize}
    \fsubspec{Focusing a window}{\Gls{user}}{\sessionactive}{The \gls{user} focuses a window. The module creates a focus-window \gls{event} (as defined in \specref{D311}) and sends it to the data-managing component.}{The data-managing component received a filled-in focus-window \gls{event}.}
    \fsubspec{Moving a window}{\Gls{user}}{\sessionactive}{The \gls{user} moves a window. The module creates a move-window \gls{event} (as defined in \specref{D312}) and sends it to the data-managing component.}{The data-managing component received a filled-in move-window \gls{event}.}
    \fsubspec{Resizing a window}{\Gls{user}}{\sessionactive}{The \gls{user} modifies the size of a window. The module creates a resize-window \gls{event} (as defined in \specref{D313}) and sends it to the data-managing component.}{The data-managing component received a filled-in resize-window \gls{event}.}
    \fsubspec{Minimizing/Maximizing/Restoring a window}{\Gls{user}}{\sessionactive}{The \gls{user} minimizes, maximizes or restores a window. The module creates a minimize-window/maximize-window/restore-window \gls{event} (as defined in \specref{D314}) and sends it to the data-managing component.}{The data-managing component received a filled-in minimize-window/maximize-window/restore-window \gls{event}.}
    \end{itemize}
\end{itemize}

\subsubsection{Mouse module}

\begin{itemize}
    \fspec{Mouse interaction events}{\Gls{user}}{\sessionactive}{When the \gls{user} performs one of the following mouse inputs, a specialized mouse \gls{event} gets created by this \gls{module} and sent to the data-managing component.}{The data-managing component received a filled-in mouse \gls{event}.}
    \begin{itemize}
    \fsubspec{Mouse Click}{\Gls{user}}{\sessionactive}{The \gls{user} presses/releases the left/right/middle mouse button. The module creates a press/release left/right/middle mouse-button \gls{event} (as defined in \specref{D320}) and sends it to the data-managing component.}{The data-managing component received a filled-in press/release left/right/middle mouse-button \gls{event}.}
    \fsubspec{Scroll Wheel}{\Gls{user}}{\sessionactive}{The \gls{user} scrolls the mouse-wheel. The module creates a scroll mouse-wheel \gls{event} (as defined in \specref{D321}) and sends it to the data-managing component.}{The data-managing component received a filled-in scroll mouse-wheel \gls{event}.}
    \fsubspec{Mouse Movement}{\Gls{user}}{\sessionactive}{The \gls{user} moves the mouse. The module creates a mouse-movement \gls{event} (as defined in \specref{D322}) and sends it to the data-managing component.}{The data-managing component received a filled-in mouse-movement \gls{event}.}
    \end{itemize}
\end{itemize}

\subsubsection{Keyboard module}

\begin{itemize}
    \fspec{Keyboard interaction events}{\Gls{user}}{\sessionactive}{When the \gls{user} performs one of the following keyboard inputs, a specialized keyboard \gls{event} gets created by this \gls{module} and sent to the data-managing component.}{The data-managing component received a filled-in keyboard \gls{event}.}
    \begin{itemize}
    \fsubspec{Releasing/Pressing a key}{\Gls{user}}{\sessionactive}{The \gls{user} presses/releases a key. The module creates a press/release keyboard-key \gls{event} (as defined in \specref{D330}) and sends it to the data-managing component.}{The data-managing component received a filled-in press/release keyboard-key \gls{event}.}
    \end{itemize}
\end{itemize}

\subsubsection{Clipboard module}

\begin{itemize}
    \fspec{Clipboard events}{\Gls{user}}{\sessionactive}{When the \gls{user} performs one of the following clipboard-interactions, a specialized clipboard \gls{event} gets created by this \gls{module} and sent to the data-managing component.}{The data-managing component received a filled-in clipboard \gls{event}.}
    \begin{itemize}
    \fsubspec{Copying to the clipboard}{\Gls{user}}{\sessionactive}{The \gls{user} copies data to the clipboard. The module creates a clipboard-copy \gls{event} (as defined in \specref{D340}) and sends it to the data-managing component.}{The data-managing component received a filled-in clipboard-copy \gls{event}.}
    \fsubspec{Pasting from the clipboard}{\Gls{user}}{\sessionactive}{The \gls{user} pastes data from the clipboard. The module creates a clipboard-paste \gls{event} (as defined in \specref{D340}) and sends it to the data-managing component.}{The data-managing component received a filled-in clipboard-paste \gls{event}.}
    \end{itemize}
\end{itemize}

\subsection{On-demand modules}
\label{sec:on-demand-modules}

On-demand modules are \glspl{module} which are dynamically run or stopped based on the software they are tracking. These \glspl{module} need to register with the application core at runtime. All on-demand \glspl{module} are \hyperref[collecting-module]{collecting modules} as specified in \ref{sec:module-types}.
\subsubsection{Browser module}

\begin{itemize}
    \fspec{Browser events}{\Gls{user}}{\sessionactive}{When the \gls{user} performs one of the following browser-interactions, a browser \gls{event} (as defined in \specref{D350}) gets created by this \gls{module} and sent to the data-managing component.}{The data-managing component received a filled-in browser \gls{event}.}
    \begin{itemize}
    \fsubspec{Opening a new tab}{\Gls{user}}{\sessionactive}{The \gls{user} opens a new tab in the \gls{browser}. The module creates an open-tab \gls{event} (as defined in \specref{D351}) and sends it to the data-managing component.}{The data-managing component received a filled-in open-tab \gls{event}.}
    \fsubspec{Switching to a tab}{\Gls{user}}{\sessionactive}{The \gls{user} switches to a tab in the \gls{browser}. The module creates a switch-tab \gls{event} (as defined in \specref{D352}) and sends it to the data-managing component.}{The data-managing component received a filled-in switch-tab \gls{event}.}
    \fsubspec{Closing a tab}{\Gls{user}}{\sessionactive}{The \gls{user} closes a tab in the \gls{browser}. The module creates a close-tab \gls{event} (as defined in \specref{D353}) and sends it to the data-managing component.}{The data-managing component received a filled-in close-tab \gls{event}.}
    \fsubspec{Navigation}{\Gls{user}}{\sessionactive}{The \gls{user} navigates to an URL. The module creates a navigation \gls{event} (as defined in \specref{D354}) and sends it to the data-managing component.}{The data-managing component received a filled-in navigation \gls{event}.}
    \fsubspec{Text input}{\Gls{user}}{\sessionactive}{The \gls{user} inputs text into a form or textbox. The module creates a text-input \gls{event} (as defined in \specref{D355}) and sends it to the data-managing component.}{The data-managing component received a filled-in text-input \gls{event}.}
    \fsubspec{Button click}{\Gls{user}}{\sessionactive}{The \gls{user} clicks a button. The module creates a button click \gls{event} (as defined in \specref{D356}) and sends it to the data-managing component.}{The data-managing component received a filled-in button-click \gls{event}.}
    \fsubspec{Hovering}{\Gls{user}}{\sessionactive}{The \gls{user} hovers the mouse-pointer over a web-element. The module creates a hover \gls{event} (as defined in \specref{D357}) and sends it to the data-managing component.}{The data-managing component received a filled-in hover \gls{event}.}
    \fsubspec{Text selection}{\Gls{user}}{\sessionactive}{The \gls{user} selects text on a website. The module creates a text-selection \gls{event} (as defined in \specref{D358}) and sends it to the data-managing component.}{The data-managing component received a filled-in text-selection \gls{event}.}
    \fsubspec{File download}{\Gls{user}}{\sessionactive}{The \gls{user} downloads a file. The module creates a download \gls{event} (as defined in \specref{D359}) and sends it to the data-managing component.}{The data-managing component received a filled-in download \gls{event}.}
    \end{itemize}
\end{itemize}