\chapter{Global test cases}
\label{ch:tests}

%%Cannot use requirementscope as we have to inject additional text into the label
\newcounter{counterTC}

\makeatletter
\newcommand{\fmt}[1]{(tests \specref{#1}\checknextarg}
\newcommand{\checknextarg}{\@ifnextchar\bgroup{\consumenextarg}{)}}
\newcommand{\consumenextarg}[1]{, \specref{#1}\@ifnextchar\bgroup{\consumenextarg}{)}}
\makeatother

\newcommand{\test}[5]{\addtocounter{counterTC}{10}
\item[TC\arabic{counterTC}\phantomsection\label{TC\arabic{counterTC}}\\\begin{footnotesize}\textit{#1}\end{footnotesize}]
    \begin{itemize}[noitemsep]
        \item[]{\textbf{#2}} %title
        \item[]{Precondition: #3}
        \item[]{Test steps: #4}
        \item[]{Expected result: #5}
    \end{itemize}
}

\newenvironment{tests}{\begin{itemize}[font = \normalfont, style = multiline, labelwidth = 60pt, leftmargin = !]}{\end{itemize}}

\section{High-level test cases}

\begin{tests}
    \test{\fmt{FS10}{FS250}{FS54}{FS70}{FS90}{FS340}{FS350}}{Identifiable sessions}{The application is running. The configuration file is valid.}
    {The \gls{user} repeats the following 256 times:
    \begin{enumerate}
        \item The \gls{user} starts a recording.
        \item The \gls{user} stops the recording.
    \end{enumerate}}
    {\begin{itemize}
        \item 256 recordings have been generated (in accordance with \ref{sec:session_recordings}) and stored at the path specified by the configuration.
        \item Each recording contains a different session identifier (as specified in \specref{D20}).
    \end{itemize}}

    \test{\fmt{FS20}{FS250}{FS54}{FS70}{FS90}{FS340}{FS350}}{Event timings}{The application is running. The configuration file is valid.}
    {\begin{enumerate}
        \item The \gls{user} starts a recording.
        \item The \gls{user} performs an action that generates an \gls{event} at time X.
        \item The \gls{user} performs another action that generates an \gls{event} at time Y.
        \item The \gls{user} stops the recording.
    \end{enumerate}}
    {\begin{itemize}
        \item A recording has been generated (in accordance with \ref{sec:session_recordings}) and stored at the path specified by the configuration.
        \item The recording contains both events in accordance with \specref{D300}, \specref{D301} and \specref{D302}. The timestamp of the first event must refer to time X, the timestamp of the second event must refer to time Y.
    \end{itemize}}

    \test{\fmt{FS40}{FS250}{FS54}{FS70}{FS90}{FS51}{FS340}{FS350}}{Event discarding}{The application is running. The configuration file is valid. The configuration specifies to discard events of type "mouse:click".}
    {\begin{enumerate}
        \item The \gls{user} starts a recording.
        \item The \gls{user} creates an \gls{event} of type "mouse:click" by clicking the left mouse button.
        \item The \gls{user} creates an \gls{event} of type "keyboard:keypress" by pressing the 'a' key on his keyboard.
        \item The \gls{user} stops the recording.
    \end{enumerate}}
    {\begin{itemize}
        \item A recording has been generated (in accordance with \ref{sec:session_recordings}) and stored at the path specified by the configuration.
        \item The recording contains only the "keyboard:keypress" event in accordance with \specref{D300}, \specref{D301}, \specref{D302} and \specref{D330}.
    \end{itemize}}

    \test{\fmt{FS40}{FS250}{FS54}{FS70}{FS90}{FS51}{FS340}{FS350}{FS411}{FS421}}{Event transforming}{The application is running. The configuration file is valid. The configuration specifies to active a \ref{transforming-module} that transforms \glspl{event} of type "mouse:click" and "keyboard:keypress(ctrl)" into an \glspl{event} of type "ctrl-leftclick".}
    {\begin{enumerate}
        \item The \gls{user} starts a recording.
        \item The \gls{user} creates an \gls{event} of type "keyboard:keypress" by pressing the 'ctrl' key on his keyboard without releasing it.
        \item The \gls{user} creates an \gls{event} of type "mouse:click" by clicking the left mouse button.
        \item The \gls{user} creates an \gls{event} of type "keyboard:keyrelease" by releasing the 'ctrl' key.
        \item The \gls{user} stops the recording.
    \end{enumerate}}
    {\begin{itemize}
        \item A recording has been generated (in accordance with \ref{sec:session_recordings}) and stored at the path specified by the configuration.
        \item The recording contains the first three events in accordance with \specref{D300}, \specref{D301},\specref{D302},\specref{D320} and \specref{D330}. It additionally contains a fourth event of type "ctrl-leftclick" that was created by transforming the first two \glspl{event}.
    \end{itemize}}

    \test{\fmt{FS51}{FS250}{FS54}{FS70}{FS90}{FS340}{FS350}{FS411}{FS421}}{Configuring active modules}{The application is running. The configuration file is valid. The configuration specifies to load the mouse module and to not load the keyboard module.}
    {\begin{enumerate}
        \item The \gls{user} starts a recording.
        \item The \gls{user} presses the left mouse button which yields an \gls{event} by the mouse module.
        \item The \gls{user} presses the 'a' key which would yield an \gls{event} by the keyboard module.
        \item The \gls{user} stops the recording.
    \end{enumerate}}
    {\begin{itemize}
        \item A recording has been generated (in accordance with \ref{sec:session_recordings}) and stored at the path specified by the configuration.
        \item The recording only contains the \gls{event} "mouse:click" event in accordance with \specref{D300}, \specref{D301} and \specref{D302}. It does not contain the \gls{event} that would have been generated by the keyboard module.
    \end{itemize}}

    \test{\fmt{FS51}{FS250}{FS54}{FS70}{FS90}{FS340}{FS350}}{Configuring video recording}{The application is running. The configuration file is valid. The configuration specifies to record video with a framerate of 60hz, a bitrate of 4096kbps and a resolution of 1920x1080 pixels.}
    {\begin{enumerate}
        \item The \gls{user} starts a recording.
        \item The \gls{user} stops the recording.
    \end{enumerate}}
    {\begin{itemize}
        \item A recording has been generated (in accordance with \ref{sec:session_recordings}) and stored at the path specified by the configuration.
        \item The videostream contained in the recording follows \specref{D40}, where the framerate is 60hz, the bitrate is 4096kpbs (or lower) and the resolution is 1920x1080 pixels.
    \end{itemize}}

    \test{\fmt{FS51}{FS250}{FS54}{FS70}{FS90}{FS340}{FS350}}{Configuring audio recording}{The application is running. The configuration file is valid. The configuration specifies to record video.}
    {\begin{enumerate}
        \item The \gls{user} starts a recording.
        \item The \gls{user} stops the recording.
    \end{enumerate}}
    {\begin{itemize}
        \item A recording has been generated (in accordance with \ref{sec:session_recordings}) and stored at the path specified by the configuration.
        \item The videostream contains the optional audiostream.
    \end{itemize}}

    \test{\fmt{FS54}{FS250}{FS70}{FS90}{FS340}{FS350}}{Configuring audio recording}{The application is running. The configuration file is valid. The configuration specifies to store the recording at path "\%userprofile\%\textbackslash documents\textbackslash recordings".}
    {\begin{enumerate}
        \item The \gls{user} starts a recording.
        \item The \gls{user} stops the recording.
    \end{enumerate}}
    {\begin{itemize}
        \item A recording has been generated (in accordance with \ref{sec:session_recordings}) and stored in the "\textbackslash documents\textbackslash recordings" folder inside the user directory of the user currently logged in.
    \end{itemize}}

    \test{\fmt{FS60}{FS310}}{Error detection}{The default-configuration file got invalidated by appending a bracket ' \{ ' to the end of the file.}
    {\begin{enumerate}
        \item The \gls{user} starts the application.
    \end{enumerate}}
    {\begin{itemize}
        \item The application detects the errors in the configuration and displays an error dialogue.
    \end{itemize}}

    \test{\fmt{FS70}{FS80}{FS340}{FS350}}{User controllability (start)}{The application is running. The configuration file is valid.}
    {\begin{enumerate}
        \item The \gls{user} opens the tray menu by clicking on the application icon in the system tray to open the tray menu.
        \item The \gls{user} clicks on the "Start Recording" entry.
    \end{enumerate}}
    {\begin{itemize}
        \item The application shows an information dialogue.
    \end{itemize}}
    
     \test{\fmt{FS70}{FS80}{FS300}}{Information dialogue}{The application currently shows the information dialogue at the start of a recording.}
    {\begin{enumerate}
        \item The user dismisses the information dialogue.
    \end{enumerate}}
    {\begin{itemize}
        \item The application starts a recording and shows a recording indicator.
    \end{itemize}}
    
\end{tests}

\section{Per-module test cases}

\subsection{System module test cases}

\subsubsection{Window-management module test cases}

\begin{tests}
	\test{\fmt{FS70}{FS54}{FS20}{FS30}{FS250}{FS401}}{Window focus event}{The application is running. The configuration file is valid.}
	{\begin{enumerate}
		\item The \gls{user} starts a new session recording.
		\item The \gls{user} focuses a "Windows Explorer" window.
		\item The \gls{user} stops the recording.
	\end{enumerate}}
	{\begin{itemize}
		\item A recording has been created and stored at the path specified by the configuration.
		\item The recording contains a focus-window event, that contains the title of the "Windows Explorer" window, along with the timestamp at which the user focused the window, the identifier of the window-management module and the "window:focus" type of the event.
	\end{itemize}}
	
	\test{\fmt{FS70}{FS54}{FS20}{FS30}{FS250}{FS402}}{Window move event}{The application is running. The configuration file is valid.}
	{\begin{enumerate}
		\item The \gls{user} starts a new session recording.
		\item The \gls{user} moves a "Windows Explorer" window.
		\item The \gls{user} stops the recording.
	\end{enumerate}}
	{\begin{itemize}
		\item A recording has been created and stored at the path specified by the configuration.
		\item The recording contains a move-window event, that contains the title of the "Windows Explorer" window, old and new location of the window along with the timestamp at which the user moved the window, the identifier of the window-management module and the "window:move" type of the event.
	\end{itemize}}
	
	\test{\fmt{FS70}{FS54}{FS20}{FS30}{FS250}{FS403}}{Window resize event}{The application is running. The configuration file is valid.}
	{\begin{enumerate}
		\item The \gls{user} starts a new session recording.
		\item The \gls{user} resizes a "Windows Explorer" window.
		\item The \gls{user} stops the recording.
	\end{enumerate}}
	{\begin{itemize}
		\item A recording has been created and stored at the path specified by the configuration.
		\item The recording contains a resize-window event, that contains the title of "Windows Explorer" window, old and new size of the window along with the timestamp at which the user resized the window, the identifier of the window-management module and the "window:resize" type of the event.
	\end{itemize}}
	
	\test{\fmt{FS70}{FS54}{FS20}{FS30}{FS250}{FS404}}{Window maximize/minimize/restore event}{The application is running. The configuration file is valid.}
	{\begin{enumerate}
		\item The \gls{user} starts a new session recording.
		\item The \gls{user} maximizes/minimizes/restores a "Windows Explorer" window.
		\item The \gls{user} stops the recording.
	\end{enumerate}}
	{\begin{itemize}
		\item A recording has been created and stored at the path specified by the configuration.
		\item The recording contains a maximize-window/minimize-window/restore-window event, that contains the title of "Windows Explorer" window, the new state of the window - "maximized"/"minimized"/"restored" along with the timestamp at which the user maximized/minimized/restored the window, the identifier of the window-management module and the "window:maximize"/"window:minimize"/"window:restore" type of the event.
	\end{itemize}}
\end{tests}

\subsubsection{Mouse module test cases}

\begin{tests}
	\test{\fmt{FS20}{FS30}{FS54}{FS70}{FS250}{FS411}}{Mouse click event}{The application is running. The configuration file is valid.}
	{\begin{enumerate}
		\item The \gls{user} starts a new session recording.
		\item The \gls{user} presses the left/right/middle mouse button in "Windows Explorer".
		\item The \gls{user} releases the left/right/middle mouse button in "Windows Explorer".
		\item The \gls{user} stops the recording.
	\end{enumerate}}
	{\begin{itemize}
		\item A recording has been created and stored at the path specified by the configuration.
		\item The recording contains a click-mouse event, that contains type of the clicked button - left/right/middle, type of the click - single, HWND associated with "Windows Explorer" window that was clicked, along with the timestamp at which the user clicked the mouse button, the identifier of the mouse-interaction module and the "mouse:left-click"/"mouse:right-click"/"mouse:middle-click" type of the event.
	\end{itemize}}
	
	\test{\fmt{FS20}{FS30}{FS54}{FS70}{FS250}{FS411}}{Mouse double click event}{The application is running. The configuration file is valid.}
	{\begin{enumerate}
		\item The \gls{user} starts a new session recording.
		\item The \gls{user} performs double-clicks the left mouse button.
		\item The \gls{user} stops the recording.
	\end{enumerate}}
	{\begin{itemize}
		\item A recording has been created and stored at the path specified by the configuration.
		\item The recording contains a double-click-mouse event, that contains type of the clicked button - left, type of the click - double, HWND associated with the "Windows Explorer" window that was clicked, along with the timestamp at which the user double clicked the mouse button, the identifier of the mouse-interaction module and the "mouse:left-double-click" type of the event.
	\end{itemize}}
	
	\test{\fmt{FS20}{FS30}{FS54}{FS70}{FS250}{FS412}}{Mouse scroll event}{The application is running. The configuration file is valid.}
	{\begin{enumerate}
		\item The \gls{user} starts a new session recording.
		\item The \gls{user} scrolls the mouse wheel in "Windows Explorer".
		\item The \gls{user} stops the recording.
	\end{enumerate}}
	{\begin{itemize}
		\item A recording has been created and stored at the path specified by the configuration.
		\item The recording contains a scroll-mouse event, that contains scroll amount, HWND associated with the "Windows Explorer" window that was scrolled, along with the timestamp at which the user used the scroll wheel, the identifier of the mouse-interaction module and the "mouse:scroll" type of the event.
	\end{itemize}}
	
	\test{\fmt{FS20}{FS30}{FS54}{FS70}{FS250}{FS413}}{Mouse move event}{The application is running. The configuration file is valid.}
	{\begin{enumerate}
		\item The \gls{user} starts a new session recording.
		\item The \gls{user} moves the mouse.
		\item The \gls{user} stops the recording.
	\end{enumerate}}
	{\begin{itemize}
		\item A recording has been created and stored at the path specified by the configuration.
		\item The recording contains a move-mouse event, that contains movement vector along with the timestamp at which the user moved the mouse, the identifier of the mouse-interaction module and the "mouse:move" type of the event.
	\end{itemize}}
\end{tests}

\subsubsection{Keyboard module test cases}

\begin{tests}
	\test{\fmt{FS20}{FS30}{FS54}{FS70}{FS250}{FS421}}{Keyboard event}{The application is running. The configuration file is valid.}
	{\begin{enumerate}
		\item The \gls{user} starts a new session recording.
		\item The \gls{user} presses a key while a "Windows Explorer" window is focused.
		\item The \gls{user} releases a key while a "Windows Explorer" window is focused.
		\item The \gls{user} stops the recording.
	\end{enumerate}}
	{\begin{itemize}
		\item A recording has been created and stored at the path specified by the configuration.
		\item The recording contains press-key-keyboard event and release-key-keyboard event. Each event contains name of the pressed/released key, names of all modifier keys that are pressed/released along with the timestamp at which the user pressed/released a key, the identifier of the keyboard-interaction module and the "keyboard:press-key"/"keyboard:release-key" type of the event.
	\end{itemize}}
\end{tests}

\subsubsection{Clipboard module test cases}
	
\begin{tests}
	\test{\fmt{FS20}{FS30}{FS54}{FS70}{FS250}{FS431}}{Clipboard copy event}{The application is running. The configuration file is valid.}
	{\begin{enumerate}
		\item The \gls{user} starts a new session recording.
		\item The \gls{user} copies text from a "Microsoft Excel" window.
		\item The \gls{user} stops the recording.
	\end{enumerate}}
	{\begin{itemize}
		\item A recording has been created and stored at the path specified by the configuration.
		\item The recording contains copy-clipboard event, that contains clipboard text associated with "Microsoft Excel", type of the clipboard interaction - copy, along with the timestamp at which the user copied the text to the clipboard, the identifier of the clipboard-interaction module and the "clipboard:copy" type of the event.
	\end{itemize}}
	
	\test{\fmt{FS20}{FS30}{FS54}{FS70}{FS250}{FS432}}{Clipboard paste event}{The application is running. The configuration file is valid.}
	{\begin{enumerate}
		\item The \gls{user} starts a new session recording.
		\item The \gls{user} pastes text in the "Mozilla Firefox/Google Chrome" address bar.
		\item The \gls{user} stops the recording.
	\end{enumerate}}
	{\begin{itemize}
		\item A recording has been created and stored at the path specified by the configuration.
		\item The recording contains paste-clipboard event, that contains clipboard text associated with "Mozilla Firefox/Google Chrome", type of the clipboard interaction - paste, along with the timestamp at which the user pasted the text from the clipboard, the identifier of the clipboard-interaction module and the "clipboard:paste" type of the event.
	\end{itemize}}
\end{tests}

\subsection{On-demand module test cases}

\subsubsection{Browser module test cases}
\begin{tests}
	\test{\fmt{FS20}{FS30}{FS54}{FS70}{FS250}{FS441}}{Browser open tab event}{The application is running. The configuration file is valid.}
	{\begin{enumerate}
		\item The \gls{user} starts a new session recording.
		\item The \gls{user} opens a new tab in "Mozilla Firefox/Google Chrome".
		\item The \gls{user} stops the recording.
	\end{enumerate}}
	{\begin{itemize}
		\item A recording has been created and stored at the path specified by the configuration.
		\item The recording contains open-tab-browser event, that contains URL of the interacted website and the unique identifier of the new tab along with the timestamp at which the user opened a new tab in the "Mozilla Firefox/Google Chrome", the identifier of the browser module and the "browser:open-tab" type of the event.
	\end{itemize}}
	
	\test{\fmt{FS20}{FS30}{FS54}{FS70}{FS250}{FS442}}{Browser switch to tab event}{The application is running. The configuration file is valid.}
	{\begin{enumerate}
		\item The \gls{user} starts a new session recording.
		\item The \gls{user} switches to a tab in "Mozilla Firefox/Google Chrome".
		\item The \gls{user} stops the recording.
	\end{enumerate}}
	{\begin{itemize}
		\item A recording has been created and stored at the path specified by the configuration.
		\item The recording contains switch-tab-browser event, that contains URL of the interacted website, the unique identifier of the old tab and the unique identifier of the tab that the user switched to along with the timestamp at which the user switched to a tab "Mozilla Firefox/Google Chrome", the identifier of the browser module and the "browser:switch-tab" type of the event.
	\end{itemize}}
	
	\test{\fmt{FS20}{FS30}{FS54}{FS70}{FS250}{FS443}}{Browser close tab event}{The application is running. The configuration file is valid.}
	{\begin{enumerate}
		\item The \gls{user} starts a new session recording.
		\item The \gls{user} closes a tab "Mozilla Firefox/Google Chrome".
		\item The \gls{user} stops the recording.
	\end{enumerate}}
	{\begin{itemize}
		\item A recording has been created and stored at the path specified by the configuration.
		\item The recording contains close-tab-browser event, that contains URL of the interacted website and the unique identifier of the tab along with the timestamp at which the user closed a tab in "Mozilla Firefox/Google Chrome", the identifier of the browser module and the "browser:close-tab" type of the event.
	\end{itemize}}
	
	\test{\fmt{FS20}{FS30}{FS54}{FS70}{FS250}{FS444}}{Browser navigation event}{The application is running. The configuration file is valid.}
	{\begin{enumerate}
		\item The \gls{user} starts a new session recording.
		\item The \gls{user} navigates to an URL on Google Search Engine Results Pages.
		\item The \gls{user} stops the recording.
	\end{enumerate}}
	{\begin{itemize}
		\item A recording has been created and stored at the path specified by the configuration.
		\item The recording contains navigation-browser event, that contains URL of the Google Search Engine Results Pages and the unique identifier of the tab, where the navigation event occured in, along with the timestamp at which the user navigated to an URL, the identifier of the browser module and the "browser:navigation" type of the event.
	\end{itemize}}
	
	\test{\fmt{FS20}{FS30}{FS54}{FS70}{FS250}{FS445}}{Browser text input event}{The application is running. The configuration file is valid.}
	{\begin{enumerate}
		\item The \gls{user} starts a new session recording.
		\item The \gls{user} inputs a text into a search box on Google Search Engine Results Pages.
		\item The \gls{user} stops the recording.
	\end{enumerate}}
	{\begin{itemize}
		\item A recording has been created and stored at the path specified by the configuration.
		\item The recording contains text-input-browser event, that contains URL of Google Search Engine Results Pages, the unique identifier of the tab, where the text-input event occured in, text which has been inputted by the user, information about search box, where text has been inputted, along with the timestamp at which the user inputted the text, the identifier of the browser module and the "browser:text-input" type of the event.
	\end{itemize}}
	
	
	\test{\fmt{FS20}{FS30}{FS54}{FS70}{FS250}{FS446}}{Browser button click event}{The application is running. The configuration file is valid.}
	{\begin{enumerate}
		\item The \gls{user} starts a new session recording.
		\item The \gls{user} clicks a button on Google Search Engine Results Pages.
		\item The \gls{user} stops the recording.
	\end{enumerate}}
	{\begin{itemize}
		\item A recording has been created and stored at the path specified by the configuration.
		\item The recording contains button-click-browser event, that contains URL of Google Search Engine Results Pagese, the unique identifier of the tab, where the button-click event occured in, title of the button item, URL of the website that the button is linked to, if applicable) along with the timestamp at which the user clicked a button on Google Search Engine Results Pages, the identifier of the browser module and the "browser:button-click" type of the event.
	\end{itemize}}
	
	
	\test{\fmt{FS20}{FS30}{FS54}{FS70}{FS250}{FS447}}{Browser hovering event}{The application is running. The configuration file is valid.}
	{\begin{enumerate}
		\item The \gls{user} starts a new session recording.
		\item The \gls{user} hovers the mouse-pointer over a web-element on Google Search Engine Results Pages.
		\item The \gls{user} stops the recording.
	\end{enumerate}}
	{\begin{itemize}
		\item A recording has been created and stored at the path specified by the configuration.
		\item The recording contains hover-browser event, that contains URL of Google Search Engine Results Pages, the unique identifier of the tab, where the hover event occured in, information about the element which has been hovered along with the timestamp at which the user hovered the web-element, the identifier of the browser module and the "browser:hover" type of the event.
	\end{itemize}}
	
	
	\test{\fmt{FS20}{FS30}{FS54}{FS70}{FS250}{FS448}}{Browser text selection event}{The application is running. The configuration file is valid.}
	{\begin{enumerate}
		\item The \gls{user} starts a new session recording.
		\item The \gls{user} selects a text on Google Search Engine Results Pages.
		\item The \gls{user} stops the recording.
	\end{enumerate}}
	{\begin{itemize}
		\item A recording has been created and stored at the path specified by the configuration.
		\item The recording contains text-selection-browser event, that contains URL of Google Search Engine Results Pages, the unique identifier of the tab, where the text-selection event occured in, selected text along with the timestamp at which the user selected the text, the identifier of the browser module and the "browser:text-selection" type of the event.
	\end{itemize}}
	
	
	\test{\fmt{FS20}{FS30}{FS54}{FS70}{FS250}{FS449}}{Browser file download event}{The application is running. The configuration file is valid.}
	{\begin{enumerate}
		\item The \gls{user} starts a new session recording.
		\item The \gls{user} downloads a file from Google Search Engine Results Pages.
		\item The \gls{user} stops the recording.
	\end{enumerate}}
	{\begin{itemize}
		\item A recording has been created and stored at the path specified by the configuration.
		\item The recording contains download-browser event, that contains URL of Google Search Engine Results Pages, the unique identifier of the tab, where the download event occured in, URL of the file that was downloaded along with the timestamp at which the user downloaded a file, the identifier of the browser module and the "browser:download" type of the event.
	\end{itemize}}
	
\end{tests}
