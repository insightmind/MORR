\chapter{Global test cases}
\label{ch:tests}

%%Cannot use requirementscope as we have to inject additional text into the label
\newcounter{counterTC}

\makeatletter
\newcommand{\fmt}[1]{(tests \specref{#1}\checknextarg}
\newcommand{\checknextarg}{\@ifnextchar\bgroup{\consumenextarg}{)}}
\newcommand{\consumenextarg}[1]{, \specref{#1}\@ifnextchar\bgroup{\consumenextarg}{)}}
\makeatother

\newcommand{\test}[5]{\addtocounter{counterTC}{10}
\item[TC\arabic{counterTC}\phantomsection\label{TC\arabic{counterTC}}\\\begin{footnotesize}\textit{#1}\end{footnotesize}]
    \begin{itemize}[noitemsep]
        \item[]{\textbf{#2}} %title
        \item[]{Precondition: #3}
        \item[]{Test steps: #4}
        \item[]{Expected result: #5}
    \end{itemize}
}

\newenvironment{tests}{\begin{itemize}[font = \normalfont, style = multiline, labelwidth = 60pt, leftmargin = !]}{\end{itemize}}

\section{High-level test cases}

\begin{tests}
    \test{\fmt{FS10}{FS54}{FS70}{FS250}{FS90}{FS340}{FS350}}{Identifiable sessions}{The application is running. The configuration file is valid.}
    {The \gls{user} repeats the following 256 times:
    \begin{enumerate}
        \item The \gls{user} starts a recording.
        \item The \gls{user} finalizes the recording.
    \end{enumerate}}
    {\begin{itemize}
        \item 256 recordings have been generated (in accordance with \ref{sec:session_recordings}) and stored at the path specified by the configuration.
        \item Each recording contains a different \gls{session} identifier (as specified in \specref{D20}).
    \end{itemize}}

    \test{\fmt{FS40}{FS54}{FS70}{FS250}{FS90}{FS51}{FS340}{FS350}}{Event discarding}{The application is running. The configuration file is valid. The configuration specifies to discard events of type "mouse:click".}
    {\begin{enumerate}
        \item The \gls{user} starts a recording.
        \item The \gls{user} creates an \gls{event} of type "mouse:left-click" by clicking the left mouse button.
        \item The \gls{user} creates an \gls{event} of type "keyboard:press-key" by pressing the 'a' key.
        \item The \gls{user} finalizes the recording.
    \end{enumerate}}
    {\begin{itemize}
        \item A recording has been generated (in accordance with \ref{sec:session_recordings}) and stored at the path specified by the configuration.
        \item The recording contains only the "keyboard:press-key" \gls{event} in accordance with \specref{D300}, \specref{D301}, \specref{D302} and \specref{D330}.
    \end{itemize}}

    \test{\fmt{FS40}{FS51}{FS54}{FS70}{FS90}{FS250}{FS340}{FS350}{FS411}{FS421}}{Event transforming}{The application is running. The configuration file is valid. The configuration specifies to active a \ref{transforming-module} that transforms \glspl{event} of type "mouse:left-click" and "keyboard:press-key" into an \glspl{event} of type "other:ctrl-leftclick".}
    {\begin{enumerate}
        \item The \gls{user} starts a recording.
        \item The \gls{user} creates an \gls{event} of type "keyboard:press-key" by pressing the 'CTRL' key without releasing it.
        \item The \gls{user} creates an \gls{event} of type "mouse:left-click" by clicking the left mouse button.
        \item The \gls{user} creates an \gls{event} of type "keyboard:release-key" by releasing the 'CTRL' key.
        \item The \gls{user} finalizes the recording.
    \end{enumerate}}
    {\begin{itemize}
        \item A recording has been generated (in accordance with \ref{sec:session_recordings}) and stored at the path specified by the configuration.
        \item The recording contains the first three events in accordance with \specref{D300}, \specref{D301}, \specref{D302}, \specref{D320} and \specref{D330}. It additionally contains a fourth \gls{event} of type "other:ctrl-leftclick" that was created by transforming the first two \glspl{event}.
    \end{itemize}}

    \test{\fmt{FS51}{FS54}{FS70}{FS90}{FS250}{FS340}{FS350}{FS411}{FS421}}{Configuring active modules}{The application is running. The configuration file is valid. The configuration specifies to load the mouse module and to not load the keyboard module.}
    {\begin{enumerate}
        \item The \gls{user} starts a recording.
        \item The \gls{user} presses the left mouse button which yields an \gls{event} by the mouse module.
        \item The \gls{user} presses the 'a' key which would yield an \gls{event} by the keyboard module.
        \item The \gls{user} finalizes the recording.
    \end{enumerate}}
    {\begin{itemize}
        \item A recording has been generated (in accordance with \ref{sec:session_recordings}) and stored at the path specified by the configuration.
        \item The recording only contains the \gls{event} "mouse:left-click" \gls{event} in accordance with \specref{D300}, \specref{D301} and \specref{D302}. It does not contain the \gls{event} that would have been generated by the keyboard module.
    \end{itemize}}

    \test{\fmt{FS51}{FS54}{FS70}{FS90}{FS250}{FS340}{FS350}}{Configuring video recording}{The application is running. The configuration file is valid. The configuration specifies to record video with a framerate of 60hz, a bitrate of 4096kbps and a resolution of 1920x1080 pixels.}
    {\begin{enumerate}
        \item The \gls{user} starts a recording.
        \item The \gls{user} finalizes the recording.
    \end{enumerate}}
    {\begin{itemize}
        \item A recording has been generated (in accordance with \ref{sec:session_recordings}) and stored at the path specified by the configuration.
        \item The \gls{videostream} contained in the recording follows \specref{D40}, where the framerate is 60hz, the bitrate is 4096kpbs (or lower) and the resolution is 1920x1080 pixels.
    \end{itemize}}

    \test{\fmt{FS51}{FS54}{FS70}{FS90}{FS250}{FS340}{FS350}}{Configuring audio recording}{The application is running. The configuration file is valid. The configuration specifies to record audio.}
    {\begin{enumerate}
        \item The \gls{user} starts a recording.
        \item The \gls{user} finalizes the recording.
    \end{enumerate}}
    {\begin{itemize}
        \item A recording has been generated (in accordance with \ref{sec:session_recordings}) and stored at the path specified by the configuration.
        \item The \gls{videostream} contains the optional audiostream.
    \end{itemize}}

    \test{\fmt{FS54}{FS70}{FS90}{FS250}{FS340}{FS350}}{Configuring file path}{The application is running. The configuration file is valid. The configuration specifies to store the recording at path "\%userprofile\%\textbackslash documents\textbackslash recordings".}
    {\begin{enumerate}
        \item The \gls{user} starts a recording.
        \item The \gls{user} finalizes the recording.
    \end{enumerate}}
    {\begin{itemize}
        \item A recording has been generated (in accordance with \ref{sec:session_recordings}) and stored in the "\textbackslash documents\textbackslash recordings" folder inside the \gls{user} directory of the \gls{user} currently logged in.
    \end{itemize}}

    \test{\fmt{FS60}{FS310}}{Error detection}{The default-configuration file got invalidated by appending a bracket ' \{ ' to the end of the file.}
    {\begin{enumerate}
        \item The \gls{user} starts the application.
    \end{enumerate}}
    {\begin{itemize}
        \item The application detects the errors in the configuration and shows an error dialog.
    \end{itemize}}

    \test{\fmt{FS70}{FS80}{FS340}{FS350}}{User controllability (start)}{The application is running. The configuration file is valid.}
    {\begin{enumerate}
        \item The \gls{user} opens the tray menu by clicking on the application icon in the system tray to open the tray menu.
        \item The \gls{user} clicks on the "Start Recording" entry.
    \end{enumerate}}
    {\begin{itemize}
        \item The application shows an information dialog.
    \end{itemize}}
    
     \test{\fmt{FS70}{FS80}{FS300}}{Information dialog}{The application is currently showing the information dialog at the start of a recording.}
    {\begin{enumerate}
        \item The \gls{user} dismisses the information dialog.
    \end{enumerate}}
    {\begin{itemize}
        \item The application starts a recording and shows a recording indicator.
    \end{itemize}}
    
    \test{\fmt{FS70}{FS90}{FS340}{FS350}}{User controllability (stop)}{The application is running. The configuration file is valid. A \gls{session} is currently being recorded.}
    {\begin{enumerate}
        \item The \gls{user} opens the tray menu by clicking on the application icon in the system tray to open the tray menu.
        \item The \gls{user} clicks on the "Stop Recording" entry.
    \end{enumerate}}
    {\begin{itemize}
        \item The recording is stopped.
        \item The application shows a save dialog.
    \end{itemize}}

    \test{\fmt{FS100}{FS340}{FS350}}{User controllability (quit)}{The application is running. No \gls{session} is currently being recorded.}
    {\begin{enumerate}
        \item The \gls{user} opens the tray menu by clicking on the application icon in the system tray to open the tray menu.
        \item The \gls{user} clicks on the "Quit" entry.
    \end{enumerate}}
    {\begin{itemize}
        \item The application is no longer running.
    \end{itemize}}

    \test{\fmt{FS90}{FS110}{FS340}{FS350}}{Quit while recording}{The application is running. A \gls{session} is currently being recorded.}
    {\begin{enumerate}
        \item The \gls{user} opens the tray menu by clicking on the application icon in the system tray to open the tray menu.
        \item The \gls{user} clicks on the "Quit" entry.
        \item The \gls{user} chooses any option in the save dialog.
    \end{enumerate}}
    {\begin{itemize}
        \item The application terminates immediately if the \gls{user} decided to discard the recording or terminates after the recording has been saved if the \gls{user} decided to save the recording.
    \end{itemize}}
        
    \test{\fmt{FS120}{FS250}}{Save dialog - Save}{The application is currently showing the save dialog.}
    {\begin{enumerate}
        \item The \gls{user} clicks on the button labeled "Save".
    \end{enumerate}}
    {\begin{itemize}
        \item A recording has been generated (in accordance with \ref{sec:session_recordings}) and stored at the path specified by the configuration.
    \end{itemize}}

    \test{\fmt{FS120}{FS250}}{Save dialog - Discard}{The application is currently showing the save dialog.}
    {\begin{enumerate}
        \item The \gls{user} clicks on the button labeled "Discard".
    \end{enumerate}}
    {\begin{itemize}
        \item The application is no longer running.
    \end{itemize}}

    \test{\fmt{FS130}}{Recordings folder}{The application is running.}
    {\begin{enumerate}
        \item The \gls{user} opens the tray menu by clicking on the application icon in the system tray to open the tray menu.
        \item The \gls{user} clicks on the "Open recordings folder" entry.
    \end{enumerate}}
    {\begin{itemize}
        \item The folder containing the \gls{user}'s recordings is opened in the Windows File Explorer.
    \end{itemize}}
\end{tests}

\section{Core application test cases}
\setcounter{counterTC}{190}
\begin{tests}
    \test{\fmt{FS70}{FS90}{FS200}{FS250}}{Video stream recording}{The application is running. The configuration file is valid.}
    {\begin{enumerate}
        \item The \gls{user} starts a recording.
        \item The \gls{user} stops a recording.
    \end{enumerate}}
    {\begin{itemize}
        \item A recording has been generated (in accordance with \ref{sec:session_recordings}) and stored at the path specified by the configuration.
        \item The recording contains a \gls{videostream}. The \gls{videostream} contains the \gls{device}'s primary screen's content.
    \end{itemize}}

    \test{\fmt{FS70}{FS90}{FS210}{FS220}{FS250}}{System module management}{The application is running. The configuration file is valid. The configuration specifies that the mouse module should be activated.}
    {\begin{enumerate}
        \item The \gls{user} starts a recording.
        \item The \gls{user} clicks the left mouse button.
        \item The \gls{user} finalizes the recording.
    \end{enumerate}}
    {\begin{itemize}
        \item A recording has been generated (in accordance with \ref{sec:session_recordings}) and stored at the path specified by the configuration.
        \item The recording contains an \gls{event} of type "mouse:left-click" corresponding to the \gls{user} interaction.
    \end{itemize}}

    \test{\fmt{FS70}{FS90}{FS230}{FS240}{FS250}}{On-demand module management}{The application is running. The configuration file is valid. The configuration specifies that the \gls{browser} module should be activated.}
    {\begin{enumerate}
        \item The \gls{user} starts a recording.
        \item The \gls{user} opens the \gls{browser}.
        \item The \gls{user} navigates to the civic.genome.wustl.edu webpage.
        \item The \gls{user} finalizes the recording.
    \end{enumerate}}
    {\begin{itemize}
        \item A recording has been generated (in accordance with \ref{sec:session_recordings}) and stored at the path specified by the configuration.
        \item The recording contains an \gls{event} of type "browser:navigate" corresponding to the \gls{user} interaction.
    \end{itemize}}

    \test{\fmt{FS70}{FS90}{FS250}}{Data serialization}{The application is running. The configuration file is valid. The configuration specifies that both the mouse module and the keyboard module should be activated.}
    {\begin{enumerate}
        \item The \gls{user} starts a recording.
        \item The \gls{user} clicks the left mouse button.
        \item The \gls{user} presses the "a"-key.
        \item The \gls{user} finalizes the recording.
    \end{enumerate}}
    {\begin{itemize}
        \item A recording has been generated (in accordance with \ref{sec:session_recordings}) and stored at the path specified by the configuration.
        \item The recording contains both the "mouse:left-click" and the "keyboard:press-key" events.
    \end{itemize}}

    \test{\fmt{FS70}{FS90}{FS250}{FS260}}{Combinability of modules}{The application is running. The configuration file is valid. The configuration specifies that both the mouse module and the keyboard module should be activated. It also specifies another module, that merges a "mouse:left-click" and a "keyboard:press-key" \gls{event} into a "other:ctrl-click" event.}
    {\begin{enumerate}
        \item The \gls{user} starts a recording.
        \item The \gls{user} presses and holds the "CTRL"-key.
        \item The \gls{user} then clicks the left mouse button.
        \item The \gls{user} finalizes the recording.
    \end{enumerate}}
    {\begin{itemize}
        \item A recording has been generated (in accordance with \ref{sec:session_recordings}) and stored at the path specified by the configuration.
        \item The recording contains a "other:ctrl-click" event.
    \end{itemize}}

    \test{\fmt{FS70}{FS90}{FS250}{FS260}}{Context management}{The application is running. The configuration file is valid. The configuration specifies that the \gls{browser} module should be activated. It also specifies that the \gls{browser} module may only be active when the \gls{user} uses the \gls{browser}.}
    {\begin{enumerate}
        \item The \gls{user} starts a recording.
        \item The \gls{user} starts navigating to the civic.genome.wustl.edu webpage.
        \item The \gls{user} then immediately switches to another application.
        \item The \gls{user} finalizes the recording.
    \end{enumerate}}
    {\begin{itemize}
        \item A recording has been generated (in accordance with \ref{sec:session_recordings}) and stored at the path specified by the configuration.
        \item The recording contains a "browser:navigation" \gls{event} but does not contain any \gls{event} connected to the network requests the \gls{browser} made while the \gls{user} did not have the \gls{browser} focused.
    \end{itemize}}
\end{tests}

\section{User interface test cases}

\begin{tests}
\setcounter{counterTC}{290}
    \test{\fmt{FS330}}{Event-Data extraction}{One or more recordings have been created and stored at the path "\%userprofile\%\textbackslash documents\textbackslash recordings\textbackslash sample.mp4".}
    {\begin{enumerate}
        \item The \gls{scientist} starts the command-line tool with the path "\%userprofile\%\textbackslash documents\textbackslash recordings\textbackslash" of the recordings and another path "\%userprofile\%\textbackslash documents\textbackslash recordings\textbackslash extracted\textbackslash" to extract the data to.
    \end{enumerate}}
    {\begin{itemize}
        \item The \gls{event} data contained in the recordings at path "\%userprofile\%\textbackslash documents\textbackslash recordings\textbackslash" is extracted in CSV-format to path "\%userprofile\%\textbackslash documents\textbackslash recordings\textbackslash extracted\textbackslash".
    \end{itemize}}

    \test{\fmt{FS340}}{Tray-Icon}{The application is not running. The configuration file is valid.}
    {\begin{enumerate}
        \item The \gls{user} starts the application.
    \end{enumerate}}
    {\begin{itemize}
        \item The application icon is added to the system tray.
    \end{itemize}}

    \test{\fmt{FS340}{FS350}}{Tray-Menu}{The application is running.}
    {\begin{enumerate}
        \item The \gls{user} clicks the tray-icon.
    \end{enumerate}}
    {\begin{itemize}
        \item The tray-menu is shown.
    \end{itemize}}
\end{tests}

\section{Per-module test cases}

\subsection{System module test cases}

\subsubsection{Window-management module test cases}

\begin{tests}
	\test{\fmt{FS20}{FS30}{FS54}{FS70}{FS250}{FS401}}{Window focus event}{The application is running. The configuration file is valid.}
	{\begin{enumerate}
		\item The \gls{user} starts a new \gls{session} recording.
		\item The \gls{user} focuses a "Windows Explorer" window.
		\item The \gls{user} finalizes the recording.
	\end{enumerate}}
	{\begin{itemize}
		\item A recording has been created and stored at the path specified by the configuration.
		\item The recording contains a focus-window \gls{event},  that contains the title of the "Windows Explorer" window, along with the timestamp at which the \gls{user} focused the window, the identifier of the window-management module and the "window:focus" type of the event.
	\end{itemize}}
	
	\test{\fmt{FS20}{FS30}{FS54}{FS70}{FS250}{FS402}}{Window move event}{The application is running. The configuration file is valid.}
	{\begin{enumerate}
		\item The \gls{user} starts a new \gls{session} recording.
		\item The \gls{user} moves a "Windows Explorer" window.
		\item The \gls{user} finalizes the recording.
	\end{enumerate}}
	{\begin{itemize}
		\item A recording has been created and stored at the path specified by the configuration.
		\item The recording contains a move-window \gls{event},  that contains the title of the "Windows Explorer" window, old and new location of the window along with the timestamp at which the \gls{user} moved the window, the identifier of the window-management module and the "window:move" type of the event.
	\end{itemize}}
	
	\test{\fmt{FS20}{FS30}{FS54}{FS70}{FS250}{FS403}}{Window resize event}{The application is running. The configuration file is valid.}
	{\begin{enumerate}
		\item The \gls{user} starts a new \gls{session} recording.
		\item The \gls{user} resizes a "Windows Explorer" window.
		\item The \gls{user} finalizes the recording.
	\end{enumerate}}
	{\begin{itemize}
		\item A recording has been created and stored at the path specified by the configuration.
		\item The recording contains a resize-window \gls{event},  that contains the title of the "Windows Explorer" window, old and new size of the window along with the timestamp at which the \gls{user} resized the window, the identifier of the window-management module and the "window:resize" type of the event.
	\end{itemize}}
	
	\test{\fmt{FS20}{FS30}{FS54}{FS70}{FS250}{FS404}}{Window maximize/minimize/restore event}{The application is running. The configuration file is valid.}
	{\begin{enumerate}
		\item The \gls{user} starts a new \gls{session} recording.
		\item The \gls{user} maximizes/minimizes/restores a "Windows Explorer" window.
		\item The \gls{user} finalizes the recording.
	\end{enumerate}}
	{\begin{itemize}
		\item A recording has been created and stored at the path specified by the configuration.
		\item The recording contains a maximize-window/minimize-window/restore-window \gls{event},  that contains the title of the "Windows Explorer" window, the new state of the window - "maximized"/"minimized"/"restored" along with the timestamp at which the \gls{user} maximized/minimized/restored the window, the identifier of the window-management module and the "window:maximize"/"window:minimize"/"window:restore" type of the event.
	\end{itemize}}
\end{tests}

\subsubsection{Mouse module test cases}

\begin{tests}
	\test{\fmt{FS20}{FS30}{FS54}{FS70}{FS250}{FS411}}{Mouse click event}{The application is running. The configuration file is valid.}
	{\begin{enumerate}
		\item The \gls{user} starts a new \gls{session} recording.
		\item The \gls{user} presses the left/right/middle mouse button in "Windows Explorer".
		\item The \gls{user} releases the left/right/middle mouse button in "Windows Explorer".
		\item The \gls{user} finalizes the recording.
	\end{enumerate}}
	{\begin{itemize}
		\item A recording has been created and stored at the path specified by the configuration.
		\item The recording contains a mouse click \gls{event},  that contains type of the clicked button - left/right/middle, type of the click - single, the HWND associated with the "Windows Explorer" window that was clicked, along with the timestamp at which the \gls{user} clicked the mouse button, the identifier of the mouse-interaction module and the "mouse:left-click"/"mouse:right-click"/"mouse:middle-click" type of the event.
	\end{itemize}}
	
	\test{\fmt{FS20}{FS30}{FS54}{FS70}{FS250}{FS411}}{Mouse double click event}{The application is running. The configuration file is valid.}
	{\begin{enumerate}
		\item The \gls{user} starts a new \gls{session} recording.
		\item The \gls{user} performs double-clicks the left mouse button.
		\item The \gls{user} finalizes the recording.
	\end{enumerate}}
	{\begin{itemize}
		\item A recording has been created and stored at the path specified by the configuration.
		\item The recording contains a double-click-mouse \gls{event},  that contains the type of the clicked button - left, type of the click - double, HWND associated with the "Windows Explorer" window that was clicked, along with the timestamp at which the \gls{user} double clicked the mouse button, the identifier of the mouse-interaction module and the "mouse:left-double-click" type of the event.
	\end{itemize}}
	
	\test{\fmt{FS20}{FS30}{FS54}{FS70}{FS250}{FS412}}{Mouse scroll event}{The application is running. The configuration file is valid.}
	{\begin{enumerate}
		\item The \gls{user} starts a new \gls{session} recording.
		\item The \gls{user} scrolls the mouse wheel in "Windows Explorer".
		\item The \gls{user} finalizes the recording.
	\end{enumerate}}
	{\begin{itemize}
		\item A recording has been created and stored at the path specified by the configuration.
		\item The recording contains a mouse scroll \gls{event},  that contains the scroll amount, the HWND associated with the "Windows Explorer" window that was scrolled, along with the timestamp at which the \gls{user} used the scroll wheel, the identifier of the mouse-interaction module and the "mouse:scroll" type of the event.
	\end{itemize}}
	
	\test{\fmt{FS20}{FS30}{FS54}{FS70}{FS250}{FS413}}{Mouse move event}{The application is running. The configuration file is valid.}
	{\begin{enumerate}
		\item The \gls{user} starts a new \gls{session} recording.
		\item The \gls{user} moves the mouse.
		\item The \gls{user} finalizes the recording.
	\end{enumerate}}
	{\begin{itemize}
		\item A recording has been created and stored at the path specified by the configuration.
		\item The recording contains a mouse move \gls{event},  that contains the movement vector along with the timestamp at which the \gls{user} moved the mouse, the identifier of the mouse-interaction module and the "mouse:move" type of the event.
	\end{itemize}}
\end{tests}

\subsubsection{Keyboard module test cases}

\begin{tests}
	\test{\fmt{FS20}{FS30}{FS54}{FS70}{FS250}{FS421}}{Keyboard event}{The application is running. The configuration file is valid.}
	{\begin{enumerate}
		\item The \gls{user} starts a new \gls{session} recording.
		\item The \gls{user} presses a key while a "Windows Explorer" window is focused.
		\item The \gls{user} releases a key while a "Windows Explorer" window is focused.
		\item The \gls{user} finalizes the recording.
	\end{enumerate}}
	{\begin{itemize}
		\item A recording has been created and stored at the path specified by the configuration.
		\item The recording contains the key-press \gls{event} and release-key event. Each \gls{event} contains the name of the pressed/released key, the names of all modifier keys that were pressed along with the timestamp at which the \gls{user} pressed/released a key, the identifier of the keyboard-interaction module and the "keyboard:press-key"/"keyboard:release-key" type of the event.
	\end{itemize}}
\end{tests}

\subsubsection{Clipboard module test cases}
	
\begin{tests}
	\test{\fmt{FS20}{FS30}{FS54}{FS70}{FS250}{FS431}}{Clipboard copy event}{The application is running. The configuration file is valid.}
	{\begin{enumerate}
		\item The \gls{user} starts a new \gls{session} recording.
		\item The \gls{user} copies text from a "Microsoft Excel" window.
		\item The \gls{user} finalizes the recording.
	\end{enumerate}}
	{\begin{itemize}
		\item A recording has been created and stored at the path specified by the configuration.
		\item The recording contains the clipboard copy \gls{event},  that contains the clipboard text associated with "Microsoft Excel", the type of the clipboard interaction - copy, along with the timestamp at which the \gls{user} copied the text to the clipboard, the identifier of the clipboard-interaction module and the "clipboard:copy" type of the event.
	\end{itemize}}
	
	\test{\fmt{FS20}{FS30}{FS54}{FS70}{FS250}{FS432}}{Clipboard paste event}{The application is running. The configuration file is valid.}
	{\begin{enumerate}
		\item The \gls{user} starts a new \gls{session} recording.
		\item The \gls{user} pastes text in the "Mozilla Firefox/Google Chrome" address bar.
		\item The \gls{user} finalizes the recording.
	\end{enumerate}}
	{\begin{itemize}
		\item A recording has been created and stored at the path specified by the configuration.
		\item The recording contains the clipboard paste \gls{event},  that contains the clipboard text associated with "Mozilla Firefox/Google Chrome", the type of the clipboard interaction - paste, along with the timestamp at which the \gls{user} pasted the text from the clipboard, the identifier of the clipboard-interaction module and the "clipboard:paste" type of the event.
	\end{itemize}}
\end{tests}

\subsection{On-demand module test cases}

\subsubsection{Browser module test cases}
\begin{tests}
	\test{\fmt{FS20}{FS30}{FS54}{FS70}{FS250}{FS441}}{Browser open tab event}{The application is running. The configuration file is valid.}
	{\begin{enumerate}
		\item The \gls{user} starts a new \gls{session} recording.
		\item The \gls{user} opens a new tab in "Mozilla Firefox/Google Chrome".
		\item The \gls{user} finalizes the recording.
	\end{enumerate}}
	{\begin{itemize}
		\item A recording has been created and stored at the path specified by the configuration.
		\item The recording contains the open-tab-browser \gls{event},  that contains the URL of the interacted website and the unique identifier of the new tab along with the timestamp at which the \gls{user} opened a new tab in "Mozilla Firefox/Google Chrome", the identifier of the \gls{browser} module and the "browser:open-tab" type of the event.
	\end{itemize}}
	
	\test{\fmt{FS20}{FS30}{FS54}{FS70}{FS250}{FS442}}{Browser switch to tab event}{The application is running. The configuration file is valid.}
	{\begin{enumerate}
		\item The \gls{user} starts a new \gls{session} recording.
		\item The \gls{user} switches to a tab in "Mozilla Firefox/Google Chrome".
		\item The \gls{user} finalizes the recording.
	\end{enumerate}}
	{\begin{itemize}
		\item A recording has been created and stored at the path specified by the configuration.
		\item The recording contains the switch-tab-browser \gls{event},  that contains the URL of the interacted website, the unique identifier of the old tab and the unique identifier of the tab that the \gls{user} switched to along with the timestamp at which the \gls{user} switched to a tab "Mozilla Firefox/Google Chrome", the identifier of the \gls{browser} module and the "browser:switch-tab" type of the event.
	\end{itemize}}
	
	\test{\fmt{FS20}{FS30}{FS54}{FS70}{FS250}{FS443}}{Browser close tab event}{The application is running. The configuration file is valid.}
	{\begin{enumerate}
		\item The \gls{user} starts a new \gls{session} recording.
		\item The \gls{user} closes a tab "Mozilla Firefox/Google Chrome".
		\item The \gls{user} finalizes the recording.
	\end{enumerate}}
	{\begin{itemize}
		\item A recording has been created and stored at the path specified by the configuration.
		\item The recording contains the close-tab-browser \gls{event},  that contains the URL of the interacted website and the unique identifier of the tab along with the timestamp at which the \gls{user} closed a tab in "Mozilla Firefox/Google Chrome", the identifier of the \gls{browser} module and the "browser:close-tab" type of the event.
	\end{itemize}}
	
	\test{\fmt{FS20}{FS30}{FS54}{FS70}{FS250}{FS444}}{Browser navigation event}{The application is running. The configuration file is valid.}
	{\begin{enumerate}
		\item The \gls{user} starts a new \gls{session} recording.
		\item The \gls{user} navigates to a URL on the google.com search results page.
		\item The \gls{user} finalizes the recording.
	\end{enumerate}}
	{\begin{itemize}
		\item A recording has been created and stored at the path specified by the configuration.
		\item The recording contains the navigation-browser \gls{event},  that contains the URL of the page that the \gls{user} navigated to and the unique identifier of the tab, which the navigation \gls{event} occured in, along with the timestamp at which the \gls{user} navigated to a URL, the identifier of the \gls{browser} module and the "browser:navigation" type of the event.
	\end{itemize}}
	
	\test{\fmt{FS20}{FS30}{FS54}{FS70}{FS250}{FS445}}{Browser text input event}{The application is running. The configuration file is valid.}
	{\begin{enumerate}
		\item The \gls{user} starts a new \gls{session} recording.
		\item The \gls{user} inputs text into a search box on the google.com page.
		\item The \gls{user} finalizes the recording.
	\end{enumerate}}
	{\begin{itemize}
		\item A recording has been created and stored at the path specified by the configuration.
		\item The recording contains the text-input-browser \gls{event},  that contains the URL of the google.com page, the unique identifier of the tab, which the text-input \gls{event} occured in, the text which has been input by the \gls{user}, information about search box, where text has been input, along with the timestamp at which the \gls{user} input the text, the identifier of the \gls{browser} module and the "browser:text-input" type of the event.
	\end{itemize}}
	
	
	\test{\fmt{FS20}{FS30}{FS54}{FS70}{FS250}{FS446}}{Browser button click event}{The application is running. The configuration file is valid.}
	{\begin{enumerate}
		\item The \gls{user} starts a new \gls{session} recording.
		\item The \gls{user} clicks a button on the google.com page.
		\item The \gls{user} finalizes the recording.
	\end{enumerate}}
	{\begin{itemize}
		\item A recording has been created and stored at the path specified by the configuration.
		\item The recording contains the button-click-browser \gls{event},  that contains the URL of the google.com page, the unique identifier of the tab, which the button-click \gls{event} occured in, the title of the button item, the URL of the website that the button is linked to (if applicable) along with the timestamp at which the \gls{user} clicked a button on the page, the identifier of the \gls{browser} module and the "browser:button-click" type of the event.
	\end{itemize}}
	
	
	\test{\fmt{FS20}{FS30}{FS54}{FS70}{FS250}{FS447}}{Browser hovering event}{The application is running. The configuration file is valid.}
	{\begin{enumerate}
		\item The \gls{user} starts a new \gls{session} recording.
		\item The \gls{user} hovers the mouse-pointer over a web-element on the google.com page.
		\item The \gls{user} finalizes the recording.
	\end{enumerate}}
	{\begin{itemize}
		\item A recording has been created and stored at the path specified by the configuration.
		\item The recording contains the hover-browser \gls{event},  that contains the URL of the google.com page, the unique identifier of the tab, which the hover \gls{event} occured in, the information about the element which has been hovered along with the timestamp at which the \gls{user} hovered the web-element, the identifier of the \gls{browser} module and the "browser:hover" type of the event.
	\end{itemize}}
	
	
	\test{\fmt{FS20}{FS30}{FS54}{FS70}{FS250}{FS448}}{Browser text selection event}{The application is running. The configuration file is valid.}
	{\begin{enumerate}
		\item The \gls{user} starts a new \gls{session} recording.
		\item The \gls{user} selects text on the google.com page.
		\item The \gls{user} finalizes the recording.
	\end{enumerate}}
	{\begin{itemize}
		\item A recording has been created and stored at the path specified by the configuration.
		\item The recording contains the text-selection-browser \gls{event},  that contains the URL of the google.com page, the unique identifier of the tab, which the text-selection \gls{event} occured in, selected text along with the timestamp at which the \gls{user} selected the text, the identifier of the \gls{browser} module and the "browser:text-selection" type of the event.
	\end{itemize}}
	
	
	\test{\fmt{FS20}{FS30}{FS54}{FS70}{FS250}{FS449}}{Browser file download event}{The application is running. The configuration file is valid.}
	{\begin{enumerate}
		\item The \gls{user} starts a new \gls{session} recording.
		\item The \gls{user} downloads a file from the google.com page.
		\item The \gls{user} finalizes the recording.
	\end{enumerate}}
	{\begin{itemize}
		\item A recording has been created and stored at the path specified by the configuration.
		\item The recording contains the download-browser \gls{event},  that contains the URL of the google.com page, the unique identifier of the tab, which the download \gls{event} occured in, the URL of the file that was downloaded along with the timestamp at which the \gls{user} downloaded the file, the identifier of the \gls{browser} module and the "browser:download" type of the event.
	\end{itemize}}
	
\end{tests}
