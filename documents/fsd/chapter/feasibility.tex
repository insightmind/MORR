\chapter{Feasibility Study}
\label{ch:feasibility}

\section{Technical Feasibility}
\label{Technical Feasibility}
The functional specification of this application can be broadly implemented by the Microsoft .NET Core infrastructure. It provides a varied and extensive collection of libraries, which provide us with tools for \gls{event} logging, video-metadata embedding and inter-module communication.\\
The user interface framework for this project is WPF. It is feasible enough because our application does not require extensive user interfaces.\\
Lastly, Google Chrome and Mozilla Firefox provide us with an application interface which can be used to detect \glspl{event} and interactions with the \gls{browser} as described in chapter \ref{ch:data}.

\section{Personnel Feasibility}
\label{Personnel Feasibility}
The personnel for this project matches the requirements. The team consists of five computer science students currently pursuing a bachelor of science degree and three supervisors. A requirement for the "Praxis der Softwareentwicklung" module is the successful completion of the "Softwaretechnik 1" module and all orientation examinations. These modules provide fundamental to intermediate knowledge about a software project process and execution.\\
The time investment for each student is approximately 270 hours. Each team member is equally involved in the implementation of this project.\\
The project will be executed using the waterfall model.

\section{Economical Feasibility}
\label{Economical Feasibility}
As this project is part of a mandatory module for the bachelor's degree in computer science, the students are not receiving compensation.\\
Therefore there are no concerns about the economic feasibility of this project.

\section{Legal Feasibility}
\label{Legal Feasibility}
This project relies heavily on the collection of data from \glspl{user}. With the introduction of the DSGVO by the European Union in 2018 an increased attention must be taken to address these legal regulations.\\
The application itself always makes sure the \gls{user} can see that a \gls{session} is currently running, as described in the functional and non-functional specifications in chapter \ref{ch:func}. The \gls{user} is also directly asked whether he wants to start or stop a \gls{session}.
It must be made sure that the \gls{user} knows how and why their data is collected on their system and that they are able to delete their data on request.

\section{Alternatives}
\label{Alternatives}
An alternative approach to this project is a manual collection of the data. This, however, results in significantly higher costs and time investment. This alternative would also not be able to record data in a matching level of detail, e.g. tracking mouse movement would be hard to realize.

