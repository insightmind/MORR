\section{CommandLineInterface}

\issue{
Different Options for different Commands
}{
Add class strategy for commands and make \class{Command} generic. 
}{
A new abstract class \class{CommandOptions} has been added which contains only those options which are a necessity for all commands. This in our case is for example the \keyword{IsVerbose} property.
All specific commands need then to generically define their specific option they need.
}

\issue{
Bad performance of \package{CORE} when using \package{CLI}.
}{
Start own \keyword{MessagePump} on main thread in \package{CLI}.
}{
To accomplish this solution we implemented a new class \class{MessageLoop} which allows to start a message loop from the main thread. This achieves better performance in the \package{CORE} as
a \keyword{MessagePump} is not naturally present in command line applications in difference to \keyword{WPF} applications.
}

\issue{
\class{MessageLoop} blocks main thread and user interaction.
}{
Introduce \class{InteractiveCommandLine} and possiblity to stop \keyword{MessagePump}
}{
A new class \class{InteractiveCommandLine} was added which is launched on a seperate thread, to allow user interaction while recording. You are now able to use the 'x' command to cancel a currently running session.
The \class{MessageLoop} has been extended to support cancellation of a currently running \keyword{MessagePump} from a different thread using the method \member{MessageLoop.Cancel}.
}

\issue{
Missing command for starting a recording session.
}{
Implement a new command to start a recording session.
}{
Add classes \class{RecordCommand} and \class{RecordOptions}.
}

\issue{
Missing command for validating a given configuration file.
}{
Implement a new command validate a given configuration file. 
}{
Add classes \class{ValidateCommand} and \class{ValidateOptions}.
}