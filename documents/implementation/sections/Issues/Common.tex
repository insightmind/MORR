\section{Common}
\subsection{Shared}
\issue{
Distinction between \class{ICollectingModule}, \class{ITransformingModule} and \class{IReceivingModule} has proven unnecessary.
}{
Do not differentiate between different modules and suggest use of general \class{IModule}.
}{
Remove interfaces: \class{ICollectingModule}, \class{ITransformingModule} and \class{IReceivingModule}.
}

\issue{
Design of different event queue strategies: \class{RingBufferStorageStrategy}, \class{RefCountedListStorageStrategy}, \class{KeepAllStorageStrategy} were not suitable for our implementation approach and use of the C\# .NET Core language library.
}{
Design new event queue storage strategies for different types of usage: \keyword{Bounded} or \keyword{Unbounded} and \keyword{SingleConsumer} or \keyword{MultiConsumer}.
}{
Add abstract storage strategies: \class{SingleConsumerChannelStrategy} and \class{MutliConsumerChannelStrategy}.\\
A \keyword{SingleConsumer} strategy allows only one consumer at a time to read events out of the queue which provides a performance boost in comparison to \keyword{MultiConsumer}.
A \keyword{MultiConsumer} strategy allows a finite and infinite amount of consumers to read simultanously. This provides multiplication of references to a specific \class{Event} and supports reference counting for each consumer.

Both strategy types have been additionally divided into \keyword{Bounded}, which acts as a RingBuffer and removes the oldest events if the queue is full, and \keyword{Unbounded}, which does not restrict the event count of the queue.
}

\issue{
Unable to use designed event queue structure in \keyword{MEF} correctly.
}{
Add new abstraction of event queues to support distinct imports in MEF and simplify usage for Modules.
}{
Add interfaces concerning distinct sections of the event pipeline and read-write access:\\
\class{IReadOnlyEventQueue}: To restrict usage to read only methods.\\
\class{ISupportDeserializationEventQueue}: Allows a strongly typed usage of events in the deserialization process that stores events serialized, which is necessary for any recording or processing session.\\
\class{IDecodableEventQueue}: Defines an strongly typed read only event queue which is used in the decoding process of the pipeline.\\
\class{IEncodableEventQueue}: Defines an strongly typed read only event queue which is used in the encoding process of the pipeline.\\

Additionally multiple default implementation using their correct storage strategies were introduced to simplify usage of queues in a Module.
}

\issue{
Unable to gather events from unowned processes and windows.
}{
Support our additional \package{HookLibrary} for use by our modules.
}{
To accomplish this and reduce the amount of multiple hooks per windows process, we had to add \class{GlobalHook} and \class{HookNativeMethods} to our shared project.
}

\issue{
Handling of directory- or filepaths using \class{string} has proven error-prone and not typesafe.
}{
Add shared support for directory- and filepaths using new classes.
}{
Add \class{DirectoryPath} and \class{FilePath} classes.
}

\issue{
Handling of to be serialized configurations using \class{string} has proven error-prone and not typesafe.
}{
Add shared support for strongly typed raw configurations.
}{
Add \class{RawConfiguration} as a wrapper to the unserialized configuration as a string at \member{RawConfiguration.RawValue}.
}
