\subsection{CLI}
\begin{class}{Program}
	\clsdiagram[width=0.6\textwidth]{resources/Classes/Core/CLI/Program.png}
	
	\clsdcl{public class Program}
	
	\clsdsp{Main entrypoint class for the CLI tool. Resolves all commands and validates their arguments using the CommandLineParser library. It uses the OutputFormatter for any outputs resulting of the commands. Lastly it keeps a reference to the RecordingManager within the main function which may be used by a command executing. }
	
	\begin{methods}
		\begin{method}{public static void Main(string[] args)}{\\ Gets called on application start and parses the arguments to resolve the given command using CommandLine.Parser. }
			\begin{parameters}
				\para{string[] args}{\\ Array of strings representing the arguments with which the application has been started using the command line. }
			\end{parameters}
		\end{method}
	\end{methods}
\end{class}

\begin{class}{OutputFormatter}
	\clsdiagram[width=0.7\textwidth]{resources/Classes/Core/CLI/OutputFormatter.png}
	
	\clsdcl{public class OutputFormatter}
	
	\clsdsp{The OutputFormatter's main functionality is about printing and formatting any output made by the program. It also filters them according of the users preferences, e.g. whether the user wants to see verbose ouput or not. }
	
	\begin{attributes}
		\attribute{public readonly bool Verbose = false}{Defines whether verbose output should be printed or not.}
	\end{attributes}
	
	\begin{methods}
		\begin{method}{public void PrintMessage(string message, bool verbose)}{\\ Prints the message to the command line. It also filters out based on the verbosity of the message}
			\begin{parameters}
				\para{string message}{\\ The message which should be printed to the command line.}
				\para{bool verbose}{\\ The verbosity of the message. If it is false it will always be printed to the command line. Otherwise it will only be printed if the formatter is set to verbose itself.}
			\end{parameters}
		
		\end{method}
		
		\begin{method}{public void PrintException(Exception exception)}{\\ Prints out an exception to the user indicating that something went wrong. This is always non-verbose as it may interrupt the process of our program. }
			\begin{parameters}
				\para{Exception exception}{\\ The exception which should be printed to the command line.}
			\end{parameters}
		
		\end{method}
	\end{methods}
	
\end{class}
\subsection*{MORR.Core.CLI.Command}
\begin{interface}{ICLICommand}
	\clsdiagram[width=0.65\textwidth]{resources/Classes/Core/CLI/ICLICommand.png}
	
	\clsdsp{The ICLICommand interface allows us to create a variety of commands which can be run by the program. It is based on the command design pattern.}
	
	\clsdcl{public interface ICLICommand}
	
	\begin{methods}
		\begin{method}{void ExecuteCommand(Options options)}{\\ Executes the implementing command using the options given to it.}
		
			\begin{parameters}
				\para{Options options}{\\ The options entered by the user, which may be used to configure the command. A command may define its own options. }
			\end{parameters}
			
			\begin{exceptions}
				\excp{Exception}{ A command may throw any exception if execution fails. This however can be defined more specifically by each implementing command itself. }
			\end{exceptions}
		\end{method}
	\end{methods}
\end{interface}

\begin{class}{ProcessCommand}
	\clsdiagram[width=0.65\textwidth]{resources/Classes/Core/CLI/ProcessCommand.png}
	
	\clsdsp{The ProcessCommand is called by the user using the keyword process. It will process the collected events from a previous recording session by extracting it out of the recording container, e.g. MPEG and saving it in another recording container, e.g. CSV. It can probably be seen as a converter, however the command allows to process the events using the event pipeline if specified by the user. }
	
	\clsdcl{public class ProcessCommand : ICLICommand }
	
	\begin{constructors}
		\begin{constructor}{public ProcessCommand(IRecordingManager recordingManager)}{Creates a new instance of the ProcessCommand which will use the provided IRecordingManager for decoding.}
			\begin{parameters}
				\para{IRecordingManager recordingManager}{The IRecordingManager to use.}
			\end{parameters}
		\end{constructor}
	\end{constructors}

	\begin{methods}
		\begin{method}{void ExecuteCommand(ProcessOptions options)}{\\ Executes the implementing command using the options given to it.}
		
			\begin{parameters}
				\para{ProcessOptions options}{\\ ProcessOptions which contain the configuration for this command. }
			\end{parameters}
			
			\begin{exceptions}
				\excp{IDecodingException}{ Is thrown if the decoding of the recording container fails. }
				\excp{IEncodingException}{ Is thrown if the encoding of the processed events fails. }
			\end{exceptions}
		\end{method}
	\end{methods}
\end{class}

\begin{class}{ProcessOptions}
	\clsdiagram[width=0.4\textwidth]{resources/Classes/Core/CLI/ProcessOptions.png}
	
	\clsdsp{The ProcessOptions contain all options which can be entered by the user when calling the command. In this case the command is the ProcessCommand. It inherits from Options which is provided by the CommandLineParser library. }
	
	\clsdcl{public class ProcessOptions : Options}
	
	\begin{attributes}
		\attribute{public string Input}{\\ The fileURL to the input file. This property is required to be entered by the user.}
		\attribute{public string Output}{\\  The fileURL for the output file. This property is required to be entered by the user.}
		\attribute{public bool ProcessingEnabled = true}{\\  This attribute defines whether processing is enabled while converting the data. This is an optional property and defaults to true.}
		\attribute{public bool Verbose = false}{\\ This attribute defines whether verbose outputs should be printed to the users command line. This is an optional property and defaults to false.}
	\end{attributes}
\end{class}