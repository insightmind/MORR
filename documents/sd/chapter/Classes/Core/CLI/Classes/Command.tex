\subsection*{MORR.Core.CLI.Command}
\begin{interface}{ICLICommand}
	\clsdiagram[width=0.65\textwidth]{resources/Classes/Core/CLI/ICLICommand.png}
	
	\clsdsp{The ICLICommand interface allows us to create a variety of commands which can be run by the program. It is based on the command design pattern.}
	
	\clsdcl{public interface ICLICommand}
	
	\begin{methods}
		\begin{method}{void ExecuteCommand(Options options)}{\\ Executes the implementing command using the options given to it.}
		
			\begin{parameters}
				\para{Options options}{\\ The options entered by the user, which may be used to configure the command. A command may define its own options. }
			\end{parameters}
			
			\begin{exceptions}
				\excp{Exception}{ A command may throw any exception if execution fails. This however can be defined more specifically by each implementing command itself. }
			\end{exceptions}
		\end{method}
	\end{methods}
\end{interface}

\begin{class}{ProcessCommand}
	\clsdiagram[width=0.65\textwidth]{resources/Classes/Core/CLI/ProcessCommand.png}
	
	\clsdsp{The ProcessCommand is called by the user using the keyword process. It will process the collected events from a previous recording session by extracting it out of the recording container, e.g. MPEG and saving it in another recording container, e.g. CSV. It can probably be seen as a converter, however the command allows to process the events using the event pipeline if specified by the user. }
	
	\clsdcl{public class ProcessCommand : ICLICommand }
	
	\begin{attributes}
		\attribute{public RecordingManager recordingManager}{The recording manager is the connection to the main application which is used to load all modules and process the decoded events through the pipeline and finally decode it. }
		
		\attribute{public IDecoder decoder}{The decoder is used to decode the given recording container and receive the captured events from it. }
	\end{attributes}
	
	\begin{methods}
		\begin{method}{void ExecuteCommand(ProcessOptions options)}{\\ Executes the implementing command using the options given to it.}
		
			\begin{parameters}
				\para{ProcessOptions options}{\\ ProcessOptions which contain the configuration for this command. }
			\end{parameters}
			
			\begin{exceptions}
				\excp{IDecodingException}{ Is thrown if the decoding of the recording container fails. }
				\excp{IEncodingException}{ Is thrown if the encoding of the processed events fails. }
			\end{exceptions}
		\end{method}
	\end{methods}
\end{class}

\begin{class}{ProcessOptions}
	\clsdiagram[width=0.4\textwidth]{resources/Classes/Core/CLI/ProcessOptions.png}
	
	\clsdsp{The ProcessOptions contain all options which can be entered by the user when calling the command. In this case the command is the ProcessCommand. It inherits from Options which is provided by the CommandLineParser library. }
	
	\clsdcl{public class ProcessOptions : Options}
	
	\begin{attributes}
		\attribute{public string Input}{\\ The fileURL to the input file. This property is required to be entered by the user.}
		\attribute{public string Output}{\\  The fileURL for the output file. This property is required to be entered by the user.}
		\attribute{public bool ProcessingEnabled = true}{\\  This attribute defines whether processing is enabled while converting the data. This is an optional property and defaults to true.}
		\attribute{public bool Verbose = false}{\\ This attribute defines whether verbose outputs should be printed to the users command line. This is an optional property and defaults to false.}
	\end{attributes}
\end{class}