\subsection{MORR}

\subsection*{MORR.Core}

\begin{interface}{IBootstrapper}
    \clsdiagram{resources/Classes/Core/MORR/IBootstrapper.png}

    \clsdcl{public interface IBootstrapper}

    \clsdsp{Responsible for bootstrapping the application and providing compositional facilities.}

    \begin{methods}
        \begin{method}{void ComposeImports(object @object)}{Composes the provided object.}
            \begin{parameters}
                \para{object @object}{The object to satisfy the imports on.}
            \end{parameters}
        \end{method}
    \end{methods}
\end{interface}

\begin{class}{Bootstrapper}
    \clsdiagram{resources/Classes/Core/MORR/Bootstrapper.png}
    
    \clsdcl{public class Bootstrapper : IBootstrapper}

    \clsdsp{Bootstraps the application using MEF.}

    \begin{methods}
        \begin{method}{public Bootstrapper()}{Creates a new instance of the Bootstrapper and loads all .MORR-Module.dll assemblies from the Modules subdirectory relative to the directory of the current executing assembly.}
        \end{method}
        \begin{method}{public void ComposeImports(object @object)}{Satisfies the imports on the provided object.}
            \begin{parameters}
                \para{object @object}{The object to satisfy the imports on.}
            \end{parameters}
        \end{method}
    \end{methods}
\end{class}

\begin{class}{BootstrapperConventions}
    \clsdiagram{resources/Classes/Core/MORR/BootstrapperConventions.png}

    \clsdcl{public static class BootstrapperConventions}

    \clsdsp{Provides conventions for composing objects with MEF.}

    \begin{methods}
        \begin{method}{public static RegistrationBuilder GetRegistrationBuilder()}{Gets a registration builder that contains all composition conventions.}
            \return{RegistrationBuilder}{A RegistrationBuilder containing the composition conventions.}
        \end{method}
    \end{methods}
\end{class}

\subsection*{MORR.Core.Configuration}

\begin{interface}{IConfigurationManager}
    \clsdiagram{resources/Classes/Core/MORR/Configuration/IConfigurationManager.png}

    \clsdcl{public interface IConfigurationManager}

    \clsdsp{Loads and manages the application's configuration.}

    \begin{methods}
        \begin{method}{void LoadConfiguration(FilePath path)}{Loads the configuration from the specified path.}
            \begin{parameters}
                \para{FilePath path}{The path to load the configuration from.}
            \end{parameters}
            \begin{exceptions}
                \excp{InvalidConfigurationException}{Exception thrown if the specified configuration is invalid.}
            \end{exceptions}
        \end{method}
    \end{methods}
\end{interface}

\begin{class}{ConfigurationManager}
    \clsdiagram{resources/Classes/Core/MORR/Configuration/ConfigurationManager.png}

    \clsdcl{public class ConfigurationManager : IConfigurationManager}

    \clsdsp{Manages the application's configuration.}
    
    \begin{constructors}
		\begin{constructor}{public ConfigurationManager())}{Creates a new instance of the ConfigurationManager}
		\end{constructor}
		\begin{constructor}{public ConfigurationManager(IFileSystem fileSystem))}{Creates a new instance of the ConfigurationManager with specified FileSystem}
			\begin{parameters}
				\para{IFileSystem fileSystem}{The file system to construct the instance with.}
			\end{parameters}
		\end{constructor}
	\end{constructors}

    \begin{methods}
        \begin{method}{public void LoadConfiguration(FilePath path)}{Loads the configuration from the specified path.}
            \begin{parameters}
                \para{FilePath path}{The path to load the configuration from.}
            \end{parameters}
            \begin{exceptions}
                \excp{InvalidConfigurationException}{Exception thrown if the specified configuration is invalid.}
            \end{exceptions}
        \end{method}
    \end{methods}
\end{class}

\begin{class}{InvalidConfigurationException}
    \clsdiagram{resources/Classes/Core/MORR/Configuration/InvalidConfigurationException.png}

    \clsdcl{public class InvalidConfigurationException : Exception}

    \clsdsp{An exception thrown when the configuration is invalid.}

    \begin{constructors}
        \begin{constructor}{public InvalidConfigurationException()}{Creates a new instance of an InvalidConfigurationException without any specific error message.}
        \end{constructor}
        \begin{constructor}{public InvalidConfigurationException(string message)}{Creates a new instance of an InvalidConfigurationException with the specified error message.}
            \begin{parameters}
                \para{string message}{The error message to construct the instance with.}
            \end{parameters}
        \end{constructor}
        \begin{constructor}{public InvalidConfigurationException(string message, Exception innerException)}{Creates a new instance of an InvalidConfigurationException with the specified error message and inner exception.}
            \begin{parameters}
                \para{string message}{The error message to construct the instance with.}
                \para{Exception innerException}{The inner exception to construct the instance with.}
            \end{parameters}
        \end{constructor}
    \end{constructors}
\end{class}

\subsection*{MORR.Core.Session}

\begin{interface}{ISessionManager}
    \clsdiagram{resources/Classes/Core/MORR/Session/ISessionManager.png}

    \clsdcl{public interface ISessionManager}

    \clsdsp{A manager responsible for all aspects of recording and processing.}

    \begin{attributes}
        \attribute{DirectoryPath? CurrentRecordingDirectory \{ get; \}}{The path to the directory containing the most recent recording or null if no recording has been created yet.}
        \attribute{DirectoryPath? RecordingsFolder \{ get; \}}{The path to the top-level folder containing the recording subdirectories.}
    \end{attributes}

    \begin{methods}
        \begin{method}{public void StartRecording()}{Starts a recording if no session is currently being recorded.}
            \begin{exceptions}
                \excp{AlreadyRecordingException}{Thrown if a recording is started while another recording is already running.}
            \end{exceptions}
        \end{method}
        \begin{method}{public void StopRecording()}{Stops a recording if a session is currently being recorded.}
            \begin{exceptions}
                \excp{NotRecordingException}{Thrown if a recording is stopped while no recording is currently running.}
            \end{exceptions}
        \end{method}
        \begin{method}{void Process(IEnumerable<DirectoryPath> recordings)}{Processes the specified recordings.}
            \begin{parameters}
                \para{IEnumerable<DirectoryPath> recordings}{The recordings to process.}
            \end{parameters}
        \end{method}
    \end{methods}
\end{interface}

\begin{class}{SessionManager}
    \clsdiagram[width=\textwidth]{resources/Classes/Core/MORR/Session/SessionManager.png}

    \clsdcl{public class SessionManager : ISessionManager}

    \clsdsp{A manager responsible for all aspects of recording and processing.}

    \begin{attributes}
        \attribute{DirectoryPath? CurrentRecordingDirectory \{ get; \}}{The path to the directory containt the most recent recording or null if no recording has been created yet.}
        \attribute{DirectoryPath? RecordingsFolder \{ get; \}}{The path to the top-level folder containing the recording subdirectories.}
    \end{attributes}

    \begin{constructors}
        \begin{constructor}{public SessionManager(FilePath configurationPath)}{Creates a new instance of a SessionManager with the specified configuration path and a default Bootstrapper, ConfigurationManager and ModuleManager}
            \begin{parameters}
                \para{FilePath configurationPath}{The configuration path to construct the instance with.}
            \end{parameters}
        \end{constructor}
        \begin{constructor}{public SessionManager(FilePath configurationPath, IBootstrapper bootstrapper, IConfigurationManager configurationManager, IModuleManager moduleManager, IFileSystem fileSystem)}{Creates a new instance of a SessionManager with the specified configuration path, bootstrapper, configuration manager, module manager and file system}
            \begin{parameters}
                \para{FilePath configurationPath}{The configuration path to construct the instance with.}
                \para{IBootstrapper bootstrapper}{The bootstrapper to construct the instance with.}
                \para{IConfigurationManager configurationManager}{The configuration manager to construct the instance with.}
                \para{IModuleManager moduleManager}{The module manager to construct the instance with.}
                \para{IFileSystem fileSystem}{The file system to construct the instance with.}
            \end{parameters}
        \end{constructor}
    \end{constructors}

    \begin{methods}
        \begin{method}{public void StartRecording()}{Starts a recording if no session is currently being recorded.}
            \begin{exceptions}
                \excp{AlreadyRecordingException}{Thrown if a recording is started while another recording is already running.}
            \end{exceptions}
        \end{method}
        \begin{method}{public void StopRecording()}{Stops a recording if a session is currently being recorded.}
            \begin{exceptions}
                \excp{NotRecordingException}{Thrown if a recording is stopped while no recording is currently running.}
            \end{exceptions}
        \end{method}
        \begin{method}{void Process(IEnumerable<DirectoryPath> recordings)}{Processes the specified recordings.}
            \begin{parameters}
                \para{IEnumerable<DirectoryPath> recordings}{The recordings to process.}
            \end{parameters}
        \end{method}
    \end{methods}
\end{class}

\begin{class}{SessionConfiguration}
    \clsdiagram{resources/Classes/Core/MORR/Session/SessionConfiguration.png}

    \clsdcl{public class Session : IConfiguration}

    \clsdsp{Encapsulates all session configuration options.}

    \begin{attributes}
        \attribute{public IEnumerable<Type> Encoders \{ get; set; \}}{The types of the encoders to use.}
        \attribute{public IEnumerable<Type>? Decoders \{ get; set; \}}{The types of the decoders to use.}
        \attribute{public DirectoryPath RecordingDirectory \{ get; set; \}}{The directory in which to store new recordings.}
    \end{attributes}
    
    \begin{methods}
        \begin{method}{public void Parse(RawConfiguration configuration)}{Parses the configuration from the provided value.}
            \begin{parameters}
                \para{RawConfiguration configuration}{The configuration value to parse from.}
            \end{parameters}
        \end{method}
    \end{methods}
\end{class}

\subsection*{MORR.Core.Session.Crypto}

\begin{class}{CryptoHelper}
	\clsdiagram{resources/Classes/Core/MORR/Session/Crypto/CryptoHelper.png}
	
	\clsdcl{public static class CryptoHelper}
	
	\clsdsp{Helper class for encrypting the name of the user.}
	
	\begin{methods}
		\begin{method}{public static string GenerateHash(string rawData)}{Generates a hash of the provided string using the SHA256 algorithm.}
			\begin{parameters}
                \para{string rawData}{The string to be hashed.}
            \end{parameters}
            \return{string}{The hashed version of the rawData string.}
        \end{method}
     \end{methods}
\end{class}

\subsection*{MORR.Core.Session.Exceptions}

\begin{class}{RecordingException}
    \clsdiagram{resources/Classes/Core/MORR/Session/Exceptions/RecordingException.png}

    \clsdcl{public class RecordingException : Exception}

    \clsdsp{A generic recording exception that all specialized recording exceptions derive from.}
\end{class}

\begin{class}{AlreadyRecordingException}
    \clsdiagram{resources/Classes/Core/MORR/Session/Exceptions/AlreadyRecordingException.png}

    \clsdcl{public class AlreadyRecordingException : RecordingException}

    \clsdsp{An exception thrown if a recording is started while another recording is already running.}
\end{class}

\begin{class}{NotRecordingException}
    \clsdiagram{resources/Classes/Core/MORR/Session/Exceptions/NotRecordingException.png}

    \clsdcl{public class NotRecordingException : RecordingException}

    \clsdsp{An exception thrown if a recording is stopped while no recording is currently running.}
\end{class}

\subsection*{MORR.Core.Modules}

\begin{interface}{IModuleManager}
    \clsdiagram{resources/Classes/Core/MORR/Modules/IModuleManager.png}

    \clsdcl{public interface IModuleManager}

    \clsdsp{Initializes and manages all modules.}

    \begin{methods}
        \begin{method}{void InitializeModules()}{Initializes all modules.}      
        \end{method}
        \begin{method}{void NotifyModulesOnSessionStart()}{Notifies all modules when a session starts.}
        \end{method}
        \begin{method}{void NotifyModulesOnSessionStop()}{Notifies all modules when a session stops.}
        \end{method}
    \end{methods}
\end{interface}

\begin{class}{ModuleManager}
    \clsdiagram{resources/Classes/Core/MORR/Modules/ModuleManager.png}

    \clsdcl{public class ModuleManager : IModuleManager}

    \clsdsp{Initializes and manages all modules..}

    \begin{methods}
        \begin{method}{void InitializeModules()}{Initializes all modules.}      
        \end{method}
        \begin{method}{void NotifyModulesOnSessionStart()}{Notifies all modules when a session starts.}
        \end{method}
        \begin{method}{void NotifyModulesOnSessionStop()}{Notifies all modules when a session stops.}
        \end{method}
    \end{methods}
\end{class}

\begin{class}{GlobalModuleConfiguration}
    \clsdiagram{resources/Classes/Core/MORR/Modules/GlobalModuleConfiguration.png}

    \clsdcl{public class GlobalModuleConfiguration : IConfiguration}

    \clsdsp{Encapsulates all global module configuration options.}

    \begin{attributes}
        \attribute{public IEnumerable<Type> EnabledModules \{ get; set; \}}{The types of all IModule instances that should be enabled.}
    \end{attributes}
    
    \begin{methods}
        \begin{method}{public void Parse(RawConfiguration configuration)}{Parses the configuration from the provided value.}
            \begin{parameters}
                \para{RawConfiguration configuration}{The configuration value to parse from.}
            \end{parameters}
        \end{method}
    \end{methods}
\end{class}

\subsection*{MORR.Core.Data.Capture}

\begin{class}{CaptureException}
    \clsdiagram[width=0.3\textwidth]{resources/Classes/Core/MORR/Capture/CaptureException.png}

    \clsdcl{public class CaptureException : Exception}

    \clsdsp{A generic capture exception that all specialized capture exceptions derive from.}
\end{class}

\subsection*{MORR.Core.Data.Capture.Video.Exceptions}

\begin{class}{VideoCaptureException}
    \clsdiagram[width=0.3\textwidth]{resources/Classes/Core/MORR/Capture/VideoCaptureException.png}

    \clsdcl{public class VideoCaptureException : CaptureException}

    \clsdsp{An exception thrown if video sample capturing fails.}
\end{class}

\subsection*{MORR.Core.Data.Capture.Video}

\begin{interface}{IVideoCapture}
    \clsdiagram[width=0.35\textwidth]{resources/Classes/Core/MORR/Capture/IVideoCapture.png}

    \clsdcl{public interface IVideoCapture}

    \clsdsp{Captures video output and provides it on a per-sample basis.}

    \begin{methods}
        \begin{method}{VideoSample NextSample()}{Gets the next VideoSample from the capture.}
            \return{VideoSample}{The next VideoSample from the capture.}
        \end{method}
    \end{methods}
\end{interface}

\subsection*{MORR.Core.Data.Capture.Video.WinAPI}

\begin{interface}{DesktopCapture}
    \clsdiagram[width=0.6\textwidth]{resources/Classes/Core/MORR/Capture/DesktopCapture.png}

    \clsdcl{public class DesktopCapture : IVideoCapture}

    \clsdsp{Captures the desktop using the Windows API and provides the capture samples.}

    \begin{attributes}
        \attribute{[Import] DesktopCaptureConfiguration CaptureConfiguration \{ get; set; \}}{The IConfiguration instance specifying configuration options regarding desktop capture.}
    \end{attributes}

    \begin{methods}
        \begin{method}{VideoSample NextSample()}{Gets the next VideoSample from the capture.}
            \return{VideoSample}{The next VideoSample from the capture.}
        \end{method}
    \end{methods}   
\end{interface}

\begin{class}{DesktopCaptureConfiguration}
    \clsdiagram[width=0.35\textwidth]{resources/Classes/Core/MORR/Capture/DesktopCaptureConfiguration.png}

    \clsdcl{public class DesktopCaptureConfiguration : IConfiguration}

    \clsdsp{Encapsulates all DesktopCapture configuration options.}

    \begin{attributes}
        \attribute{string DeviceToRecord \{ get; set; \}}{The device name of the screen to record.}
    \end{attributes}

    \begin{methods}
        \begin{method}{public void Parse(string configuration)}{Parses the configuration from the provided value.}
            \begin{parameters}
                \para{string configuration}{The configuration value to parse from.}
            \end{parameters}
        \end{method}
    \end{methods}
\end{class}

\subsection*{MORR.Core.Data.Capture.Video.WinAPI.Utility}

\begin{class}{CaptureHelper}
    \clsdiagram[width=0.55\textwidth]{resources/Classes/Core/MORR/Capture/CaptureHelper.png}

    \clsdcl{internal static class CaptureHelper}

    \clsdsp{Provides utility methods for working with Windows' GraphicsCaptureItems.}

    \begin{attributes}
        \attribute{static bool CanCreateItemWithoutPicker \{ get; \}}{Indicates whether the creation of a GraphicsCaptureItem instance requires user interaction. True if GraphicsCaptureItems can be created without user interaction, false otherwise.}
    \end{attributes}
    \begin{methods}
        \begin{method}{public static GraphicsCaptureItem CreateItemForMonitor(IntPtr hMon)}{Creates a GraphicsCaptureItem for the specified monitor.}
            \begin{parameters}
                \para{IntPtr hMon}{The handle of the monitor to create the GraphicsCaptureItem for.}
            \end{parameters}
        \end{method}
    \end{methods}
\end{class}

\begin{class}{Direct3D11Helper}
    \clsdiagram[width=0.8\textwidth]{resources/Classes/Core/MORR/Capture/Direct3D11Helper.png}

    \clsdcl{internal static class Direct3D11Helper}

    \clsdsp{Provides utility methods for using Direct3D11 objects.}

    \begin{methods}
        \begin{method}{public static IDirect3DDevice CreateDevice(bool useWARP = false)}{Creates a new IDirect3DDevice device using the specified driver type.}
            \begin{parameters}
                \para{bool useWARP}{Indicates whether the created device should utilize hardware support or software emulation. True if software emulation should be used, false otherwise.}
            \end{parameters}
            \return{IDirect3DDevice}{The created device.}
        \end{method}
        \begin{method}{public static IDirect3DSurface \\CreateDirect3DSurfaceFromSharpDXTexture(Texture2D texture)}{Creates a IDirect3DSurface from a SharpDX texture.}
            \begin{parameters}
                \para{Texture2D texture}{The SharpDX texture to convert.}
            \end{parameters}
            \return{IDirect3DSurface}{The created Direct3D surface.}
        \end{method}
        \begin{method}{public static Device CreateSharpDXDevice(IDirect3DDevice device)}{Create a SharpDX device from the specified Direct3D device.}
            \begin{parameters}
                \para{IDirect3DDevice device}{The device to create a SharpDX device from.}
            \end{parameters}
            \return{Device}{The created SharpDX device.}
        \end{method}
        \begin{method}{public static Texture2D CreateSharpDXTexture2D(IDirect3DSurface surface)}{Creates a SharpDX texture from the specified Direct3D surface.}
            \begin{parameters}
                \para{IDirect3DSurface surface}{The surface to convert.}
            \end{parameters}
            \return{Texture2D}{The created SharpDX texture.}
        \end{method}
    \end{methods}
\end{class}

\begin{class}{MonitorEnumerationHelper}
    \clsdiagram[width=0.4\textwidth]{resources/Classes/Core/MORR/Capture/MonitorEnumerationHelper.png}

    \clsdcl{internal static class MonitorEnumerationHelper}

    \clsdsp{Provides utility methods for enumerating monitors.}

    \begin{methods}
        \begin{method}{public static IEnumerable<MonitorInfo> GetMonitors()}{Gets MonitorInfo instances containing information about all connected monitors.}
            \return{IEnumerable<MonitorInfo>}{A list of MonitorInfo instances containing information about all connected monitors.}
        \end{method}
    \end{methods}
\end{class}

\begin{class}{MonitorInfo}
    \clsdiagram[width=0.35\textwidth]{resources/Classes/Core/MORR/Capture/MonitorInfo.png}

    \clsdcl{internal class MonitorInfo}

    \clsdsp{Represents information about a connected monitor.}

    \begin{attributes}
        \attribute{public bool IsPrimary \{ get; private set; \}}{Indicates whether the describe monitor is the primary monitor. True if the monitor is the primary monitor, false otherwise.}
        \attribute{public Vector2 ScreenSize \{ get; private set; \}}{The size of the screen as (width, height) vector.}
        \attribute{public Rect MonitorArea \{ get; private set; \}}{The size of the monitor.}
        \attribute{public Rect WorkArea \{ get; private set;\}}{The size of the working area.}
        \attribute{public string DeviceName \{ get; private set;\}}{The name of the device.}
        \attribute{public IntPtr HMon \{ get; private set;\}}{The handle of the monitor.}
    \end{attributes}
\end{class}
\subsection*{MORR.Core.Data.IntermediateFormat}

\begin{absclass}{IntermediateFormatSample}
    \clsdiagram{resources/Classes/Core/MORR/Data/IntermediateFormat/IntermediateFormatSample.png}

    \clsdcl{public abstract class IntermediateFormatSample : Event}

    \clsdsp{A sample in an intermediate format.}

    \begin{attributes}
        \attribute{public Type Type \{ get; set; \}}{The type of the event that is serialized.}
        \attribute{public byte[] Data \{ get; set; \}}{The data that is serialized.}
    \end{attributes}
\end{absclass}

\subsection*{MORR.Core.Data.IntermediateFormat.Json}

\begin{class}{JsonIntermediateFormatSample}
    \clsdiagram{resources/Classes/Core/MORR/Data/IntermediateFormat/Json/JsonIntermediateFormatSample.png}

    \clsdcl{public class JsonIntermediateFormatSample : IntermediateFormatSample}

    \clsdsp{A sample in JSON intermediate format.}

    \begin{attributes}
        \attribute{public JsonDocument JsonEncodedData \{ get; \}}{The data that is serialized in JSON-compatible format.}
        \attribute{public JsonEncodedText JsonEncodedType \{ get; \}}{The type of the event that is serialized in JSON-compatible format.}
    \end{attributes}
\end{class}

\begin{class}{JsonIntermediateFormatSerializer}
    \clsdiagram{resources/Classes/Core/MORR/Data/IntermediateFormat/Json/JsonIntermediateFormatSerializer.png}

    \clsdcl{public class JsonIntermediateFormatSerializer : DefaultEncodeableEventQueue<JsonIntermediateFormatSample>, IModule}

    \clsdsp{A serializer for converting typed events into their representation in the JSON-intermediate format.}

    \begin{attributes}
        \attribute{public bool IsActive \{ get; set; \}}{Indicates whether the module is active. True if it is active, false otherwise.}
        \attribute{public Guid Identifier \{ get; \}}{The identifier of the module.}
    \end{attributes}

    \begin{methods}
        \begin{method}{public bool Initialize(bool isEnabled)}{Initializes the module.}
            \begin{parameters}
                \para{bool isEnabled}{Indicates whether the module is enabled. True if it is enabled, false otherwise.}
            \end{parameters}
        \end{method}
    \end{methods}
\end{class}

\begin{class}{JsonIntermediateFormatDeserializer}
    \clsdiagram{resources/Classes/Core/MORR/Data/IntermediateFormat/Json/JsonIntermediateFormatDeserializer.png}

    \clsdcl{public class JsonIntermediateFormatDeserializer : IModule}

    \clsdsp{A deserializer for converting events in their JSON-intermediate format representation to typed events.}

    \begin{attributes}
        \attribute{public bool IsActive \{ get; set; \}}{Indicates whether the module is active. True if it is active, false otherwise.}
        \attribute{public Guid Identifier \{ get; \}}{The identifier of the module.}
    \end{attributes}

    \begin{methods}
        \begin{method}{public bool Initialize(bool isEnabled)}{Initializes the module.}
            \begin{parameters}
                \para{bool isEnabled}{Indicates whether the module is enabled. True if it is enabled, false otherwise.}
            \end{parameters}
        \end{method}
    \end{methods}
\end{class}
\subsection*{MORR.Core.Data.Transcoding.Metadata.EventHandlers}

\begin{class}{MetadataSampleRequestedEventHandler}
    \clsdiagram[width=0.4\textwidth]{resources/Classes/Core/MORR/Transcoding/Metadata/MetadataSampleRequestedEventHandler.png}

    \clsdcl{public delegate MetadataSample? MetadataSampleRequestedEventHandler()}

    \clsdsp{A handler for the IEncoder.MetadataSampleRequested event.}

    \return{MetadataSample?}{The next MetadataSample to encode or null if there are no more samples.}
\end{class}

\begin{class}{MetadataSampleDecodedEventHandler}
    \clsdiagram[width=0.4\textwidth]{resources/Classes/Core/MORR/Transcoding/Metadata/MetadataSampleDecodedEventHandler.png}

    \clsdcl{public delegate void MetadataSampleDecodedEventHandler(MetadataSample sample)}
    
    \clsdsp{A handler for the IDecoder.MetadataSampleDecoded event.}

    \begin{parameters}
        \para{MetadataSample sample}{The MetadataSample that was decoded.}
    \end{parameters}
\end{class}

\subsection*{MORR.Core.Data.Transcoding.Metadata.Exceptions}

\begin{class}{MetadataEncodingException}
    \clsdiagram[width=0.3\textwidth]{resources/Classes/Core/MORR/Transcoding/Metadata/MetadataEncodingException.png}

    \clsdcl{public MetadataEncodingException : EncodingException}

    \clsdsp{An exception thrown if metadata encoding fails.}
\end{class}

\begin{class}{MetadataDecodingException}
    \clsdiagram[width=0.3\textwidth]{resources/Classes/Core/MORR/Transcoding/Metadata/MetadataDecodingException.png}

    \clsdcl{public MetadataDecodingException : DecodingException}

    \clsdsp{An exception thrown if metadata decoding fails.}
\end{class}

\subsection*{MORR.Core.Data.Transcoding.Metadata}

\begin{interface}{IMetadataDeserializer}
    \clsdiagram[width=0.3\textwidth]{resources/Classes/Core/MORR/Transcoding/Metadata/IMetadataDeserializer.png}

    \clsdcl{public interface IMetadataDeserializer}

    \clsdsp{Provides deserialization from generic events from a decoder to type-specific events.}
\end{interface}

\begin{class}{MetadataDeserializer}
    \clsdiagram[width=0.5\textwidth]{resources/Classes/Core/MORR/Transcoding/Metadata/MetadataDeserializer.png}

    \clsdcl{public class MetadataDeserializer : IMetadataDeserializer}

    \clsdsp{Provides deserialization from generic events from a decoder to type-specific events.}

    \begin{attributes}
        \attribute{[ImportMany] public IEnumerable<EventQueue<Event>> EventQueues \{ get; private set\}}{All available event queues.}
    \end{attributes}
\end{class}
\subsection*{MORR.Core.Data.Transcoding.Video.EventHandlers}

\begin{class}{VideoSampleRequestedEventHandler}
    \clsdcl{public delegate VideoSample? VideoSampleRequestedEventHandler()}

    \clsdsp{A handler for the IEncoder.VideoSampleRequested event.}

    \return{VideoSample?}{The next VideoSample to encode or null if there are no more samples.}
\end{class}

\begin{class}{VideoSampleDecodedEventHandler}
    \clsdcl{public delegate void VideoSampleDecodedEventHandler(VideoSample sample)}
    
    \clsdsp{A handler for the IDecoder.VideoSampleDecoded event.}

    \begin{parameters}
        \para{VideoSample sample}{The VideoSample that was decoded.}
    \end{parameters}
\end{class}

\subsection*{MORR.Core.Data.Transcoding.Video.Exceptions}

\begin{class}{VideoEncodingException}
    \clsdcl{VideoEncodingException : EncodingException}

    \clsdsp{An exception thrown if video encoding fails.}
\end{class}

\begin{class}{VideoDecodingException}
    \clsdcl{VideoDecodingException : DecodingException}

    \clsdsp{An exception thrown if video decoding fails.}
\end{class}
\subsection*{MORR.Core.Data.Transcoding.WinAPI}

\begin{class}{MPEGEncoder}
    \clsdiagram[width=0.7\textwidth]{resources/Classes/Core/MORR/Transcoding/WinAPI/MPEGEncoder.png}

    \clsdcl{public class MPEGEncoder : IEncoder}
    
    \clsdsp{Encodes the provided samples into a file at a path in MPEG format using the Windows API.}

    \begin{attributes}
        \attribute{public event VideoSampleRequestedEventHandler VideoSampleRequested}{Event raised when the next VideoSample may be added for encoding.}
        \attribute{public event MetadataSampleRequestedEventHandler MetadataSampleRequested}{Event raised when the next MetadataSample may be added for encoding.}
        \attribute{public [Import] MPEGEncoderConfiguration EncoderConfiguration \{ get; set; \}}{The IConfiguration instance specifying configuration options regarding MPEG encoding.}
    \end{attributes}

    \begin{methods}
        \begin{method}{public void Encode()}{Encodes all provided samples to a file at the location specified by the configuration.}
            \begin{exceptions}
                \excp{EncodingException}{Exception thrown if the encoding fails. This may be a MetadataEncodingException or a VideoEncodingException.}
            \end{exceptions}
        \end{method}
    \end{methods}
\end{class}

\begin{class}{MPEGEncoderConfiguration}
    \clsdiagram[width=0.35\textwidth]{resources/Classes/Core/MORR/Transcoding/WinAPI/MPEGEncoderConfiguration.png}

    \clsdcl{public class MPEGEncoderConfiguration : IConfiguration}

    \clsdsp{Encapsulates all MPEGEncoder configuration options.}

    \begin{attributes}
        \attribute{public ushort Bitrate \{ get; set; \}}{The bitrate in bits per second of the video stream.}
        \attribute{public byte Framerate \{ get; set; \}}{The framerate in frames per second of the video stream.}
        \attribute{public Size Resolution \{ get; set; \}}{The resolution in pixels of the video stream.}
        \attribute{public string Directory \{ get; set; \}}{The directory to store the encoded file in.}
    \end{attributes}

    \begin{methods}
        \begin{method}{public void Parse(string configuration)}{Parses the configuration from the provided value.}
            \begin{parameters}
                \para{string configuration}{The configuration value to parse from.}
            \end{parameters}
        \end{method}
    \end{methods}
\end{class}

\begin{class}{MPEGDecoder}
    \clsdiagram[width=0.7\textwidth]{resources/Classes/Core/MORR/Transcoding/WinAPI/MPEGDecoder.png}

    \clsdcl{public class MPEGDecoder : IDecoder}

    \clsdsp{Decodes samples stored in a file in MPEG format at a path using the Windows API and provides the decoded samples.}

    \begin{attributes}
        \attribute{public event VideoSampleDecodedEventHandler VideoSampleDecoded}{Event raised when a VideoSample gets decoded.}
        \attribute{public event MetadataSampleDecodedEventHandler MetadataSampleDecoded}{Event raised when a MetadataSample gets decoded.}
        \attribute{public [Import] MPEGDecoderConfiguration DecoderConfiguration \{ get; set; \}}{The IConfiguration instance specifying configuration options regarding MPEG decoding.}
    \end{attributes}

    \begin{methods}
        \begin{method}{public void DecodeFrom(string path)}{Decodes samples contained in a file at a specified location.}
            \begin{parameters}
                \para{string path}{The path to decode from.}
            \end{parameters}
            \begin{exceptions}
                \excp{DecodingException}{Exception thrown if decoding fails. This may be a MetadataDecodingException or a VideoDecodingException.}
            \end{exceptions}
        \end{method}
    \end{methods}
\end{class}

\begin{class}{MPEGDecoderConfiguration}
    \clsdiagram[width=0.35\textwidth]{resources/Classes/Core/MORR/Transcoding/WinAPI/MPEGDecoderConfiguration.png}

    \clsdcl{public class MPEGDecoderConfiguration : IConfiguration}

    \clsdsp{Encapsulates all MPEGDecoder configuration options.}

    \begin{attributes}
        \attribute{bool ShouldExtractVideo \{ get; set; \}}{Indicates whether the encoder should extract video in addition to metadata. True if the encoder should extract video, false otherwise.}
    \end{attributes}

    \begin{methods}
        \begin{method}{public void Parse(string configuration)}{Parses the configuration from the provided value.}
            \begin{parameters}
                \para{string configuration}{The configuration value to parse from.}
            \end{parameters}
        \end{method}
    \end{methods}
\end{class}

\begin{interface}{IEncoder}
    \clsdcl{public interface IEncoder}

    \clsdsp{Encodes the provided samples into a file at a path.}

    \begin{attributes}
        \attribute{event VideoSampleRequestedEventHandler VideoSampleRequested}{Event raised when the next VideoSample may be added for encoding.}
        \attribute{event MetadataSampleRequestedEventHandler MetadataSampleRequested}{Event raised when the next MetadataSample may be added for encoding.}
    \end{attributes}

    \begin{methods}
        \begin{method}{void EncodeTo(string path)}{Encodes all provided sample to a file at the specified location.}
            \begin{parameters}
                \para{string path}{The path to encode to.}
            \end{parameters}
            \begin{exceptions}
                \excp{EncodingException}{Exception thrown if the encoding fails. This may be a MetadataEncodingException or a VideoEncodingException.}
            \end{exceptions}
        \end{method}
    \end{methods}
\end{interface}

\begin{interface}{IDecoder}
    \clsdcl{public interface IDecoder}

    \clsdsp{Decodes samples stored in a file at a path and provides the decoded samples.}

    \begin{attributes}
        \attribute{event VideoSampleDecodedEventHandler VideoSampleDecoded}{Event raised when a VideoSample gets decoded.}
        \attribute{event MetadataSampleDecodedEventHandler MetadataSampleDecoded}{Event raised when a MetadataSample gets decoded.}
    \end{attributes}

    \begin{methods}
        \begin{method}{void DecodeFrom(string path)}{Decodes samples contained in a file at a specified location.}
            \begin{parameters}
                \para{string path}{The path to decode from.}
            \end{parameters}
            \begin{exceptions}
                \excp{DecodingException}{Exception thrown if decoding fails. This may be a MetadataDecodingException or a VideoDecodingException.}
            \end{exceptions}
        \end{method}
    \end{methods}
\end{interface}

\begin{class}{EncodingException}
    \clsdcl{public class EncodingException : Exception}

    \clsdsp{A generic encoding exception that all specialized encoding exceptions derive from.}
\end{class}

\begin{class}{DecodingException}
    \clsdcl{public class DecodingException : Exception}

    \clsdsp{A generic decoding exception that all specialized decoding exceptions derive from.}
\end{class}