\subsection*{MORR.Core.Data.Capture}

\begin{class}{CaptureException}
    \clsdiagram[width=0.3\textwidth]{resources/Classes/Core/MORR/Capture/CaptureException.png}

    \clsdcl{public class CaptureException : Exception}

    \clsdsp{A generic capture exception that all specialized capture exceptions derive from.}
\end{class}

\subsection*{MORR.Core.Data.Capture.Video.Exceptions}

\begin{class}{VideoCaptureException}
    \clsdiagram[width=0.3\textwidth]{resources/Classes/Core/MORR/Capture/VideoCaptureException.png}

    \clsdcl{public class VideoCaptureException : CaptureException}

    \clsdsp{An exception thrown if video sample capturing fails.}
\end{class}

\subsection*{MORR.Core.Data.Capture.Video}

\begin{interface}{IVideoCapture}
    \clsdiagram[width=0.35\textwidth]{resources/Classes/Core/MORR/Capture/IVideoCapture.png}

    \clsdcl{public interface IVideoCapture}

    \clsdsp{Captures video output and provides it on a per-sample basis.}

    \begin{methods}
        \begin{method}{VideoSample NextSample()}{Gets the next VideoSample from the capture.}
            \return{VideoSample}{The next VideoSample from the capture.}
        \end{method}
    \end{methods}
\end{interface}

\subsection*{MORR.Core.Data.Capture.Video.WinAPI}

\begin{interface}{DesktopCapture}
    \clsdiagram[width=0.6\textwidth]{resources/Classes/Core/MORR/Capture/DesktopCapture.png}

    \clsdcl{public class DesktopCapture : IVideoCapture}

    \clsdsp{Captures the desktop using the Windows API and provides the capture samples.}

    \begin{attributes}
        \attribute{[Import] DesktopCaptureConfiguration CaptureConfiguration \{ get; set; \}}{The IConfiguration instance specifying configuration options regarding desktop capture.}
    \end{attributes}

    \begin{methods}
        \begin{method}{VideoSample NextSample()}{Gets the next VideoSample from the capture.}
            \return{VideoSample}{The next VideoSample from the capture.}
        \end{method}
    \end{methods}   
\end{interface}

\begin{class}{DesktopCaptureConfiguration}
    \clsdiagram[width=0.35\textwidth]{resources/Classes/Core/MORR/Capture/DesktopCaptureConfiguration.png}

    \clsdcl{public class DesktopCaptureConfiguration : IConfiguration}

    \clsdsp{Encapsulates all DesktopCapture configuration options.}

    \begin{attributes}
        \attribute{string DeviceToRecord \{ get; set; \}}{The device name of the screen to record.}
    \end{attributes}

    \begin{methods}
        \begin{method}{public void Parse(string configuration)}{Parses the configuration from the provided value.}
            \begin{parameters}
                \para{string configuration}{The configuration value to parse from.}
            \end{parameters}
        \end{method}
    \end{methods}
\end{class}

\subsection*{MORR.Core.Data.Capture.Video.WinAPI.Utility}

\begin{class}{CaptureHelper}
    \clsdiagram[width=0.55\textwidth]{resources/Classes/Core/MORR/Capture/CaptureHelper.png}

    \clsdcl{internal static class CaptureHelper}

    \clsdsp{Provides utility methods for working with Windows' GraphicsCaptureItems.}

    \begin{attributes}
        \attribute{static bool CanCreateItemWithoutPicker \{ get; \}}{Indicates whether the creation of a GraphicsCaptureItem instance requires user interaction. True if GraphicsCaptureItems can be created without user interaction, false otherwise.}
    \end{attributes}
    \begin{methods}
        \begin{method}{public static GraphicsCaptureItem CreateItemForMonitor(IntPtr hMon)}{Creates a GraphicsCaptureItem for the specified monitor.}
            \begin{parameters}
                \para{IntPtr hMon}{The handle of the monitor to create the GraphicsCaptureItem for.}
            \end{parameters}
        \end{method}
    \end{methods}
\end{class}

\begin{class}{Direct3D11Helper}
    \clsdiagram[width=0.8\textwidth]{resources/Classes/Core/MORR/Capture/Direct3D11Helper.png}

    \clsdcl{internal static class Direct3D11Helper}

    \clsdsp{Provides utility methods for using Direct3D11 objects.}

    \begin{methods}
        \begin{method}{public static IDirect3DDevice CreateDevice(bool useWARP = false)}{Creates a new IDirect3DDevice device using the specified driver type.}
            \begin{parameters}
                \para{bool useWARP}{Indicates whether the created device should utilize hardware support or software emulation. True if software emulation should be used, false otherwise.}
            \end{parameters}
            \return{IDirect3DDevice}{The created device.}
        \end{method}
        \begin{method}{public static IDirect3DSurface \\CreateDirect3DSurfaceFromSharpDXTexture(Texture2D texture)}{Creates a IDirect3DSurface from a SharpDX texture.}
            \begin{parameters}
                \para{Texture2D texture}{The SharpDX texture to convert.}
            \end{parameters}
            \return{IDirect3DSurface}{The created Direct3D surface.}
        \end{method}
        \begin{method}{public static Device CreateSharpDXDevice(IDirect3DDevice device)}{Create a SharpDX device from the specified Direct3D device.}
            \begin{parameters}
                \para{IDirect3DDevice device}{The device to create a SharpDX device from.}
            \end{parameters}
            \return{Device}{The created SharpDX device.}
        \end{method}
        \begin{method}{public static Texture2D CreateSharpDXTexture2D(IDirect3DSurface surface)}{Creates a SharpDX texture from the specified Direct3D surface.}
            \begin{parameters}
                \para{IDirect3DSurface surface}{The surface to convert.}
            \end{parameters}
            \return{Texture2D}{The created SharpDX texture.}
        \end{method}
    \end{methods}
\end{class}

\begin{class}{MonitorEnumerationHelper}
    \clsdiagram[width=0.4\textwidth]{resources/Classes/Core/MORR/Capture/MonitorEnumerationHelper.png}

    \clsdcl{internal static class MonitorEnumerationHelper}

    \clsdsp{Provides utility methods for enumerating monitors.}

    \begin{methods}
        \begin{method}{public static IEnumerable<MonitorInfo> GetMonitors()}{Gets MonitorInfo instances containing information about all connected monitors.}
            \return{IEnumerable<MonitorInfo>}{A list of MonitorInfo instances containing information about all connected monitors.}
        \end{method}
    \end{methods}
\end{class}

\begin{class}{MonitorInfo}
    \clsdiagram[width=0.35\textwidth]{resources/Classes/Core/MORR/Capture/MonitorInfo.png}

    \clsdcl{internal class MonitorInfo}

    \clsdsp{Represents information about a connected monitor.}

    \begin{attributes}
        \attribute{public bool IsPrimary \{ get; private set; \}}{Indicates whether the describe monitor is the primary monitor. True if the monitor is the primary monitor, false otherwise.}
        \attribute{public Vector2 ScreenSize \{ get; private set; \}}{The size of the screen as (width, height) vector.}
        \attribute{public Rect MonitorArea \{ get; private set; \}}{The size of the monitor.}
        \attribute{public Rect WorkArea \{ get; private set;\}}{The size of the working area.}
        \attribute{public string DeviceName \{ get; private set;\}}{The name of the device.}
        \attribute{public IntPtr HMon \{ get; private set;\}}{The handle of the monitor.}
    \end{attributes}
\end{class}

\subsection*{MORR.Core.Data.Capture.Metadata}

\begin{interface}{IMetadataCapture}
    \clsdiagram[width=0.35\textwidth]{resources/Classes/Core/MORR/Capture/IMetadataCapture.png}

    \clsdcl{public interface IMetadataCapture : IReceivingModule}

    \clsdsp{Captures all events from all event queues and provides them on a per-sample basis.}

    \begin{methods}
        \begin{method}{MetadataSample NextSample()}{Gets the next MetadataSample from the capture.}
            \return{MetadataSample}{The next MetadataSample from the capture.}
        \end{method}
    \end{methods}
\end{interface}