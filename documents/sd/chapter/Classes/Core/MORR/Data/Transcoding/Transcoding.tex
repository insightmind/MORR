\subsection*{MORR.Core.Data.Transcoding}

\begin{interface}{IEncoder}
    \clsdiagram{resources/Classes/Core/MORR/Data/Transcoding/IEncoder.png}

    \clsdcl{public interface IEncoder}

    \clsdsp{Encodes the provided samples into a file at a path.}

    \begin{attributes}
        \attribute{ManualResetEvent EncodeFinished \{ get; \}}{An event raised when encoding finishes.}
    \end{attributes}

    \begin{methods}
        \begin{method}{void Encode(DirectoryPath recordingDirectoryPath)}{Encodes all provided samples to a recording at the specified location.}
            \begin{parameters}
                \para{DirectPath recordingDirectoryPath}{The DirectoryPath to contain the recording.}
            \end{parameters}
            \begin{exceptions}
                \excp{EncodingException}{Exception thrown if the encoding fails.}
            \end{exceptions}
        \end{method}
    \end{methods}
\end{interface}

\begin{interface}{IDecoder}
    \clsdiagram{resources/Classes/Core/MORR/Data/Transcoding/IDecoder.png}

    \clsdcl{public interface IDecoder}

    \clsdsp{Decodes samples stored in a file at a path and provides the decoded samples.}

    \begin{attributes}
        \attribute{ManualResetEvent DecodeFinished \{ get; \}}{An event raised when decoding finishes.}
    \end{attributes}

    \begin{methods}
        \begin{method}{void Decode(DirectoryPath recordingDirectoryPath)}{Decodes samples contained in a recording at a specified location.}
            \begin{parameters}
                \para{DirectoryPath recordingDirectoryPath}{The DirectoryPath of the file to decode from.}
            \end{parameters}
            \begin{exceptions}
                \excp{DecodingException}{Exception thrown if decoding fails.}
            \end{exceptions}
        \end{method}
    \end{methods}
\end{interface}

\subsection*{MORR.Core.Data.Transcoding.Exceptions}

\begin{class}{EncodingException}
    \clsdiagram{resources/Classes/Core/MORR/Data/Transcoding/Exceptions/EncodingException.png}

    \clsdcl{public class EncodingException : Exception}

    \clsdsp{A generic encoding exception that all specialized encoding exceptions derive from.}

    \begin{constructors}
        \begin{constructor}{public EncodingException()}{Creates a new instance of a EncodingException without any specific error message.}
        \end{constructor}
        \begin{constructor}{public EncodingException(string message)}{Creates a new instance of a EncodingException with the specified error message.}
            \begin{parameters}
                \para{string message}{The error message to construct the instance with.}
            \end{parameters}
        \end{constructor}
        \begin{constructor}{public EncodingException(string message, Exception innerException)}{Creates a new instance of a EncodingException with the specified error message and inner exception.}
            \begin{parameters}
                \para{string message}{The error message to construct the instance with.}
                \para{Exception innerException}{The inner exception to construct the instance with.}
            \end{parameters}
        \end{constructor}
    \end{constructors}
\end{class}

\begin{class}{DecodingException}
    \clsdiagram{resources/Classes/Core/MORR/Data/Transcoding/Exceptions/DecodingException.png}

    \clsdcl{public class DecodingException : Exception}

    \clsdsp{A generic decoding exception that all specialized decoding exceptions derive from.}

    \begin{constructors}
        \begin{constructor}{public DecodingException()}{Creates a new instance of a DecodingException without any specific error message.}
        \end{constructor}
        \begin{constructor}{public DecodingException(string message)}{Creates a new instance of a DecodingException with the specified error message.}
            \begin{parameters}
                \para{string message}{The error message to construct the instance with.}
            \end{parameters}
        \end{constructor}
        \begin{constructor}{public DecodingException(string message, Exception innerException)}{Creates a new instance of a DecodingException with the specified error message and inner exception.}
            \begin{parameters}
                \para{string message}{The error message to construct the instance with.}
                \para{Exception innerException}{The inner exception to construct the instance with.}
            \end{parameters}
        \end{constructor}
    \end{constructors}
\end{class}

\subsection*{MORR.Core.Data.Transcoding.Json}

\begin{class}{JsonDecoder}
    \clsdiagram{resources/Classes/Core/MORR/Data/Transcoding/Json/JsonDecoder.png}

    \clsdcl{public class JsonDecoder : DefaultDecodeableEventQueue<JsonIntermediateFormatSample>, IDecoder}
    
    \clsdsp{Decodes samples from a recording previously created with the JsonEncoder.}

    \begin{attributes}
        \attribute{public ManualResetEvent DecodeFinished \{ get; \}}{An event raised when decoding finishes.}
    \end{attributes}

    \begin{methods}
        \begin{method}{void Decode(DirectoryPath recordingDirectoryPath)}{Decodes samples contained in a recording at a specified location.}
            \begin{parameters}
                \para{DirectoryPath recordingDirectoryPath}{The DirectoryPath of the file to decode from.}
            \end{parameters}
            \begin{exceptions}
                \excp{DecodingException}{Exception thrown if decoding fails.}
            \end{exceptions}
        \end{method}
    \end{methods}
\end{class}

\begin{class}{JsonEncoder}
    \clsdiagram{resources/Classes/Core/MORR/Data/Transcoding/Json/JsonEncoder.png}

    \clsdcl{public class JsonEncoder : IEncoder}

    \clsdsp{Encodes samples into a recording in JSON format.}

    \begin{attributes}
        \attribute{public ManualResetEvent EncodeFinished \{ get; \}}{An event raised when encoding finishes.}
    \end{attributes}

    \begin{methods}
        \begin{method}{void Encode(DirectoryPath recordingDirectoryPath)}{Encodes all provided samples to a recording at the specified location.}
            \begin{parameters}
                \para{DirectoryPath recordingDirectoryPath}{The DirectoryPath of the file to encode to.}
            \end{parameters}
            \begin{exceptions}
                \excp{EncodingException}{Exception thrown if encoding fails.}
            \end{exceptions}
        \end{method}
    \end{methods}
\end{class}

\begin{class}{JsonDecoderConfiguration}
    \clsdiagram{resources/Classes/Core/MORR/Data/Transcoding/Json/JsonDecoderConfiguration.png}

    \clsdcl{public class JsonDecoderConfiguration : IConfiguration}

    \clsdsp{Encapsulates all JSON decoding configuration options.}

    \begin{attributes}
        \attribute{public FilePath RelativeFilePath \{ get; set; \}}{The path to the file to store the data in relative to the recording directory.}
    \end{attributes}
    
    \begin{methods}
        \begin{method}{public void Parse(RawConfiguration configuration)}{Parses the configuration from the provided value.}
            \begin{parameters}
                \para{RawConfiguration configuration}{The configuration value to parse from.}
            \end{parameters}
        \end{method}
    \end{methods}
\end{class}

\begin{class}{JsonEncoderConfiguration}
    \clsdiagram{resources/Classes/Core/MORR/Data/Transcoding/Json/JsonEncoderConfiguration.png}

    \clsdcl{public class JsonEncoderConfiguration : IConfiguration}

    \clsdsp{Encapsulates all JSON encoding configuration options.}

    \begin{attributes}
        \attribute{public FilePath RelativeFilePath \{ get; set; \}}{The path to the file to store the data in relative to the recording directory.}
    \end{attributes}
    
    \begin{methods}
        \begin{method}{public void Parse(RawConfiguration configuration)}{Parses the configuration from the provided value.}
            \begin{parameters}
                \para{RawConfiguration configuration}{The configuration value to parse from.}
            \end{parameters}
        \end{method}
    \end{methods}
\end{class}

\subsection*{MORR.Core.Data.Transcoding.Mpeg}

\begin{class}{MpegEncoder}
    \clsdiagram{resources/Classes/Core/MORR/Data/Transcoding/Mpeg/MPEGEncoder.png}

    \clsdcl{public class MpegEncoder : IEncoder}

    \clsdsp{Encodes samples into a recording in MPEG format.}

    \begin{attributes}
        \attribute{public ManualResetEvent EncodeFinished \{ get; \}}{An event raised when encoding finishes.}
    \end{attributes}

    \begin{methods}
        \begin{method}{void Encode(DirectoryPath recordingDirectoryPath)}{Encodes all provided samples to a recording at the specified location.}
            \begin{parameters}
                \para{DirectoryPath recordingDirectoryPath}{The DirectoryPath of the file to encode to.}
            \end{parameters}
            \begin{exceptions}
                \excp{EncodingException}{Exception thrown if encoding fails.}
            \end{exceptions}
        \end{method}
    \end{methods}
\end{class}

\begin{class}{MpegEncoderConfiguration}
    \clsdiagram{resources/Classes/Core/MORR/Data/Transcoding/Mpeg/MPEGEncoderConfiguration.png}

    \clsdcl{public class MpegEncoderConfiguration : IConfiguration}

    \clsdsp{Encapsulates all MPEG encoding configuration options.}

    \begin{attributes}
        \attribute{public uint Width \{ get; set; \}}{The width in pixels of the resulting video stream.}
        \attribute{public uint Height \{ get; set; \}}{The height in pixels of the resulting video stream.}
        \attribute{public uint KiloBitsPerSecond \{ get; set; \}}{The bit rate in kilo bits per second of the resulting video stream.}
        \attribute{public uint FramesPerSecond \{ get; set; \}}{The frame rate in frames per second of the resulting video stream.}
        \attribute{public FilePath RelativeFilePath \{ get; set; \}}{The path to the file to store the data in relative to the recording directory.}
    \end{attributes}
    
    \begin{methods}
        \begin{method}{public void Parse(RawConfiguration configuration)}{Parses the configuration from the provided value.}
            \begin{parameters}
                \para{RawConfiguration configuration}{The configuration value to parse from.}
            \end{parameters}
        \end{method}
    \end{methods}
\end{class}