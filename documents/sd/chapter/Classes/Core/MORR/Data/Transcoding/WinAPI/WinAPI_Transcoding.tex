\subsection*{MORR.Core.Data.Transcoding.WinAPI}

\begin{class}{MPEGEncoder}
    \clsdcl{public class MPEGEncoder : IEncoder}
    
    \clsdsp{Encodes the provided samples into a file at a path in MPEG format using the Windows API.}

    \begin{attributes}
        \attribute{public event VideoSampleRequestedEventHandler VideoSampleRequested}{Event raised when the next VideoSample may be added for encoding.}
        \attribute{public event MetadataSampleRequestedEventHandler MetadataSampleRequested}{Event raised when the next MetadataSample may be added for encoding.}
        \attribute{public [Import] MPEGEncoderConfiguration EncoderConfiguration \{ get; set; \}}{The IConfiguration instance specifying configuration options regarding MPEG encoding.}
    \end{attributes}

    \begin{methods}
        \begin{method}{public void Encode()}{Encodes all provided sample to a file at the location specified by the configuration.}
            \begin{exceptions}
                \excp{EncodingException}{Exception thrown if the encoding fails. This may be a MetadataEncodingException or a VideoEncodingException.}
            \end{exceptions}
        \end{method}
    \end{methods}
\end{class}

\begin{class}{MPEGEncoderConfiguration}
    \clsdcl{public class MPEGEncoderConfiguration : IConfiguration}

    \clsdsp{Encapsulates all MPEGEncoder configuration options.}

    \begin{attributes}
        \attribute{public ushort Bitrate \{ get; set; \}}{The bitrate in bits per second of the video stream.}
        \attribute{public byte Framerate \{ get; set; \}}{The framerate in frames per second of the video stream.}
        \attribute{public Size Resolution \{ get; set; \}}{The resolution in pixels of the video stream.}
        \attribute{public string Directory \{ get; set; \}}{The directory to store the encoded file in.}
    \end{attributes}

    \begin{methods}
        \begin{method}{public void Parse(string configuration)}{Parses the configuration from the provided value.}
            \begin{parameters}
                \para{string configuration}{The configuration value to parse from.}
            \end{parameters}
        \end{method}
    \end{methods}
\end{class}{MPEGEncoderConfiguration}

\begin{class}{MPEGDecoder}
    \clsdcl{public class MPEGDecoder : IDecoder}

    \clsdsp{Decodes samples stored in a file in MPEG format at a path using the Windows API and provides the decoded samples.}

    \begin{attributes}
        \attribute{public event VideoSampleDecodedEventHandler VideoSampleDecoded}{Event raised when a VideoSample gets decoded.}
        \attribute{public event MetadataSampleDecodedEventHandler MetadataSampleDecoded}{Event raised when a MetadataSample gets decoded.}
        \attribute{public [Import] MPEGDecoderConfiguration DecoderConfiguration \{ get; set; \}}{The IConfiguration instance specifying configuration options regarding MPEG decoding.}
    \end{attributes}

    \begin{methods}
        \begin{method}{public void DecodeFrom(string path)}{Decodes samples contained in a file at a specified location.}
            \begin{parameters}
                \para{string path}{The path to decode from.}
            \end{parameters}
            \begin{exceptions}
                \excp{DecodingException}{Exception thrown if decoding fails. This may be a MetadataDecodingException or a VideoDecodingException.}
            \end{exceptions}
        \end{method}
    \end{methods}
\end{class}

\begin{class}{MPEGDecoderConfiguration}
    \clsdcl{public class MPEGDecoderConfiguration : IConfiguration}

    \clsdsp{Encapsulates all MPEGDecoder configuration options.}

    \begin{attributes}
        \attribute{bool ShouldExtractVideo \{ get; set; \}}{Indicates whether the encoder should extract video in addition to metadata. True if the encoder should extract video, false otherwise.}
    \end{attributes}

    \begin{methods}
        \begin{method}{public void Parse(string configuration)}{Parses the configuration from the provided value.}
            \begin{parameters}
                \para{string configuration}{The configuration value to parse from.}
            \end{parameters}
        \end{method}
    \end{methods}
\end{class}