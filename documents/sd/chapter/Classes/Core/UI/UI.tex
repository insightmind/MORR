\subsection{UI}

\begin{class}{App}
	\clsdiagram[scale = 0.7]{resources/Classes/Core/UI/App.png}

    \clsdcl{public partial class App : Application}

    \clsdsp{Auto-generated WPF application class}

\end{class}

\subsection*{Controls. NotifyIcon}
\begin{class}{NotifyIcon}
	\clsdiagram[scale = 0.7]{resources/Classes/Core/UI/NotifyIcon.png}

    \clsdcl{public class NotifyIcon : Control, IDisposable}

    \clsdsp{Manages a tray icon}

    \begin{attributes}
        \attribute{public Uri IconUri}{The uri of the icon shown in the tray.}
        \attribute{public string Tooltip}{The tooltip shown during hovering}
        \attribute{public ICommand Command}{The command to execute on left click}
        \attribute{public object CommandParameter}{The parameter of the command to execute on left click}
        \attribute{public new ContextMenu ContextMenu}{The context menu to show on right click}
    \end{attributes}

   	\begin{constructors}
        \begin{constructor}{public NotifyIcon()}{Initializes a NotifyIcon}
        \end{constructor}
    \end{constructors}
    
    \begin{methods}
    	\begin{method}{public void Dispose()}{Frees all unmanaged resources}
        \end{method}
     \end{methods}

\end{class}

\subsection*{Dialogs}

\begin{class}{ConfirmationDialog}
	\clsdiagram[scale = 1]{resources/Classes/Core/UI/ConfirmationDialog.png}

    \clsdcl{public partial class ConfirmationDialog : Window}

    \clsdsp{A dialog shown when user clicks on the "Discard" button of the SaveDialog.}
    
    \begin{constructors}
        \begin{constructor}{public ConfirmationDialog()}{Initializes a ConfirmationDialog.}
        \end{constructor}
     \end{constructors}

\end{class}

\begin{class}{ErrorDialog}
	\clsdiagram[scale = 0.7]{resources/Classes/Core/UI/ErrorDialog.png}

    \clsdcl{public partial class ErrorDialog : Window}

    \clsdsp{A dialog shown when an error occured}

    \begin{attributes}
        \attribute{public string ErrorMessage}{Error message of the error dialog}
    \end{attributes}
    
    \begin{constructors}
        \begin{constructor}{public ErrorDialog(string errorMessage)}{Initializes an ErrorDialog with errorMessage}
            \begin{parameters}
                \para{String errorMessage}{A string with an error message}
            \end{parameters}
        \end{constructor}
     \end{constructors}

\end{class}

\begin{class}{InfomationDialog}
	\clsdiagram[scale = 0.7]{resources/Classes/Core/UI/InformationDialog.png}

    \clsdcl{public partial class InformationDialog : Window }

    \clsdsp{A dialog shown before the start of the recording}

     \begin{constructors}
        \begin{constructor}{public InformationDialog()}{Initializes an InformationDialog}
        \end{constructor}
     \end{constructors}

\end{class}

\begin{class}{SaveDialog}
	\clsdiagram[scale = 0.7]{resources/Classes/Core/UI/SaveDialog.png}

    \clsdcl{public partial class SaveDialog : Window}

    \clsdsp{A dialog shown after the end of the recording}
    
    \begin{constructors}
        \begin{constructor}{public SaveDialog()}{Initializes a SaveDialog}
        \end{constructor}
     \end{constructors}

\end{class}

\subsection*{ViewModels}

\begin{class}{ApplicationViewModel}
	\clsdiagram[scale = 0.7]{resources/Classes/Core/UI/ApplicationViewModel.png}

    \clsdcl{public class ApplicationViewModel : DependencyObject}

    \clsdsp{}

	\begin{attributes}
        \attribute{public bool IsRecording}{Current state of the recording process}
    \end{attributes}    
    
    \begin{constructors}
        \begin{constructor}{public ApplicationViewModel()}{Initializes an ApplicationViewModel}
        \end{constructor}
     \end{constructors}
 

\end{class}

\subsection*{Utility}

\begin{class}{RelayCommand}
	\clsdiagram[scale = 0.7]{resources/Classes/Core/UI/RelayCommand.png}

    \clsdcl{public RelayCommand : ICommand }

    \clsdsp{A concrete command that implements ICommand interface}

	\begin{constructors}
        \begin{constructor}{public RelayCommand(Action<object> execute, Predicate<object>? canExecute = null)}{Creates a new RelayCommand}
        \begin{parameters}
                \para{Action<object> execute}{The execution logic}
                \para{Predicate<object>? canExecute = null}{The execution status logic, set to null}
            \end{parameters}
        \end{constructor}
     \end{constructors}
    
    
    \begin{methods}
    	\begin{method}{public bool CanExecute(object parameter)}{Determines whether the RelayCommand can be executed in current state}
            \return{bool}{True if this RelayCommand can be executed; otherwise, false.}
            \begin{parameters}
                \para{object parameter}{Data used by the RelayCommand}
            \end{parameters}
        \end{method}
        \begin{method}{public void Execute(object parameter)}{Executes the RelayCommand}
            \begin{parameters}
               \para{object parameter}{Data used by the RelayCommand}
            \end{parameters}
        \end{method}
        
    \end{methods}

\end{class}

