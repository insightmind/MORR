\subsection{Shared}
\begin{interface}{IEvent}

\clsdiagram[width=0.5\textwidth]{resources/Classes/BrowserExtension/Shared/IEvent.png}

\clsdcl{interface IEvent}

\clsdsp{Defines the same interface to use for recorded events as present in the MORR application.}

\begin{attributes}
\attribute{public timeStamp : Date}{The time at which the event was created.}
\attribute{public issuingModule : number}{The ID of the browser module. This is a fixed value in the context of the browser extension.}
\attribute{public type : EventType}{The type of the event.}
\end{attributes}
\begin{methods}
\begin{method}{public serialize() : string}{Serialize the event to a JSON string.}
\return{string}{The JSON string encoding the event.}
\end{method}
\end{methods}
\end{interface}

\begin{class}{BrowserEvent}

\clsdiagram[width=0.5\textwidth]{resources/Classes/BrowserExtension/Shared/BrowserEvent.png}

\clsdcl{class BrowserEvent implements IEvent}

\clsdsp{Implements the interface IEvent and extends it by browser specific, generic event data.}

\begin{attributes}
\attribute{public timeStamp : Date}{The time at which the event was created.}
\attribute{public issuingModule : number}{The ID of the browser module. This is a fixed value in the context of the browser extension.}
\attribute{public type : EventType}{The type of the event.}
\attribute{public windowID : number}{The ID of the browser window the event occured in. According to the functional specification this does not need to be serialized.}
\attribute{public tabID : number}{The ID of the tab the event occured in.}
\attribute{public url : URL}{The URL that was opened in the tab with ID tabID as the event occured.}
\end{attributes}
\begin{constructors}
\begin{constructor}{constructor(type : EventType, tabID : number, windowID: number, url : string)}{Create a new generic browser event.}
\begin{parameters}
\para{type : EventType}{The type to set for the event.}
\para{tabID : number}{The tabID to set for the event.}
\para{windowID : number}{The windowID to set for the event.}
\para{url : string}{The url to set for the event.}
\end{parameters}
\end{constructor}
\end{constructors}
\begin{methods}
\begin{method}{public serialize() : string}{Serialize the event to a JSON string.}
\return{string}{The JSON string encoding the event.}
\end{method}
\end{methods}
\end{class}

\begin{interface}{IListener}

\clsdiagram[width=0.5\textwidth]{resources/Classes/BrowserExtension/Shared/IListener.png}

\clsdcl{interface IListener}

\clsdsp{A listener is a class responsible for recording certain browser-specific events and sending them to the BackgroundScript.}

\begin{constructors}
\begin{constructor}{constructor(callback: (event : BrowserEvent) => void)}{Create a new listener.}
\begin{parameters}
\para{callback : (event : BrowserEvent) => void}{The function to invoke on created events in order to send them back to the BackgroundScript.}
\end{parameters}
\end{constructor}
\end{constructors}
\begin{methods}
\begin{method}{public start() : void}{Start the listener. The listener will generate events and invoke the callback method on them until it is stopped.}
\end{method}
\begin{method}{public stop() : void}{Stop the listener.}
\end{method}
\end{methods}
\end{interface}