\subsection{Content}
\begin{class}{DOMEventRecorder}

\clsdiagram[width=0.5\textwidth]{resources/Classes/BrowserExtension/Content/DOMEventRecorder.png}

\clsdcl{class DOMEventRecorder}

\clsdsp{Responsible for recording the events occuring in the context of websites and sending them back to the DOMEventListener.}

\begin{methods}
\begin{method}{public start() : void}{Start recording DOM events and sending the respective data back to the DOMEventListener.}
\end{method}
\begin{method}{public stop() : void}{Stop recording DOM events and sending them back. When this is issued, the contentscript will terminate.
    To launch the DOMEventRecorder again, the contentscript has to be injected again by the DOMEventListener.}
\end{method}
\end{methods}
\end{class}

\begin{class}{DOMEventFactory}

\clsdiagram[width=1.0\textwidth]{resources/Classes/BrowserExtension/Content/DOMEventFactory.png}

\clsdcl{class DOMEventFactory}

\clsdsp{The DOMEventFactory is responsible for creating BrowserEvent objects from the DOM-Event objects created by the browser upon user-interaction from a website.}

\begin{methods}
\begin{method}{createEvent(domEvent : Event) : Promise<BrowserEvent | undefined>}{Asynchronously create a new BrowserEvent from an Event object raised by the browser.}
\return{Promise<BrowserEvent | undefined>}{A promise which will be filled with an event if one could be created from the passed domEvent, or filled with undefined otherwise.}
\begin{parameters}
\para{domEvent : Event}{The DOM event raised by the browser.}
\end{parameters}
\end{method}
\end{methods}
\end{class}