\subsection{Listeners}
%%Tab%%
\begin{class}{TabListener}

\clsdiagram[width=1.0\textwidth]{resources/Classes/BrowserExtension/Listeners/TabListener.png}

\clsdcl{class TabListener implements IListener}

\clsdsp{Listener responsible for recording events connected to the tab-API.}

\begin{constructors}
\begin{constructor}{constructor(callback: (event: BrowserEvent) => void)}{Create a new TabListener.}
\begin{parameters}
\para{callback : (event : BrowserEvent) => void}{The function to invoke on created events.}
\end{parameters}
\end{constructor}
\end{constructors}
\begin{methods}
\begin{method}{public start() : void}{Start the listener. While the listener is running, it will create events and invoke the callback function on them.}
\end{method}
\begin{method}{public stop() : void}{Start the listener. While the listener is running, it will create events and invoke the callback function on them.}
\end{method}
\end{methods}
\end{class}

\begin{class}{OpenTabEvent}

\clsdiagram[width=0.5\textwidth]{resources/Classes/BrowserExtension/Listeners/OpenTabEvent.png}

\clsdcl{class OpenTabEvent extends BrowserEvent}

\clsdsp{An OpenTabEvent occurs when a new tab created. This does include the creation of a new browser window, which automatically creates at least one tab.}

\begin{constructors}
\begin{constructor}{constructor(tabID : number, windowID: number)}{Create a new OpenTabEvent.}
\begin{parameters}
\para{tabID : number}{The ID of the newly created tab.}
\para{windowID : number}{The ID of the window the tab with tabID belongs to.}
\end{parameters}
\end{constructor}
\end{constructors}
\end{class}

\begin{class}{CloseTabEvent}

\clsdiagram[width=0.5\textwidth]{resources/Classes/BrowserExtension/Listeners/CloseTabEvent.png}

\clsdcl{class CloseTabEvent extends BrowserEvent}

\clsdsp{A CloseTabEvent occurs when a tab is closed. This does include the closing of a browser window, where all attached tabs are automatically closed.}

\begin{constructors}
\begin{constructor}{constructor(tabID : number, windowID: number, url : string)}{Create a new CloseTabEvent.}
\begin{parameters}
\para{tabID : number}{The ID of the closed tab.}
\para{windowID : number}{The ID of the window the tab with tabID belongs to.}
\para{url : string}{The URL of the webpage which was opened in the now closed tab.}
\end{parameters}
\end{constructor}
\end{constructors}
\end{class}

\begin{class}{SwitchTabEvent}

\clsdiagram[width=0.8\textwidth]{resources/Classes/BrowserExtension/Listeners/SwitchTabEvent.png}

\clsdcl{class SwitchTabEvent extends BrowserEvent}

\clsdsp{A SwitchTabEvent occurs when the user switches between two existing tabs.}

\begin{attributes}
\attribute{public newTab : number}{The ID of the tab which gained focus.}
\end{attributes}
\begin{constructors}
\begin{constructor}{constructor(tabID : number, windowID: number, newTabID : number, url : string)}{Create a new SwitchTabEvent.}
\begin{parameters}
\para{tabID : number}{The ID of the previously focused tab.}
\para{windowID : number}{The ID of the window the tab with tabID belongs to.}
\para{newTabID : number}{The ID of the newly focused tab.}
\para{url : string}{The URL opened in the newly focused tab.}
\end{parameters}
\end{constructor}
\end{constructors}
\end{class}

\begin{class}{NavigationEvent}

\clsdiagram[width=0.5\textwidth]{resources/Classes/BrowserExtension/Listeners/NavigationEvent.png}

\clsdcl{class NavigationEvent extends BrowserEvent}

\clsdsp{A NavigationEvent occurs when the browser navigates to an URL. This also includes navigation/redirection without explicit user input.}

\begin{constructors}
\begin{constructor}{constructor(tabID : number, windowID: number, url : string)}{Create a new NavigationEvent.}
\begin{parameters}
\para{tabID : number}{The ID of the tab in which the navigation occured in.}
\para{windowID : number}{The ID of the window the tab with tabID belongs to.}
\para{url : string}{The URL that was navigated to.}
\end{parameters}
\end{constructor}
\end{constructors}
\end{class}

\begin{class}{TabEventFactory}

\clsdcl{class TabEventFactory}

\clsdsp{The TabEventFactory is responsible for creating BrowserEvent objects from the information a TabListener gathers. As different listeners get different input parameters, the createEvent method is overloaded several times to account for the possible combinations of parameters.}

\begin{methods}
\begin{method}{public createEvent(tabId : number, changeInfo : tabs.TabChangeInfo, tab : tabs.Tab) : BrowserEvent}{Create a new BrowserEvent from the input parameters.}
\begin{parameters}
\para{tabID : number}{The ID of the tab which provided the TabChangeInfo.}
\para{changeInfo : tabs.TabChangeInfo}{Create a new BrowserEvent from the input parameters.}
\para{tab : tabs.Tab}{}
\end{parameters}
\end{method}
\begin{method}{public createEvent(activeInfo : tabs.TabActiveInfo) : BrowserEvent}{Create a new BrowserEvent from the input parameters.}
\begin{parameters}
\para{activeInfo : tabs.TabActiveInfo}{The TabActiveInfo to create the event from.}
\end{parameters}
\end{method}
\begin{method}{public createEvent(tabId : number, removeInfo: tabs.TabRemoveInfo) : BrowserEvent}{}
\begin{parameters}
\para{tabId : number}{The ID of the tab which notified the caller.}
\para{removeInfo : tabs.TabRemoveInfo}{The TabRemoveInfo to create the event from.}
\end{parameters}
\end{method}
\begin{method}{public createEvent(tab : tabs.Tab) : BrowserEvent}{Create a new BrowserEvent from the input parameters.}
\begin{parameters}
\para{tab : tabs.Tab}{The tab to create the event from.}
\end{parameters}
\end{method}
\end{methods}
\end{class}
%%End Tab%%

%%DOM%%
\begin{class}{DOMListener}

\clsdiagram[width=1.0\textwidth]{resources/Classes/BrowserExtension/Listeners/DOMListener.png}

\clsdcl{class DOMListener implements IListener}

\clsdsp{The DOMListener is responsible for recording events which occur in the context of websites. To do so, the DOMListener needs to inject a ContentScript into opened websites and collect the events sent by the ContentScript through the browser message API.}

\begin{constructors}
\begin{constructor}{constructor(callback : (event: BrowserEvent) => void)}{Create a new DOMListener.}
\begin{parameters}
\para{callback : (event : BrowserEvent) => void}{The function to invoke on created events.}
\end{parameters}
\end{constructor}
\end{constructors}
\begin{methods}
\begin{method}{public start() : void}{Start the listener. While the listener is running, it will create events and invoke the callback function on them.}
\end{method}
\begin{method}{public stop() : void}{Start the listener. While the listener is running, it will create events and invoke the callback function on them.}
\end{method}
\end{methods}
\end{class}

\begin{class}{TextSelectionEvent}

\clsdiagram[width=0.5\textwidth]{resources/Classes/BrowserExtension/Listeners/TextSelectionEvent.png}

\clsdcl{class TextSelectionEvent extends BrowserEvent}

\clsdsp{A TextSelectionEvent occurs when the user selects text on a website. This event should not be triggered while the text selection is still pending, i. e. when the user has started selecting text, but not yet released the mouse button.}

\begin{attributes}
\attribute{public textSelection : string}{The selected text.}
\end{attributes}
\begin{constructors}
\begin{constructor}{constructor(tabID : number, windowID: number, textSelection : string, url : string)}{Create a new TextSelectionEvent.}
\begin{parameters}
\para{tabID : number}{The ID of the tab in which the event occured in.}
\para{windowID : number}{The ID of the window the tab with tabID belongs to.}
\para{url : string}{The URL of the open webpage.}
\para{textSelection : string}{The selected text.}
\end{parameters}
\end{constructor}
\end{constructors}
\end{class}

\begin{class}{TextInputEvent}

\clsdiagram[width=0.5\textwidth]{resources/Classes/BrowserExtension/Listeners/TextInputEvent.png}

\clsdcl{class TextInputEvent extends BrowserEvent}

\clsdsp{A TextInputEvent occurs when the user inputs text into a DOM element, e. g. a textbox.}

\begin{attributes}
\attribute{public text : string}{The input text.}
\attribute{public target : string}{String describing the target element which received the input, e. g. a textbox.}
\end{attributes}
\begin{constructors}
\begin{constructor}{constructor(tabID : number, windowID: number, text : string, target : string, url : string)}{Create a new TextInputEvent.}
\begin{parameters}
\para{tabID : number}{The ID of the tab in which the event occured in.}
\para{windowID : number}{The ID of the window the tab with tabID belongs to.}
\para{text : string}{The input text.}
\para{target}{String describing the target element which received the input, e. g. a textbox.}
\para{url : string}{The URL of the open webpage.}
\end{parameters}
\end{constructor}
\end{constructors}
\end{class}

\begin{class}{ButtonClickEvent}

\clsdiagram[width=0.5\textwidth]{resources/Classes/BrowserExtension/Listeners/ButtonClickEvent.png}

\clsdcl{class ButtonClickEvent extends BrowserEvent}

\clsdsp{A ButtonClickEvent occurs when the user clicks on a button element on a website.}

\begin{attributes}
\attribute{public buttonTitle : string}{The title/label of the button which was clicked.}
\attribute{public buttonHref : string}{Optional. The URL which the clicked button links to.}
\end{attributes}
\begin{constructors}
\begin{constructor}{constructor(tabID : number, windowID: number, buttonTitle : string, url : string, buttonHref? : string)}{Create a new ButtonClickEvent.}
\begin{parameters}
\para{tabID : number}{The ID of the tab in which the event occured in.}
\para{windowID : number}{The ID of the window the tab with tabID belongs to.}
\para{buttonTitle : string}{The title/label of the button which was clicked.}
\para{url : string}{The URL of the open webpage.}
\para{buttonHref : string}{Optional. The URL which the clicked button links to.}
\end{parameters}
\end{constructor}
\end{constructors}
\end{class}

\begin{class}{HoverEvent}

\clsdcl{class HoverEvent extends BrowserEvent}

\clsdiagram[width=0.5\textwidth]{resources/Classes/BrowserExtension/Listeners/HoverEvent.png}

\clsdsp{A HoverEvent occurs when the user hovers the mouse over an element on a website.}

\begin{attributes}
\attribute{public target : string}{String describing the element which was hovered.}
\end{attributes}
\begin{constructors}
\begin{constructor}{constructor(tabID : number, windowID: number, target : string, url : string)}{Create a new HoverEvent.}
\begin{parameters}
\para{tabID : number}{The ID of the tab in which the event occured in.}
\para{windowID : number}{The ID of the window the tab with tabID belongs to.}
\para{target : string}{String describing the element which was hovered.}
\para{url : string}{The URL of the open webpage.}
\end{parameters}
\end{constructor}
\end{constructors}
\end{class}
%%End DOM%%

%%Download%%
\begin{class}{DownloadListener}

\clsdiagram[width=0.5\textwidth]{resources/Classes/BrowserExtension/Listeners/DownloadListener.png}

\clsdcl{class DownloadListener implements IListener}

\clsdsp{The DownloadListener is responsible for recording events connected to the download browser-API.}

\begin{constructors}
\begin{constructor}{constructor(callback: (event : BrowserEvent) => void)}{Create a new DownloadListener.}
\begin{parameters}
\para{callback : (event : BrowserEvent) => void}{The function to invoke on created events.}
\end{parameters}
\end{constructor}
\end{constructors}
\begin{methods}
\begin{method}{public start() : void}{Start the listener. While the listener is running, it will create events and invoke the callback function on them.}
\end{method}
\begin{method}{public stop() : void}{Start the listener. While the listener is running, it will create events and invoke the callback function on them.}
\end{method}
\end{methods}
\end{class}

\begin{class}{DownloadEvent}

\clsdiagram[width=0.5\textwidth]{resources/Classes/BrowserExtension/Listeners/DownloadEvent.png}

\clsdcl{class DownloadEvent extends BrowserEvent}

\clsdsp{A DownloadEvent occurs when the user starts downloading a file.}

\begin{attributes}
\attribute{public mimeType : string}{The MIME type of the downloaded file.}
\attribute{public fileURL : string}{The URL of the downloaded file.}
\end{attributes}
\begin{constructors}
\begin{constructor}{constructor(tabID : number, windowID : number, mimeType : string, fileURL : string, url : string)}{Create a new DownloadEvent.}
\begin{parameters}
\para{tabID : number}{The ID of the tab in which the event occured in.}
\para{windowID : number}{The ID of the window the tab with tabID belongs to.}
\para{mimeType : string}{The MIME type of the downloaded file.}
\para{fileURL : string}{The URL of the downloaded file.}
\para{url : string}{The URL of the opened webpage.}
\end{parameters}
\end{constructor}
\end{constructors}
\end{class}

\begin{class}{DownloadEventFactory}

\clsdcl{class DownloadEventFactory}

\clsdsp{The TabEventFactory is responsible for creating DownloadEvent objects from the information a DownloadListener gathers.}

\begin{methods}
\begin{method}{createEvent(downloadItem: downloads.DownloadItem) : DownloadEvent}{Create a new DownloadEvent from a DownLoadItem.}
\begin{parameters}
\para{downloadItem : downloads.DownloadItem}{The DownLoadItem created by the browser when a download was started.}
\end{parameters}
\end{method}
\end{methods}
\end{class}
%%End Download%%