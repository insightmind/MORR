\subsection*{MORR.Shared.Events}

\begin{absclass}{Event}
    \clsdiagram[width=0.4\textwidth]{resources/Classes/Common/Shared/Events/Event.png}

    \clsdcl{public abstract class Event}

    \clsdsp{Provides attributes shared between every user interaction event.}

    \begin{attributes}
        \attribute{public DateTime Timestamp \{ get; set; \}}{The timestamp at which the event occured.}
        \attribute{public Guid IssuingModule \{ get; set; \}}{The identifier of the module that issued the event.}
    \end{attributes}
\end{absclass}

\subsection*{MORR.Shared.Events.Queue}

\begin{absclass}{DefaultEventQueue<T>}
	\clsdiagram[width=0.6\textwidth]{resources/Classes/Common/Shared/Events/DefaultEventQueue.png}
	
	\clsdcl{public abstract class DefaultEventQueue<T> : SupportDeserializationEventQueue<T> where T : Event}
	
	\clsdsp{Provides an event queue for the most common scenarios that supports deserialization.}
\end{absclass}

\begin{absclass}{DefaultEncodableEventQueue<T>}
	\clsdiagram[width=0.6\textwidth]{resources/Classes/Common/Shared/Events/DefaultEncodableEventQueue.png}
	
	\clsdcl{public abstract class DefaultEncodableEventQueue<T> : BaseEventQueue<T>, IEncodableEventQueue<T> where T : Event}
	
	\clsdsp{Provides an event queue for events intended for encoding.}
\end{absclass}

\begin{absclass}{DefaultDecodableEventQueue<T>}
	\clsdiagram[width=0.6\textwidth]{resources/Classes/Common/Shared/Events/DefaultDecodableEventQueue.png}
	
	\clsdcl{public abstract class DefaultDecodableEventQueue<T> : BaseEventQueue<T>, IDecodableEventQueue<T> where T : Event}
	
	\clsdsp{Provides an event queue for events intended for decoding.}
\end{absclass}

\begin{absclass}{NonDeserializableEventQueue<T>}
	\clsdiagram[width=0.6\textwidth]{resources/Classes/Common/Shared/Events/NonDeserializableEventQueue.png}
	
	\clsdcl{public abstract class NonDeserializableEventQueue<T> : BaseEventQueue<T>, IReadOnlyEventQueue<T> where T : Event}
	
	\clsdsp{Provides an event queue for events that do not support deserialization.}
\end{absclass}
	
\begin{absclass}{BaseEventQueue<TEvent>}
	\clsdiagram[width=0.6\textwidth]{resources/Classes/Common/Shared/Events/BaseEventQueue.png}
	
	\clsdcl{public abstract class BaseEventQueue<TEvent> where TEvent : Event}
	
	\clsdsp{Provides the basic structure of an EventQueue.}
	
	\begin{attributes}
		\attribute{public bool IsClosed}{Describes whether the queue is currently enabled to queue new events or not.}
	\end{attributes}

    \begin{methods}
        \begin{method}{public IAsyncEnumerable<TEvent> GetEvents()}{Asynchronously gets all events as concrete type TEvent.}
            \return{IAsyncEnumerable<TEvent>}{A stream of events of type T.}
        \end{method}
        \begin{method}{public void Enqueue(TEvent @event)}{Asynchronously enqueues a new event.}
        	\begin{parameters}
        		\para{TEvent @event}{The event to enqueue}
        	\end{parameters}
        \end{method}
        \begin{method}{public void Open()}{Opens the EventQueue so new events can be queued.}
        \end{method}
        \begin{method}{public void Close()}{Closes the EventQueue so no new event can be queued.}
        \end{method}
    \end{methods}
\end{absclass}

\begin{interface}{IDecodableEventQueue<out T>}
	\clsdiagram[width=0.7\textwidth]{resources/Classes/Common/Shared/Events/IDecodableEventQueue.png}

	\clsdcl{public interface IDecodableEventQueue<out T> where T : Event}
	
	\clsdsp{Provides a read-only queue for event types intended for decoding.}
	
	\begin{attributes}
		\attribute{public bool IsClosed{get;}}{Describes whether the queue is currently enabled to queue new events or not.}
	\end{attributes}
	
\end{interface}

\begin{interface}{IEncodableEventQueue<out T>}
	\clsdiagram[width=0.7\textwidth]{resources/Classes/Common/Shared/Events/IEncodableEventQueue.png}

	\clsdcl{public interface IEncodableEventQueue<out T> where T : Event}
	
	\clsdsp{Provides a read-only queue for event types intended for encoding.}
	
	\begin{attributes}
		\attribute{public bool IsClosed{get;}}{Describes whether the queue is currently enabled to queue new events or not.}
	\end{attributes}
	
\end{interface}

\begin{interface}{IReadOnlyEventQueue<out T>}
	\clsdiagram[width=0.7\textwidth]{resources/Classes/Common/Shared/Events/IReadOnlyEventQueue.png}

	\clsdcl{public interface IReadOnlyEventQueue<out T> where T : Event}
	
	\clsdsp{Provides a read-only access to a queue for events as concrete type T.}
	
	\begin{attributes}
		\attribute{public bool IsClosed{get;}}{Describes whether the queue is currently enabled to queue new events or not.}
	\end{attributes}
	
\end{interface}

\begin{interface}{ISupportDeserializationEventQueue<out T>}
	\clsdiagram[width=0.7\textwidth]{resources/Classes/Common/Shared/Events/ISupportDeserializationEventQueue.png}

	\clsdcl{public interface ISupportDeserializationEventQueue<out T> where T : Event}
	
	\clsdsp{Provides a single-writer-multiple-reader queue for event types with support for deserialization.}
	
	\begin{methods}
		\begin{method}{public void Open()}{Opens the EventQueue so new events can be queued.}
        \end{method}
        \begin{method}{public void Close()}{Closes the EventQueue so no new event can be queued.}
        \end{method}
    \end{methods}

\end{interface}

\begin{absclass}{SupportDeserializationEventQueue<T>}
	\clsdiagram[width=0.7\textwidth]{resources/Classes/Common/Shared/Events/SupportDeserializationEventQueue.png}
	
	\clsdcl{public abstract class SupportDeserializationEventQueue<T> : BaseEventQueue<T>, IReadOnlyEventQueue<T>, ISupportDeserializationEventQueue<T> where T : Event}
	
	\clsdsp{Provides a single-writer-multiple-reader event queue for event types which may be deserialized for processing.}
	
	\begin{methods}
		\begin{method}{public void Enqueue(object @event)}{Asynchronously enqueues a new event of type T.}
            \begin{parameters}
                \para{T @event}{The event to enqueue.}
            \end{parameters}
            \begin{exceptions}
            	\excp{ArgumentException}{Exception is thrown when event type is not T}
        \end{method}
    \end{methods}
\end{absclass}
	
\subsection*{MORR.Shared.Events.Queue.Strategy}

\begin{class}
	\clsdiagram[width=0.4\textwidth]{resources/Classes/Common/Shared/Events/Strategy/ChannelConsumingException.png}
	
	\clsdcl{public class ChannelConsumingException : Exception}	
	
	\clsdsp{A simple Exception that encapsulates errors occuring in ChannelStrategies.}
	
\end{class}
	
\begin{interface}{IEventQueueStorageStrategy}
    \clsdiagram[width=0.4\textwidth]{resources/Classes/Common/Shared/Events/Strategy/IEventQueueStorageStrategy.png}

    \clsdcl{public interface IEventQueueStorageStrategy<T> where T : Event}

    \clsdsp{Provides the backing store of an EventQueue with a specific storage strategy.}

\end{interface}

\subsubsection*{MORR.Shared.Event.Queue.Strategy.MultiConsumer}

\begin{class}{BoundedMultiConsumerChannelStrategy<TEvent>}
	\clsdiagram[width=0.4\textwidth]{resources/Classes/Common/Shared/Events/Queue/Strategy/MultiConsumer/BoundedMultiConsumerChannelStrategy.png}
	
	\clsdcl{public class BoundedMultiConsumerChannelStrategy<TEvent> : MultiConsumerChannelStrategy<TEvent> where TEvent : Event}
	
	\clsdsp{A BoundedMultiConsumerChannelStrategies is a Distributive-FIFO Queue which allows multiple producers and multiple consumers. However every event is propagated to each consumer once. This queue is bounded by a buffer capacity. To be more detailed each channel is bound by the bufferCapacity. Is the capacity reached for a specific channel it will drop the oldest event and write the new event.}
	
	\begin{methods}	
		\begin{method}{protected override Channel<TEvent> CreateOfferingChannel()}
			\return{Channel<TEvent>}{}
		\end{method}
		\begin{method}{protected override Channel<TEvent> CreateRecievingChannel()}
			\return{Channel<TEvent>}{}
		\end{method}
	\end{methods}
\end{class}

\begin{absclass}{MultiConsumerChannelStrategy<TEvent>} 
	\clsdiagram[width=0.4\textwidth]{resources/Classes/Common/Shared/Events/Queue/Strategy/MultiConsumer/MultiConsumerChannelStrategy.png}
	
	\clsdcl{public abstract class MultiConsumerChannelStrategy<TEvent> : IEventQueueStorageStrategy<TEvent> where TEvent : Event}
	
	\clsdsp{A MultiConsumerChannelStrategies is a Distributive-FIFO Queue which allows multiple producers and multiple consumers. However every event is propagated to each consumer once.}
	
	\begin{attributes}
		\attribute{public bool IsClosed { get; private set; }}{Describes whether the queue is currently enabled to queue new events or not}
	\end{attributes}
	
	\begin{methods}
		\begin{method}{protected void StartReceiving(uint? maxChannelConsumers)}{Sets maximal number of consumers}
			\begin{parameters}
				\para{uint? maxChannelConsumers}{Maximal number of consumers}
			\end{parameters}
		\end{method}
		\begin{method}{public IAsyncEnumerable<TEvent> GetEvents(CancellationToken token = default)}{Asynchronously gets all events as concrete type T}
			\begin{parameters}
				\para{CancellationToken token = default}
			\end{parameters}
			\begin{exceptions}
				\excp{ChannelConsumingException}{Exception is thrown when max number of consumers is reached}
			\end{exceptions}
			\return{IAsyncEnumerable<TEvent>}{A stream of events type T}
		\end{method}
		\begin{method}{public async void Enqueue(TEvent @event)}{Asynchronously enqueues a new event}
			\begin{parameters}
				\para{TEvent @event}{The event to enqueue}
			\end{parameters}
		\end{method}
		\begin{method}{public void Open()}{Notifies the event queue that it is free to accept input.}
        \end{method}
        \begin{method}{public void Close()}{Notifies the event queue that no more events will be queued.}
        \end{method}
        \begin{method}{protected abstract Channel<TEvent> CreateOfferingChannel();}
        \end{method}
        \begin{method}{protected abstract Channel<TEvent> CreateReceivingChannel();}
        \end{method}
	\end{methods}
\end{absclass}

\begin{class}{UnboundedMultiConsumerChannelStrategy<TEvent>}
	\clsdiagram[width=0.4\textwidth]{resources/Classes/Common/Shared/Events/Queue/Strategy/MultiConsumer/UnboundedMultiConsumerChannelStrategy.png}
	
	\clsdcl{public class UnboundedMultiConsumerChannelStrategy<TEvent> : MultiConsumerChannelStrategy<TEvent> where TEvent : Event}
	
	\clsdsp{An UnboundedMultiConsumerChannelStrategy is a Distributive-FIFO Queue which allows multiple producers and multiple consumers. However every event is propagated to each consumer once. This queue is unbound, which means that an event is never dismissed. Make sure to only use this queue if you are sure it is bound by a maximum number of events.}
	
	\begin{methods}	
		\begin{method}{protected override Channel<TEvent> CreateOfferingChannel()}
			\return{Channel<TEvent>}{}
		\end{method}
		\begin{method}{protected override Channel<TEvent> CreateRecievingChannel()}
			\return{Channel<TEvent>}{}
		\end{method}
	\end{methods}
\end{class}

\subsubsection*{SingleConsumer}

\begin{class}{BoundedSingleConsumerChannelStrategy<TEvent>}
	\clsdiagram[width=0.4\textwidth]{resources/Classes/Common/Shared/Events/Queue/Strategy/SingleConsumer/BoundedSingleConsumerChannelStrategy.png}
	
	\clsdcl{public class BoundedSingleConsumerChannelStrategy<TEvent> : SingleConsumerChannelStrategy<TEvent> where TEvent : Event}
	
	\clsdsp{A BoundedSingleConsumerChannelStrategy is a Distributive-FIFO Queue which allows multiple producers and a single consumer. This queue is performance optimized for a single consumer and should be preferred in this case. This queue is bounded by a buffer capacity. If capacity is reached it will dismiss the oldest event and queue the new event.}
	
	\begin{methods}	
		\begin{method}{protected override Channel<TEvent> CreateChannel()}
			\return{Channel<TEvent>}{}
		\end{method}
	\end{methods}
\end{class}

\begin{absclass}{SingleConsumerChannelStrategy<TEvent>} 
	\clsdiagram[width=0.4\textwidth]{resources/Classes/Common/Shared/Events/Queue/Strategy/SingleConsumer/SingleConsumerChannelStrategy.png}
	
	\clsdcl{public abstract class SingleConsumerChannelStrategy<TEvent> : IEventQueueStorageStrategy<TEvent> where TEvent : Event}
	
	\clsdsp{A SingleConsumerChannelStrategies is a Distributive-FIFO Queue which allows multiple producers and singer consumers. This queue is performance optimized for a single consumer and should be preferred in this case.}
	
	\begin{attributes}
		\attribute{public bool IsClosed { get; private set; }}{Describes whether the queue is currently enabled to queue new events or not}
	\end{attributes}
	
	\begin{methods}
		\begin{method}{public IAsyncEnumerable<TEvent> GetEvents(CancellationToken token = default)}{Asynchronously gets all events as concrete type T}
			\begin{parameters}
				\para{CancellationToken token = default}
			\end{parameters}
			\begin{exceptions}
				\excp{ChannelConsumingException}{Exception is thrown when max number of consumers is reached}
			\end{exceptions}
			\return{IAsyncEnumerable<TEvent>}{A stream of events type T}
		\end{method}
		\begin{method}{public async void Enqueue(TEvent @event)}{Asynchronously enqueues a new event}
			\begin{parameters}
				\para{TEvent @event}{The event to enqueue}
			\end{parameters}
		\end{method}
		\begin{method}{public void Open()}{Notifies the event queue that it is free to accept input.}
        \end{method}
        \begin{method}{public void Close()}{Notifies the event queue that no more events will be queued.}
        \end{method}
        \begin{method}{protected abstract Channel<TEvent> CreateChannel();}
        \end{method}
	\end{methods}
\end{absclass}

\begin{class}{UnboundedSingleConsumerChannelStrategy<TEvent>}
	\clsdiagram[width=0.4\textwidth]{resources/Classes/Common/Shared/Events/Queue/Strategy/SingleConsumer/UnboundedSingleConsumerChannelStrategy.png}
	
	\clsdcl{public class UnboundedSingleConsumerChannelStrategy<TEvent> : SingleConsumerChannelStrategy<TEvent> where TEvent : Event}
	
	\clsdsp{A UnboundedSingleConsumerChannelStrategy is a Distributive-FIFO Queue which allows multiple producers and a single consumer. This queue is performance optimized for a single consumer and should be preferred in this case.  This queue is unbounded and will therefore don't dismiss any events.}
	
	\begin{methods}	
		\begin{method}{protected override Channel<TEvent> CreateChannel()}
			\return{Channel<TEvent>}{}
		\end{method}
	\end{methods}
\end{class}





