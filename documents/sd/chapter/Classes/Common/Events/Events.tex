\subsection*{MORR.Shared.Events}

\begin{absclass}{Event}
    \clsdiagram[width=0.4\textwidth]{resources/Classes/Common/Shared/Events/Event.png}

    \clsdcl{public abstract class Event}

    \clsdsp{Provides attributes shared between every user interaction event.}

    \begin{attributes}
        \attribute{public DateTime Timestamp \{ get; set; \}}{The timestamp at which the event occured.}
        \attribute{public Guid IssuingModule \{ get; set; \}}{The identifier of the module that issued the event.}
    \end{attributes}

    \begin{methods}
        \begin{method}{public string Serialize()}{Serializes the event to a string.}
            \return{string}{The serialized event.}
        \end{method}
        \begin{method}{public void Deserialize(string serialized)}{Deserializes the event from a string.}
            \begin{parameters}
                \para{string serialized}{A string containing the serialized event.}
            \end{parameters}
        \end{method}
    \end{methods}
\end{absclass}

\subsection*{MORR.Shared.Events.Queue}

\begin{interface}{IReadOnlyEventQueue<T>}
    \clsdiagram[width=0.6\textwidth]{resources/Classes/Common/Shared/Events/IReadOnlyEventQueue.png}

    \clsdcl{public interface IReadOnlyEventQueue<out T> where T : Event}

    \clsdsp{Provides read-only access to a queue of events as concrete type T.}

    \begin{methods}
        \begin{method}{IAsyncEnumerable<T> GetEvents()}{Asynchronously gets all event as their concrete type T.}
            \return{IAsyncEnumerable<T>}{A stream of events of type T.}
        \end{method}
    \end{methods}
\end{interface}

\begin{absclass}{EventQueue<T>}
    \clsdiagram[width=0.4\textwidth]{resources/Classes/Common/Shared/Events/EventQueue.png}

    \clsdcl{public abstract class EventQueue<T> : IReadOnlyEventQueue<T> where T : Event}

    \clsdsp{Provides a single-writer-multiple-reader queue for Event types}

    \begin{attributes}
        \attribute{private IEventQueueStorageStrategy<T> storageStrategy}{The storage strategy to use.}
    \end{attributes}

    \begin{methods}
        \begin{method}{public abstract IAsyncEnumerable<T> GetEvents()}{Asynchronously gets all events as concrete type T.}
            \return{IAsyncEnumerable<T>}{A stream of events of type T.}
        \end{method}
        \begin{method}{protected abstract void Enqueue(T @event)}{Asynchronously enqueues a new event of type T.}
            \begin{parameters}
                \para{T @event}{The event to enqueue.}
            \end{parameters}
        \end{method}
    \end{methods}
\end{absclass}

\begin{interface}{IEventQueueStorageStrategy}
    \clsdiagram[width=0.4\textwidth]{resources/Classes/Common/Shared/Events/IEventQueueStorageStrategy.png}

    \clsdcl{public interface IEventQueueStorageStrategy<T> where T : Event}

    \clsdsp{Provides the backing store of an EventQueue with a specific storage strategy.}

    \begin{methods}
        \begin{method}{IAsyncEnumerable<T> GetEventsInternal()}{Asynchronously gets all events as concrete type T from this storage strategy.}
            \return{IAsyncEnumerable<T>}{A stream of events of type T.}
        \end{method}
        \begin{method}{void Enqueue(T @event)}{Asynchronously enqueues a new event of type T with this storage strategy.}
            \begin{parameters}
                \para{T @event}{The event to enqueue.}
            \end{parameters}
        \end{method}
    \end{methods}
\end{interface}

\begin{class}{RingBufferStorageStrategy}
    \clsdiagram[width=0.4\textwidth]{resources/Classes/Common/Shared/Events/RingBufferStorageStrategy.png}

    \clsdcl{public class RingBufferStorageStrategy<T> : IEventQueueStorageStrategy<T> where T : Event}

    \clsdsp{Provides event storage using a ring buffer strategy. That means if the buffer is full, events will be overwritten even if they have not been read by all consumers yet.}

    \begin{attributes}
        \attribute{public uint Capacity \{ get; set; \}}{The capacity of the ring buffer.}
    \end{attributes}
\end{class}

\begin{class}{RefCountedListStorageStrategy}
    \clsdiagram[width=0.4\textwidth]{resources/Classes/Common/Shared/Events/RefCountedListStorageStrategy.png}

    \clsdcl{public class RefCountedListStorageStrategy<T> : IEventQueueStorageStrategy<T> where T : Event}

    \clsdsp{Provides event storage using a reference counted list strategy. That means that events will be deleted from the list they are stored in if all consumers have read them.}
\end{class}

\begin{class}{KeepAllStorageStrategy}
    \clsdiagram[width=0.4\textwidth]{resources/Classes/Common/Shared/Events/KeepAllStorageStrategy.png}

    \clsdcl{public class KeepAllStorageStrategy<T> : IEventQueueStorageStrategy<T> where T : Event}

    \clsdsp{Provides event storage using a keep all strategy. That means that events that have been enqueued will not be removed from the queue.}
\end{class}