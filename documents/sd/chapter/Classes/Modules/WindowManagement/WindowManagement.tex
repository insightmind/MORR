\subsection{WindowManagement}

\begin{class}{WindowManagementModule} 
    \clsdiagram[scale = 0.7]{resources/Classes/Modules/WindowManagement/WindowManagementModule.png}

    \clsdcl{public class WindowManagementModule : ICollectingModule}

    \clsdsp{The WindowManagementModule is responsible for recording all window related user interactions.}
\end{class}

\subsection*{Events}

\begin{absclass}{WindowEvent} 
    \clsdiagram{resources/Classes/Modules/WindowManagement/Events/WindowEvent.png}

    \clsdcl{public abstract class WindowEvent: Event}

    \clsdsp{A window management event which all specific WindowEvents inherit from.}

    \begin{attributes}
        \attribute{public string Title}{The title of the interacted window.}
        \attribute{public string ProcessName}{The name of the process associated with the window}
    \end{attributes}
\end{absclass}

\begin{class}{WindowFocusEvent} 
    \clsdiagram{resources/Classes/Modules/WindowManagement/Events/WindowFocusEvent.png}

    \clsdcl{public class WindowFocusEvent : WindowEvent}

    \clsdsp{A window focus user interaction.}
\end{class}

\begin{class}{WindowMovementEvent} 
    \clsdiagram{resources/Classes/Modules/WindowManagement/Events/WindowMovementEvent.png}

    \clsdcl{public class WindowMovementEvent : WindowEvent}

    \clsdsp{ A window movement user interaction.}

    \begin{attributes}
        \attribute{public System.Numerics.Vector2 OldLocation}{The old location of the window.}
        \attribute{public System.Numerics.Vector2 NewLocation}{The new location of the window.}
    \end{attributes}
\end{class}

\begin{class}{WindowResizingEvent} 
    \clsdiagram{resources/Classes/Modules/WindowManagement/Events/WindowResizingEvent.png}

    \clsdcl{public class WindowResizingEvent : WindowEvent}

    \clsdsp{A window resizing user interaction.}

    \begin{attributes}
        \attribute{public System.Drawing.Size OldSize}{The old size of the window.}
        \attribute{public System.Drawing.Size NewSize}{The new size of the window.}
    \end{attributes}
\end{class}

\begin{class}{WindowStateChangedEvent} 
    \clsdiagram{resources/Classes/Modules/WindowManagement/Events/WindowStateChangedEvent.png}

    \clsdcl{public class WindowStateChangedEvent : WindowEvent}

    \clsdsp{An user interaction that changes the state of a window.}

    \begin{attributes}
        \attribute{public System.Windows.WindowState State}{The new state of the window.}
    \end{attributes}
\end{class}

\subsection*{Producers}

\begin{class}{WindowFocusEventProducer} 
	\clsdiagram[scale = 0.7]{resources/Classes/Modules/WindowManagement/Producers/WindowFocusEventProducer.png}
    
    \clsdcl{public class WindowFocusEventProducer : EventQueue<WindowFocusEvent>}

    \clsdsp{A single-writer-multiple-reader queue for WindowFocusEvent}

    \begin{methods}
        \begin{method}{public override IAsyncEnumerable<WindowFocusEvent> GetEvents()}{Asynchronously gets all window focus events}
          \return{IAsyncEnumerable<WindowFocusEvent>}{A stream of WindowFocusEvent}
        \end{method}
        \begin{method}{protected override void Enqueue(WindowFocusEvent event)}{Asynchronously enqueues a new window focus event}
            \begin{parameters}
                \para{WindowFocusEvent event}{WindowFocusEvent to enqueue}
            \end{parameters}
        \end{method}
    \end{methods}
\end{class}

\begin{class}{WindowMovementEventProducer} 
	\clsdiagram[scale = 0.7]{resources/Classes/Modules/WindowManagement/Producers/WindowMovementEventProducer.png}
    
    \clsdcl{public class WindowMovementEventProducer : EventQueue<WindowMovementEvent>}

    \clsdsp{A single-writer-multiple-reader queue for WindowMovementEvent}

    \begin{methods}
        \begin{method}{public override IAsyncEnumerable<WindowMovementEvent> GetEvents()}{Asynchronously gets all window movement events}
          \return{IAsyncEnumerable<WindowMovementEvent>}{A stream of WindowMovementEvent}
        \end{method}
        \begin{method}{protected override void Enqueue(WindowMovementEvent event)}{Asynchronously enqueues a new window movement event}
            \begin{parameters}
                \para{WindowMovementEvent event}{WindowMovementEvent to enqueue}
            \end{parameters}
        \end{method}
    \end{methods}
\end{class}

\begin{class}{WindowResizingEventProducer} 
	\clsdiagram[scale = 0.7]{resources/Classes/Modules/WindowManagement/Producers/WindowResizingEventProducer.png}
    
    \clsdcl{public class WindowResizingEventProducer : EventQueue<WindowResizingEvent>}

    \clsdsp{A single-writer-multiple-reader queue for WindowResizingEvent}

    \begin{methods}
        \begin{method}{public override IAsyncEnumerable<WindowResizingEvent> GetEvents()}{Asynchronously gets all window resizing events}
          \return{IAsyncEnumerable<WindowResizingEvent>}{A stream of WindowResizingEvent}
        \end{method}
        \begin{method}{protected override void Enqueue(WindowResizingEvent event)}{Asynchronously enqueues a new window resizing event}
            \begin{parameters}
                \para{WindowResizingEvent event}{WindowResizingEvent to enqueue}
            \end{parameters}
        \end{method}
    \end{methods}
\end{class}

\begin{class}{WindowStateChangedEventProducer} 
	\clsdiagram[scale = 0.7]{resources/Classes/Modules/WindowManagement/Producers/WindowStateChangedEventProducer.png}
    
    \clsdcl{public class WindowStateChangedEventProducer : EventQueue<WindowStateChangedEvent>}

    \clsdsp{A single-writer-multiple-reader queue for WindowRStateChangedEvent}

    \begin{methods}
        \begin{method}{public override IAsyncEnumerable<WindowStateChangedEvent> GetEvents()}{Asynchronously gets all user interaction with changing the state of the window}
          \return{IAsyncEnumerable<WindowStateChangedEvent>}{A stream of WindowStateChangedEvent}
        \end{method}
        \begin{method}{protected override void Enqueue(WindowStateChangedEvent event)}{Asynchronously enqueues a new window state changed event}
            \begin{parameters}
                \para{WindowStateChangedEvent event}{WindowStateChangedEvent to enqueue}
            \end{parameters}
        \end{method}
    \end{methods}
\end{class}