\subsection{Keyboard}
\begin{class}{KeyboardModule} 
    \clsdiagram[scale = 0.7]{resources/Classes/Modules/Keyboard/KeyboardModule.png}

    \clsdcl{public class KeyboardModule : IModule}

    \clsdsp{The KeyboardModule is responsible for recording all keyboard related user interactions.}
\end{class}

\subsection*{Events}

\begin{absclass}{KeyboardEvent} 
    \clsdiagram{resources/Classes/Modules/Keyboard/Events/KeyboardEvent.png}

    \clsdcl{public abstract class KeyboardEvent : Event}

    \clsdsp{A generic keyboard event which all specific KeyboardEvents inherit from.}
\end{absclass}

\begin{class}{KeyboardInteractEvent} 
    \clsdiagram{resources/Classes/Modules/Keyboard/Events/KeyBoardInteractEvent.png}

    \clsdcl{public class KeyboardInteractEvent : KeyboardEvent}

    \clsdsp{A keyboard user interaction.}

    \begin{attributes}
        \attribute{public System.Windows.Input.Key PressedKey\_System\_Windows\_Input\_Key}{The key that was pressed.}
        \attribute{public string PressedKeyName}{The name of the key on the keyboard that is pressed.}
        \attribute{public System.Windows.Input.ModifierKeys ModifierKeys\_System\_Windows\_Input\_ModifierKeys}{The modifier keys to the key pressed.}
        \attribute{public string ModifierKeysName}{The name of the key on the keyboard that is pressed.}
        \attribute{public char MappedCharacter\_Unicode}{The actual user input according to Input locale.}
    \end{attributes}
\end{class}

\subsection*{Producers}

\begin{class}{KeyboardInteractEventProducer} 
	\clsdiagram[scale = 0.9]{resources/Classes/Modules/Keyboard/Producers/KeyboardInteractEventProducer.png}
    
    \clsdcl{public class KeyboardInteractEventProducer:\\\ DefaultEventQueue<KeyboardInteractEvent>}

    \clsdsp{Provides single-writer-multiple-reader queue for KeyboardInteractEvent}

    \begin{methods}
        \begin{method}{public void StartCapture(INativeKeyboard nativeK)}{Starts capturing keyboard interaction events}
        	\begin{parameters}
        		\para{INativeKeyboard nativeK}{Required for capturing KeyboardInteractEvent.}
        	\end{parameters}
        \end{method}
        \begin{method}{public void StopCapture()}{Stops capturing keyboard interaction events}
        \end{method}
    \end{methods}
\end{class}