\subsection{Mouse}

\begin{class}{MouseModule} 
    \clsdiagram[scale = 0.7]{resources/Classes/Modules/Mouse/MouseModule.png}

    \clsdcl{public class MouseModule : IModule}

    \clsdsp{The MouseModule is responsible for recording all mouse related user interactions.}
\end{class}

\subsection*{Configuration}

\begin{class}{MouseModuleConfiguration}
	\clsdiagram[scale = 0.7]{resources/Classes/Modules/Mouse/MouseModuleConfiguration.png}
	
	\clsdcl{public class MouseModuleConfiguration: IConfiguration}
	
	\clsdsp{Configuration for Mouse module}
	
	\begin{attributes}
		\attribute{public uint SamplingRateInHz}{The sampling rate of the mouse position capture, in Hz.}
		\attribute{public int Threshold}{ The minimal distance(computed with screen coordinates) a mouse move must reach in a period to be recorded.}
	\end{attributes}
\end{class}

\subsection*{Events}

\begin{absclass}{MouseEvent} 
    \clsdiagram{resources/Classes/Modules/Mouse/Events/MouseEvent.png}

    \clsdcl{public abstract class MouseEvent: Event}

    \clsdsp{A generic mouse event which all specific MouseEvents inherit from.}
    
    \begin{attributes}
    	\attribute{public System.Windows.Point MousePosition}{The current position of the mouse in screen coordinates}
    \end{attributes}
\end{absclass}

\begin{class}{MouseClickEvent} 
    \clsdiagram{resources/Classes/Modules/Mouse/Events/MouseClickEvent.png}

    \clsdcl{public class MouseClickEvent : MouseEvent}

    \clsdsp{ A mouse click user interaction.}

    \begin{attributes}
        \attribute{public System.Windows.Input.MouseAction MouseAction}{Specifies constants that define actions performed by the mouse.}
        \attribute{public string HWnd}{The handle of the window in which the mouse click occurred.}
    \end{attributes}
\end{class}

\begin{class}{MouseMoveEvent} 
    \clsdiagram{resources/Classes/Modules/Mouse/Events/MouseMoveEvent.png}

    \clsdcl{public class MouseMoveEvent : MouseEvent}

    \clsdsp{A mouse move user interaction.}
\end{class}

\begin{class}{MouseScrollEvent} 
    \clsdiagram{resources/Classes/Modules/Mouse/Events/MouseScrollEvent.png}

    \clsdcl{public class MouseScrollEvent : MouseEvent}

    \clsdsp{A mouse scroll user interaction.}

    \begin{attributes}
        \attribute{public short ScrollAmount}{The amount of the wheel being scrolled.}
        \attribute{public string HWnd}{The handle of the window in which the mouse scroll occurred.}
    \end{attributes}
\end{class}

\subsection*{Producers}

\begin{class}{MouseClickEventProducer} 
	\clsdiagram[scale = 0.7]{resources/Classes/Modules/Mouse/Producers/MouseClickEventProducer.png}
    
    \clsdcl{public class MouseClickEventProducer : DefaultEventQueue<MouseClickEvent>}

    \clsdsp{Provides single-writer-multiple-reader queue for MouseClickEvent}

    \begin{methods}
        \begin{method}{public void StartCapture()}{Starts capturing mouse click events}
        \end{method}
        \begin{method}{public void StopCapture()}{Stop capturing mouse click events}
        \end{method}
    \end{methods}
\end{class}

\begin{class}{MouseMoveEventProducer} 
	\clsdiagram[scale = 0.7]{resources/Classes/Modules/Mouse/Producers/MouseMoveEventProducer.png}
    
    \clsdcl{public class MouseMoveEventProducer : DefaultEventQueue<MouseMoveEvent>}

    \clsdsp{A single-writer-multiple-reader queue for MouseMoveEvent}

    \begin{methods}
        \begin{method}{public void StartCapture(INativeMouse nativeM)}{Starts capturing mouse move events}
        	\begin{parameters}
        		\para{INativeMouse nativeM}{Required for capturing MouseMoveEvent.}
        	\end{parameters}
        \end{method}
        \begin{method}{public void StopCapture()}{Stop capturing mouse move events}
        \end{method}
    \end{methods}
\end{class}

\begin{class}{MouseScrollEventProducer} 
	\clsdiagram[scale = 0.7]{resources/Classes/Modules/Mouse/Producers/MouseScrollEventProducer.png}
    
    \clsdcl{public class MouseScrollEventProducer : DefaultEventQueue<MouseScrollEvent>}

    \clsdsp{Provides single-writer-multiple-reader queue for MouseScrollEvent}

    \begin{methods}
        \begin{method}{public void StartCapture()}{Starts capturing mouse scroll events}
        \end{method}
        \begin{method}{public void StopCapture()}{Stop capturing mouse scroll events}
        \end{method}
    \end{methods}
\end{class}
