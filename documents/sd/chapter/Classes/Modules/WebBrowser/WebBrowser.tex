\subsection{Webbrowser}

\begin{class}{WebBrowserModule} 
	\clsdiagram[width=0.80\textwidth]{resources/Classes/Modules/WebBrowser/WebBrowserModule.png}
	
    \clsdcl{public class WebBrowserModule : IModule}

    \clsdsp{The WebBrowserModule is responsible for recording all browser related user interactions}

\end{class}

\begin{enum}{EventLabel}	
	\clsdiagram[scale = 1]{resources/Classes/Modules/WebBrowser/EventLabel.png}
	
	\clsdcl{public enum EventLabel}	
	
	\clsdsp{Enum used to map the incoming serialized data (from the browser extension)}
	
	\begin{elements}
		\element{BUTTONCLICK}{Maps to button click event}
		\element{CLOSETAB}{Maps to close tab event}
		\element{DOWNLOAD}{Maps to file download event}
		\element{HOVER}{Maps to hover event}
		\element{NAVIGATION}{Maps to navigation event}
		\element{OPENTAB}{Maps to open tab event}
		\element{SWITCHTAB}{Maps to switch tab event}
		\element{TEXTINPUT}{Maps to text input event}
		\element{TEXTSELECTION}{Maps to text selection event}
	\end{elements}

\end{enum}

\subsection*{Configuration}

\begin{class}{WebBrowserModuleConfiguration}
	\clsdiagram[scale = 0.7]{resources/Classes/Modules/WebBrowser/WebBrowserModuleConfiguration.png}
	
	\clsdcl{public class WebBrowserModuleConfiguration: IConfiguration}
	
	\clsdsp{Configuration for WebBrowser module}
	
	\begin{attributes}
		\attribute{public string UrlSuffix}{A combination of port number and optionally directory.}
	\end{attributes}
\end{class}

\subsection*{Events}

\begin{absclass}{WebBrowserEvent}
	\clsdiagram[scale = 0.7]{resources/Classes/Modules/WebBrowser/Events/WebBrowserEvent.png}

    \clsdcl{public abstract class WebBrowserEvent : Event}

    \clsdsp{A generic web browser event which all specific WebBrowserEvents inherit from}

    \begin{attributes}
        \attribute{public int TabID}{The identifier of the tab where the web browser event occured in}
        \attribute{public System.Uri CurrentURL}{The URL of the website where the web browser event occured in}
    \end{attributes}
    
    \begin{methods}
    	\begin{method}{public void Deserialize(string serialized)}{Deserialize a browser event from a string.}
    		\begin{parameters}
    			\para{string serialized}{The serialized event}
    		\end{parameters}
    	\end{method}
    	\begin{method}{public void Deserialize(System.Text.Json.JsonElement parsed)}{Deserialize a browser event from a JSONElement instance.}
    		\begin{parameters}	
    			\para{System.Text.Json.JsonElement parsed}{A JSONElement parsed from a serialized event.}
    		\end{parameters}
    	\end{method}
    \end{methods}

\end{absclass}

\begin{class}{ButtonClickEvent : WebBrowserEvent} 
	\clsdiagram[scale = 0.7]{resources/Classes/Modules/WebBrowser/Events/ButtonClickEvent.png}
    
    \clsdcl{public class ButtonClickEvent}

    \clsdsp{A button click user interaction}

    \begin{attributes}
        \attribute{public string Button}{The title of the button item that was clicked on the website}
        \attribute{public string Href}{The URL of the website that the button is linked to}
    \end{attributes}
\end{class}

\begin{class}{CloseTabEvent} 
    \clsdiagram[scale = 0.7]{resources/Classes/Modules/WebBrowser/Events/CloseTabEvent.png}
    	
    \clsdcl{public class CloseTabEvent : WebBrowserEvent}

    \clsdsp{A close tab user interaction}

\end{class}

\begin{class}{FileDownloadEvent} 
 	\clsdiagram[scale = 0.7]{resources/Classes/Modules/WebBrowser/Events/FileDownloadEvent.png}
 	   
    \clsdcl{public class FileDownloadEvent : WebBrowserEvent}

    \clsdsp{A file download user interaction}

    \begin{attributes}
        \attribute{public string MIMEType}{MIME type of the file that was downloaded}
        \attribute{public System.Uri FileURL}{The URL of the file that was downloaded}
    \end{attributes}
\end{class}

\begin{class}{HoverEvent} 
    \clsdiagram[scale = 0.7]{resources/Classes/Modules/WebBrowser/Events/HoverEvent.png}
    	
    \clsdcl{public class HoverEvent : WebBrowserEvent}

    \clsdsp{A hover user interaction}

    \begin{attributes}
        \attribute{public string HoveredElement}{The element on the website that has been hovered}
    \end{attributes}
\end{class}

\begin{class}{NavigationEvent} 
    \clsdiagram[scale = 0.7]{resources/Classes/Modules/WebBrowser/Events/NavigationEvent.png}
    
    \clsdcl{public class NavigationEvent : WebBrowserEvent}

    \clsdsp{A navigation user interaction}

\end{class}

\begin{class}{OpenTabEvent} 
    \clsdiagram[scale = 0.7]{resources/Classes/Modules/WebBrowser/Events/OpenTabEvent.png}
    
    \clsdcl{public class OpenTabEvent : WebBrowserEvent}

    \clsdsp{An open tab user interaction}

\end{class}

\begin{class}{SwitchTabEvent} 
    \clsdiagram[scale = 0.7]{resources/Classes/Modules/WebBrowser/Events/SwitchTabEvent.png}
    
    \clsdcl{public class SwitchTabEvent : WebBrowserEvent}

    \clsdsp{A switch tab user interaction}

    \begin{attributes}
        \attribute{public int NewTabID}{The identifier of the tab that the user switched to}
    \end{attributes}
\end{class}

\begin{class}{TextInputEvent} 
    \clsdiagram[scale = 0.7]{resources/Classes/Modules/WebBrowser/Events/TextInputEvent.png}
    
    \clsdcl{public class TextInputEvent : WebBrowserEvent}

    \clsdsp{A text input user interaction}

    \begin{attributes}
        \attribute{public string InputtedText}{The text that was inputted by the user on the website}
        \attribute{public string Textbox}{The textbox where the text was inputted in}
    \end{attributes}
\end{class}

\begin{class}{TextSelectionEvent} 
    \clsdiagram[scale = 0.7]{resources/Classes/Modules/WebBrowser/Events/TextSelectionEvent.png}
    
    \clsdcl{public class TextSelectionEvent : WebBrowserEvent}

    \clsdsp{A text selection user interaction}

    \begin{attributes}
        \attribute{public string SelectedText}{The text that was selected on the website}
    \end{attributes}
\end{class}

\subsection*{Producers}

\begin{absclass}{WebBrowserEventProducer}
	\clsdiagram[scale = 0.9]{resources/Classes/Modules/WebBrowser/Producers/WebBrowserEventProducer.png}
	
	\clsdcl{public abstract class WebBrowserEventProducer<T> : DefaultEventQueue<T>, IWebBrowserEventObserver where T : WebBrowserEvent}
	
	\clsdsp{A generic producer for WebBrowserEvents, which will need to be subscribed to a IWebBrowserEventObserver to gather incoming event data.}
	
	\begin{attributes}
		\attribute{public virtual EventLabel HandledEventLabel}{The BrowserEvent label to be handled by the producer.}
	\end{attributes}
	
	\begin{methods}
		\begin{method}{public virtual void Notify(System.Text.Json.JsonElement eventJson)}{Simply forward the event to the internal queue if its of the appropriate type. Ignore otherwise.}
			\begin{parameters}
				\para{System.Text.Json.JsonElement eventJson}{A JsonElement holding an event.}
			\end{parameters}
			
			\begin{exceptions}
				\excp{NotSupportedException}{Exception is thrown when trying to invoke notify() on abstract WebBrowserEventProducer.}
			\end{exceptions}
			
		\end{method}
		
		\begin{method}{public void EnqueueFinished()}{Lets consumers know that no more events will be enqueued.}
		\end{method}
	
	\end{methods}
\end{absclass}
			
\begin{class}{ButtonClickEventProducer}
 
	\clsdiagram[scale = 0.9]{resources/Classes/Modules/WebBrowser/Producers/ButtonClickEventProducer.png}
    
    \clsdcl{public class ButtonClickEventProducer : WebBrowserEventProducer<ButtonClickEvent>}

    \clsdsp{Provides single-writer-multiple-reader queue for ButtonClickEvent}
    
    \begin{attributes}
    	\attribute{public override EventLabel HandledEventLabel => EventLabel.BUTTONCLICK}{The BrowserEvent label to be handled by ButtonClickEventProducer.}
    \end{attributes}
    
\end{class}

\begin{class}{CloseTabEventProducer} 
	\clsdiagram[scale = 0.9]{resources/Classes/Modules/WebBrowser/Producers/CloseTabEventProducer.png}
    
    \clsdcl{public class CloseTabEventProducer : WebBrowserEventProducer<CloseTabEvent>}

    \clsdsp{Provides single-writer-multiple-reader queue for CloseTabEvent}
    
    \begin{attributes}
    	\attribute{public override EventLabel HandledEventLabel => EventLabel.CLOSETAB}{The BrowserEvent label to be handled by CloseTabEventProducer.}
    \end{attributes}

\end{class}

\begin{class}{FileDownloadEventProducer} 
	\clsdiagram[scale = 0.9]{resources/Classes/Modules/WebBrowser/Producers/FileDownloadEventProducer.png}
    
    \clsdcl{public class FileDownloadEventProducer :  WebBrowserEventProducer<FileDownloadEvent>}

    \clsdsp{Provides single-writer-multiple-reader queue for FileDownloadEvent}

    \begin{attributes}
    	\attribute{public override EventLabel HandledEventLabel => EventLabel.DOWNLOAD}{The BrowserEvent label to be handled by FileDownloadEventProducer.}
    \end{attributes}
    
\end{class}

\begin{class}{HoverEventProducer} 
	\clsdiagram[scale = 0.9]{resources/Classes/Modules/WebBrowser/Producers/HoverEventProducer.png}
    
    \clsdcl{public class HoverEventProducer : WebBrowserEventProducer<HoverEvent>}

    \clsdsp{Provides single-writer-multiple-reader queue for HoverEvent}

    \begin{attributes}
    	\attribute{public override EventLabel HandledEventLabel => EventLabel.HOVER}{The BrowserEvent label to be handled by HoverEventProducer.}
    \end{attributes}
\end{class}

\begin{class}{NavigationEventProducer} 
	\clsdiagram[scale = 0.9]{resources/Classes/Modules/WebBrowser/Producers/NavigationEventProducer.png}
    
    \clsdcl{public class NavigationEventProducer :  WebBrowserEventProducer<NavigationEvent>}

    \clsdsp{Provides single-writer-multiple-reader queue for NavigationEvent}

    \begin{attributes}
    	\attribute{public override EventLabel HandledEventLabel => EventLabel.NAVIGATION}{The BrowserEvent label to be handled by NavigationEventProducer.}
    \end{attributes}
\end{class}

\begin{class}{OpenTabEventProducer} 
	\clsdiagram[scale = 0.9]{resources/Classes/Modules/WebBrowser/Producers/OpenTabEventProducer.png}
    
    \clsdcl{public class OpenTabEventProducer : WebBrowserEventProducer<OpenTabEvent>}

    \clsdsp{Provides single-writer-multiple-reader queue for OpenTabEvent}

    \begin{attributes}
    	\attribute{public override EventLabel HandledEventLabel => EventLabel.OPENTAB}{The BrowserEvent label to be handled by OpenTabEventProducer.}
    \end{attributes}
\end{class}

\begin{class}{SwitchTabEventProducer} 
	\clsdiagram[scale = 0.9]{resources/Classes/Modules/WebBrowser/Producers/SwitchTabEventProducer.png}
    
    \clsdcl{public class SwitchTabEventProducer : WebBrowserEventProducer<SwitchTabEvent>}

    \clsdsp{Provides single-writer-multiple-reader queue for SwitchTabEvent}

    \begin{attributes}
    	\attribute{public override EventLabel HandledEventLabel => EventLabel.SWITCHTAB}{The BrowserEvent label to be handled by SwitchTabEventProducer.}
    \end{attributes}
\end{class}

\begin{class}{TextInputEventProducer} 
	\clsdiagram[scale = 0.9]{resources/Classes/Modules/WebBrowser/Producers/TextInputEventProducer.png}
    
    \clsdcl{public class TextInputEventProducer : WebBrowserEventProducer<TextInputEvent>}

    \clsdsp{Provides single-writer-multiple-reader queue for TextInputEvent}

    \begin{attributes}
    	\attribute{public override EventLabel HandledEventLabel => EventLabel.TEXTINPUT}{The BrowserEvent label to be handled by TextInputEventProducer.}
    \end{attributes}
\end{class}

\begin{class}{TextSelectionEventProducer} 
	\clsdiagram[scale = 0.9]{resources/Classes/Modules/WebBrowser/Producers/TextSelectionEventProducer.png}
    
    \clsdcl{public class TextSelectionEventProducer:\\\ WebBrowserEventProducer<TextSelectionEvent>}

    \clsdsp{Provides single-writer-multiple-reader queue for TextSelectionEvent}

    \begin{attributes}
    	\attribute{public override EventLabel HandledEventLabel => EventLabel.TEXTSELECTION}{The BrowserEvent label to be handled by TextSelectionEventProducer.}
    \end{attributes}
\end{class}

