\chapter{Classes}
\label{ch:class}
%%%%%%%%%%%%%%%%%%%%%%%%%%%%%%%%%% Notes about this chapter%%%%%%%%%%%%%%%%%%%%%%%%%%%%%%%%%%%%%
% Each namespace correspondes to a section
% Each project in a namespace correspondes to a subsection in its namespaces' section.
% Each class, abstract class or interface in the project corresponds to a subsubsection in its project's subsection.
% Each class, abstract class or interface has some of the followings: class diagram, description, declaration, constructor, fields and Methods.
% Each of the class diagram, description, declaration, constructor, fields and Methods corresponds to a paragraph in its class' subsubsection.
%
% namespace------section
%     |            |
%  project-----subsection
%     ||           |
%   class-------subsubsection
%     |            |
%   methods----paragraph
%
%  IMPORTANT: the structure above should corresponds to the directory structure chapter. \autoref{ch:dirstructure}
%
% At the start of each namespace \section{namespace name} should be called to start a new section
% At the start of each project \subsection{project name} should be called to start a new subsection
% At the start of each class, abstract class or interface, a corresponding command \cls----- will be called to start a new subsubsection
% At the start of class diagram, description, declaration, constructor, fields and Methods,
% a corresponding commad \cls----- will be called to start a new paragraph.
% the content of the corresponding paragraph will be descibed either in text(in Description) or with a itemize (e.g. fields)
% and the content of a declaration, fields and method signature should use \texttt{}.
%
%%%%%%%%%%%%%%%%%%%%%%%%%%%%%%%%%% Notes about this chapter%%%%%%%%%%%%%%%%%%%%%%%%%%%%%%%%%%%%%

\newenvironment{class}[1]{\subsubsection{{\textbf{Class: #1}}}}{\newpage}
\newenvironment{absclass}[1]{\subsubsection{{\textbf{Abstract Class: #1}}}}{\newpage}
\newenvironment{interface}[1]{\subsubsection{{\textbf{Interface: #1}}}}{\newpage}

\newcommand{\clsdiagram}{\paragraph{\textbf{Class diagram:\\}}}
\newcommand{\clsdsp}{\paragraph{\textbf{Description:\\}}}
\newcommand{\clsdcl}{\paragraph{\textbf{Declaration:\\}}}
\newcommand{\clscon}{\paragraph{\textbf{Constructor:\\}}}
\newcommand{\clsfield}{\paragraph{\textbf{Fields:\\}}}
\newcommand{\clsmethod}{\paragraph{\textbf{Methods:\\}}}

\newcommand{\dsp}[1]{Description: #1}   %escription of a constructor, a method, a field, a parameter, a return value and so on.
\section{Core}

\section{Modules}

\subsection{Window Management}
\subsection{Mouse Interaction}
\subsection{Keyboard Interaction}
\subsection{Clipboard}
\subsection{Web browser}

\section{Shared}
