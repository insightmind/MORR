\chapter{Classes}
\label{ch:class}
%%%%%%%%%%%%%%%%%%%%%%%%%%%%%%%%%% Notes about this chapter%%%%%%%%%%%%%%%%%%%%%%%%%%%%%%%%%%%%%
% Each namespace correspondes to a section
% Each project in a namespace correspondes to a subsection in its namespaces' section.
% Each class, abstract class or interface in the project corresponds to a subsubsection in its project's subsection.
% Each class, abstract class or interface has some of the followings: class diagram, description, declaration, constructor, attributes and Methods.
%
% namespace------section
%     |            |
%  project-----subsection
%     ||           |
%   class-------subsubsection
%
%  IMPORTANT: the structure above should corresponds to the directory structure chapter. \autoref{ch:dirstructure}
%
% At the start of each namespace \section{namespace name} should be called to start a new section
% At the start of each project \subsection{project name} should be called to start a new subsection
% At the start of each class, abstract class or interface, a corresponding environment \begin{class},... will be called to start a new subsubsection
% At the start of class diagram, description, declaration, constructor, fields and Methods,
% a corresponding commad \cls----- or environment \begin{methods}... will be called to start a new paragraph.
%
%%%%%%%%%%%%%%%%%%%%%%%%%%%%%%%%%% Notes about this chapter%%%%%%%%%%%%%%%%%%%%%%%%%%%%%%%%%%%%%

\newenvironment{class}[1]{\subsubsection{{\textbf{Class: #1}}}}{\newpage}
\newenvironment{absclass}[1]{\subsubsection{{\textbf{Abstract Class: #1}}}}{\newpage}
\newenvironment{interface}[1]{\subsubsection{{\textbf{Interface: #1}}}}{\newpage}

%include a class diagram
%para1: the directory of the diagram
\newcommand{\clsdiagram}[1]{\textbf{Class diagram:} \newline 
\begin{center}
    \includegraphics{#1}
\end{center}}

% The declaration of a class, abstrat class or interface.
% para1: declaration in text
\newcommand{\clsdcl}[1]{\textbf{Declaration:} \newline \texttt{#1}} %para1: description 

% The description of a class, abstrat class or interface.
% para1: description in text
\newcommand{\clsdsp}[1]{\textbf{Description:} \newline #1}

% Constructors itemize with citem(constructor item)
\newenvironment{constructors}{\textbf{Constructors:}  \begin{itemize}
}{\end{itemize}}
% citem
% para1: declaration para2: description
\newcommand{\citem}[2]{\item \texttt{#1} \\ Description: #2}

% Attributes itemize with aitem(attribute item)
\newenvironment{attributes}{\textbf{Attributes:}  \begin{itemize}
}{\end{itemize}}
% aitem
% para1: declaration para2: description
\newcommand{\aitem}[2]{\item \texttt{#1} \\ Description: #2}

% Methods itemize with mitem(method item)
\newenvironment{methods}{\textbf{Methods:}  \begin{itemize}
}{\end{itemize}}
% mitem
% para1: declaration para2: description
\newcommand{\mitem}[2]{\item \texttt{#1} \\ Description: #2}

%%%%%%%%%%%%%%%%%%%%%%%%%%%%%%%% EXAMPLE %%%%%%%%%%%%%%%%%%%%%%%%%%%%%%%
\section{DummyNamespace}
\subsection{DummyProject}
\begin{absclass}{Dummy} 
    \clsdiagram{resources/class/DummyNamespace/DummyProject/dummy.png}

    \clsdcl{public abstract class Dummy}

    \clsdsp{This is a dummy class}

    \begin{attributes}
        \aitem{private String data}{The data of the dummy}
    \end{attributes}

    \begin{methods}
        \aitem{public void reproduce()}{reproduce a new dummy}
    \end{methods}
\end{absclass}
%%%%%%%%%%%%%%%%%%%%%%%%%%%%%%%% EXAMPLE %%%%%%%%%%%%%%%%%%%%%%%%%%%%%%%

\section{Core}

\section{Modules}

\subsection{Window Management}
\subsection{Mouse Interaction}
\subsection{Keyboard Interaction}
\subsection{Clipboard}
\subsection{Web browser}

\section{Shared}
