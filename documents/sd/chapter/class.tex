\chapter{Classes}
\label{ch:class}
%%%%%%%%%%%%%%%%%%%%%%%%%%%%%%%%%% Notes about this chapter%%%%%%%%%%%%%%%%%%%%%%%%%%%%%%%%%%%%%
% Each namespace correspondes to a section
% Each project in a namespace correspondes to a subsection in its namespaces' section.
% Each class, abstract class, enumeration or interface in the project corresponds to a subsubsection in its project's subsection.
% Each class, abstract class, enumeration or interface has some of the followings: class diagram, description, declaration, constructor, attributes and Methods.
%
% namespace------section
%     |            |
%  project-----subsection
%     ||           |
%   class-------subsubsection
%
%  IMPORTANT: the structure above should corresponds to the directory structure chapter. \autoref{ch:dirstructure}
%
% At the start of each namespace \section{namespace name} should be called to start a new section
% At the start of each project \subsection{project name} should be called to start a new subsection
% At the start of each class, abstract class, enumeration or interface, a corresponding environment \begin{class},... will be called to start a new subsubsection
% At the start of class diagram, description, declaration, constructor, fields and Methods,
% a corresponding commad \cls----- or environment \begin{methods}... will be called to start a new paragraph.
%
%%%%%%%%%%%%%%%%%%%%%%%%%%%%%%%%%% Notes about this chapter%%%%%%%%%%%%%%%%%%%%%%%%%%%%%%%%%%%%%

%T he description
% para1: the description in text
\newcommand{\dsp}{\textbf{Description: }}

\newenvironment{class}[1]{\subsubsection{{\textbf{Class: #1}}}\index{#1 (Class)}}{\newpage}
\newenvironment{absclass}[1]{\subsubsection{{\textbf{Abstract Class: #1}}}\index{#1 (Abstract Class)}}{\newpage}
\newenvironment{interface}[1]{\subsubsection{{\textbf{Interface: #1}}}\index{#1 (Interface)}}{\newpage}
\newenvironment{enum}[1]{\subsubsection{{\textbf{Enumeration: #1}}}\index{#1 (Enumeration)}}{\newpage}
 
% Enum elements enumeration
\newenvironment{elements}{\textbf{Elements:}  \begin{enumerate}
}{\end{enumerate}}
% Eeum element
% para1: declaration para2: description
\newcommand{\element}[2]{\item \texttt{#1} \\ \dsp #2}

%include a class diagram
% optional para1: the optional argument of \includegraphics
% para2: the directory of the diagram
\newcommand{\clsdiagram}[2][]{\textbf{Class diagram:} \newline
\begin{center}
    \includegraphics[#1]{#2}
\end{center}}

% The declaration of a class, abstrat class or interface.
% para1: declaration in text
\newcommand{\clsdcl}[1]{\textbf{Declaration:} \newline \texttt{#1}} %para1: description 

% The description of a class, abstrat class or interface.
% para1: description in text
\newcommand{\clsdsp}[1]{\textbf{\dsp} \newline #1}

% Constructors enumeration with \constructor (constructor item)
\newenvironment{constructors}{\textbf{Constructors:}  \begin{enumerate}
}{\end{enumerate}}
% constructor
% para1: declaration para2: description
\newenvironment{constructor}[2]{\item \texttt{#1} \\ \dsp #2 \\}{}

% Attributes enumeration with attribute(attribute item)
\newenvironment{attributes}{\textbf{Attributes:}  \begin{enumerate}
}{\end{enumerate}}
% attribute
% para1: declaration para2: description
\newcommand{\attribute}[2]{\item \texttt{#1} \\ \dsp #2}

% Methods enumeration with method(method item)
\newenvironment{methods}{\textbf{Methods:}  \begin{enumerate}
}{\end{enumerate}}
% method
% para1: declaration para2: description
\newenvironment{method}[2]{\item \texttt{#1} \\ \dsp #2 \\}{}

% Parameters enumeration with \para (parameter item)
\newenvironment{parameters}{\textbf{Parameters:}  \begin{enumerate}
}{\end{enumerate}}
% parameter
% para1: declaration para2: description
\newcommand{\para}[2]{\item \texttt{#1} \\ \dsp #2}

% Exceptions enumeration with \excp (exception item)
\newenvironment{exceptions}{\textbf{Exceptions:} \begin{enumerate}
}{\end{enumerate}}
% exception
% para1: exception type para2: condition for throwing
\newcommand{\excp}[2]{\item \texttt{#1} \\ \dsp #2}

% return value
% para1: declaration para2: description
\newcommand{\return}[2]{\textbf{Return values:} \begin{itemize} \item \texttt{#1} \\ Description: #2 \end{itemize}}

\section{Core}
\subsection{CLI}
\input{chapter/Classes/Core/CLI/Classes/Program}

\subsection{MORR}

\subsection*{MORR.Core}

\begin{interface}{IBootstrapper}
    \clsdiagram{resources/Classes/Core/MORR/IBootstrapper.png}

    \clsdcl{public interface IBootstrapper}

    \clsdsp{Responsible for bootstrapping the application and providing compositional facilities.}

    \begin{methods}
        \begin{method}{void ComposeImports(object @object)}{Composes the provided object.}
            \begin{parameters}
                \para{object @object}{The object to satisfy the imports on.}
            \end{parameters}
        \end{method}
    \end{methods}
\end{interface}

\begin{class}{Bootstrapper}
    \clsdiagram{resources/Classes/Core/MORR/Bootstrapper.png}
    
    \clsdcl{public class Bootstrapper : IBootstrapper}

    \clsdsp{Bootstraps the application using MEF.}

    \begin{methods}
        \begin{method}{public Bootstrapper()}{Creates a new instance of the Bootstrapper and loads all .MORR-Module.dll assemblies from the Modules subdirectory relative to the directory of the current executing assembly.}
        \end{method}
        \begin{method}{public void ComposeImports(object @object)}{Satisfies the imports on the provided object.}
            \begin{parameters}
                \para{object @object}{The object to satisfy the imports on.}
            \end{parameters}
        \end{method}
    \end{methods}
\end{class}

\begin{class}{BootstrapperConventions}
    \clsdiagram{resources/Classes/Core/MORR/BootstrapperConventions.png}

    \clsdcl{public static class BootstrapperConventions}

    \clsdsp{Provides conventions for composing objects with MEF.}

    \begin{methods}
        \begin{method}{public static RegistrationBuilder GetRegistrationBuilder()}{Gets a registration builder that contains all composition conventions.}
            \return{RegistrationBuilder}{A RegistrationBuilder containing the composition conventions.}
        \end{method}
    \end{methods}
\end{class}

\subsection*{MORR.Core.Configuration}

\begin{interface}{IConfigurationManager}
    \clsdiagram{resources/Classes/Core/MORR/Configuration/IConfigurationManager.png}

    \clsdcl{public interface IConfigurationManager}

    \clsdsp{Loads and manages the application's configuration.}

    \begin{methods}
        \begin{method}{void LoadConfiguration(FilePath path)}{Loads the configuration from the specified path.}
            \begin{parameters}
                \para{FilePath path}{The path to load the configuration from.}
            \end{parameters}
            \begin{exceptions}
                \excp{InvalidConfigurationException}{Exception thrown if the specified configuration is invalid.}
            \end{exceptions}
        \end{method}
    \end{methods}
\end{interface}

\begin{class}{ConfigurationManager}
    \clsdiagram{resources/Classes/Core/MORR/Configuration/ConfigurationManager.png}

    \clsdcl{public class ConfigurationManager : IConfigurationManager}

    \clsdsp{Manages the application's configuration.}

    \begin{methods}
        \begin{method}{public void LoadConfiguration(FilePath path)}{Loads the configuration from the specified path.}
            \begin{parameters}
                \para{FilePath path}{The path to load the configuration from.}
            \end{parameters}
            \begin{exceptions}
                \excp{InvalidConfigurationException}{Exception thrown if the specified configuration is invalid.}
            \end{exceptions}
        \end{method}
    \end{methods}
\end{class}

\begin{class}{InvalidConfigurationException}
    \clsdiagram{resources/Classes/Core/MORR/Configuration/InvalidConfigurationException.png}

    \clsdcl{public class InvalidConfigurationException : Exception}

    \clsdsp{An exception thrown when the configuration is invalid.}

    \begin{constructors}
        \begin{constructor}{public InvalidConfigurationException()}{Creates a new instance of an InvalidConfigurationException without any specific error message.}
        \end{constructor}
        \begin{constructor}{public InvalidConfigurationException(string message)}{Creates a new instance of an InvalidConfigurationException with the specified error message.}
            \begin{parameters}
                \para{string message}{The error message to construct the instance with.}
            \end{parameters}
        \end{constructor}
        \begin{constructor}{public InvalidConfigurationException(string message, Exception innerException)}{Creates a new instance of an InvalidConfigurationException with the specified error message and inner exception.}
            \begin{parameters}
                \para{string message}{The error message to construct the instance with.}
                \para{Exception innerException}{The inner exception to construct the instance with.}
            \end{parameters}
        \end{constructor}
    \end{constructors}
\end{class}

\subsection*{MORR.Core.Session}

\begin{interface}{ISessionManager}
    \clsdiagram{resources/Classes/Core/MORR/Session/ISessionManager.png}

    \clsdcl{public interface ISessionManager}

    \clsdsp{A manager responsible for all aspects of recording and processing.}

    \begin{attributes}
        \attribute{DirectoryPath? CurrentRecordingDirectory \{ get; \}}{The path to the directory containing the most recent recording or null if no recording has been created yet.}
        \attribute{DirectoryPath? RecordingsFolder \{ get; \}}{The path to the top-level folder containing the recording subdirectories.}
    \end{attributes}

    \begin{methods}
        \begin{method}{public void StartRecording()}{Starts a recording if no session is currently being recorded.}
            \begin{exceptions}
                \excp{AlreadyRecordingException}{Thrown if a recording is started while another recording is already running.}
            \end{exceptions}
        \end{method}
        \begin{method}{public void StopRecording()}{Stops a recording if a session is currently being recorded.}
            \begin{exceptions}
                \excp{NotRecordingException}{Thrown if a recording is stopped while no recording is currently running.}
            \end{exceptions}
        \end{method}
        \begin{method}{void Process(IEnumerable<DirectoryPath> recordings)}{Processes the specified recordings.}
            \begin{parameters}
                \para{IEnumerable<DirectoryPath> recordings}{The recordings to process.}
            \end{parameters}
        \end{method}
    \end{methods}
\end{interface}

\begin{class}{SessionManager}
    \clsdiagram{resources/Classes/Core/MORR/Session/SessionManager.png}

    \clsdcl{public class SessionManager : ISessionManager}

    \clsdsp{A manager responsible for all aspects of recording and processing.}

    \begin{attributes}
        \attribute{DirectoryPath? CurrentRecordingDirectory \{ get; \}}{The path to the directory containt the most recent recording or null if no recording has been created yet.}
        \attribute{DirectoryPath? RecordingsFolder \{ get; \}}{The path to the top-level folder containing the recording subdirectories.}
    \end{attributes}

    \begin{constructors}
        \begin{constructor}{public SessionManager(FilePath configurationPath)}{Creates a new instance of a SessionManager with the specified configuration path and a default Bootstrapper, ConfigurationManager and ModuleManager}
            \begin{parameters}
                \para{FilePath configurationPath}{The configuration path to construct the instance with.}
            \end{parameters}
        \end{constructor}
        \begin{constructor}{public SessionManager(FilePath configurationPath, IBootstrapper bootstrapper, IConfigurationManager configurationManager, IModuleManager moduleManager)}{Creates a new instance of a SessionManager with the specified configuration path, bootstrapper, configuration manager and module manager.}
            \begin{parameters}
                \para{FilePath configurationPath}{The configuration path to construct the instance with.}
                \para{IBootstrapper bootstrapper}{The bootstrapper to construct the instance with.}
                \para{IConfigurationManager configurationManager}{The configuration manager to construct the instance with.}
                \para{IModuleManager moduleManager}{The module manager to construct the instance with.}
            \end{parameters}
        \end{constructor}
    \end{constructors}

    \begin{methods}
        \begin{method}{public void StartRecording()}{Starts a recording if no session is currently being recorded.}
            \begin{exceptions}
                \excp{AlreadyRecordingException}{Thrown if a recording is started while another recording is already running.}
            \end{exceptions}
        \end{method}
        \begin{method}{public void StopRecording()}{Stops a recording if a session is currently being recorded.}
            \begin{exceptions}
                \excp{NotRecordingException}{Thrown if a recording is stopped while no recording is currently running.}
            \end{exceptions}
        \end{method}
        \begin{method}{void Process(IEnumerable<DirectoryPath> recordings)}{Processes the specified recordings.}
            \begin{parameters}
                \para{IEnumerable<DirectoryPath> recordings}{The recordings to process.}
            \end{parameters}
        \end{method}
    \end{methods}
\end{class}

\subsection*{MORR.Core.Session.Exceptions}

\begin{class}{RecordingException}
    \clsdiagram{resources/Classes/Core/MORR/Session/Exceptions/RecordingException.png}

    \clsdcl{public class RecordingException : Exception}

    \clsdsp{A generic recording exception that all specialized recording exceptions derive from.}
\end{class}

\begin{class}{AlreadyRecordingException}
    \clsdiagram{resources/Classes/Core/MORR/Session/Exceptions/AlreadyRecordingException.png}

    \clsdcl{public class AlreadyRecordingException : RecordingException}

    \clsdsp{An exception thrown if a recording is started while another recording is already running.}
\end{class}

\begin{class}{NotRecordingException}
    \clsdiagram{resources/Classes/Core/MORR/Session/Exceptions/NotRecordingException.png}

    \clsdcl{public class NotRecordingException : RecordingException}

    \clsdsp{An exception thrown if a recording is stopped while no recording is currently running.}
\end{class}

\subsection*{MORR.Core.Modules}

\begin{interface}{IModuleManager}
    \clsdiagram{resources/Classes/Core/MORR/Modules/IModuleManager.png}

    \clsdcl{public interface IModuleManager}

    \clsdsp{Initializes and manages all modules.}

    \begin{methods}
        \begin{method}{void InitializeModules()}{Initializes all modules.}      
        \end{method}
        \begin{method}{void NotifyModulesOnSessionStart()}{Notifies all modules when a session starts.}
        \end{method}
        \begin{method}{void NotifyModulesOnSessionStop()}{Notifies all modules when a session stops.}
        \end{method}
    \end{methods}
\end{interface}

\begin{class}{ModuleManager}
    \clsdiagram{resources/Classes/Core/MORR/Modules/ModuleManager.png}

    \clsdcl{public class ModuleManager : IModuleManager}

    \clsdsp{Initializes and manages all modules..}

    \begin{methods}
        \begin{method}{void InitializeModules()}{Initializes all modules.}      
        \end{method}
        \begin{method}{void NotifyModulesOnSessionStart()}{Notifies all modules when a session starts.}
        \end{method}
        \begin{method}{void NotifyModulesOnSessionStop()}{Notifies all modules when a session stops.}
        \end{method}
    \end{methods}
\end{class}

\begin{class}{GlobalModuleConfiguration}
    \clsdiagram{resources/Classes/Core/MORR/Modules/GlobalModuleConfiguration.png}

    \clsdcl{public class GlobalModuleConfiguration : IConfiguration}

    \clsdsp{Encapsulates all global module configuration options.}

    \begin{attributes}
        \attribute{public IEnumerable<Type> EnabledModules \{ get; \}}{The types of all IModule instances that should be enabled.}
    \end{attributes}
    
    \begin{methods}
        \begin{method}{public void Parse(RawConfiguration configuration)}{Parses the configuration from the provided value.}
            \begin{parameters}
                \para{RawConfiguration configuration}{The configuration value to parse from.}
            \end{parameters}
        \end{method}
    \end{methods}
\end{class}

\subsection*{MORR.Core.Data.Capture}

\begin{class}{CaptureException}
    \clsdiagram{resources/Classes/Core/MORR/Data/Capture/CaptureException.png}

    \clsdcl{public class CaptureException : Exception}

    \clsdsp{A generic capture exception that all specialized capture exceptions derive from.}
    
    \begin{constructors}
        \begin{constructor}{public CaptureException()}{Creates a new instance of a CaptureException without any specific error message.}
        \end{constructor}
        \begin{constructor}{public Exception(string message)}{Creates a new instance of a CaptureException with the specified error message.}
            \begin{parameters}
                \para{string message}{The error message to construct the instance with.}
            \end{parameters}
        \end{constructor}
        \begin{constructor}{public CaptureException(string message, Exception innerException)}{Creates a new instance of a CaptureException with the specified error message and inner exception.}
            \begin{parameters}
                \para{string message}{The error message to construct the instance with.}
                \para{Exception innerException}{The inner exception to construct the instance with.}
            \end{parameters}
        \end{constructor}
    \end{constructors}
\end{class}

\subsection*{MORR.Core.Data.Capture.Video}

\begin{class}{VideoSample}
    \clsdiagram{resources/Classes/Core/MORR/Data/Capture/Video/VideoSample.png}

    \clsdcl{public class VideoSample : Event}

    \clsdsp{A single video capture sample.}  
\end{class}

\begin{class}{DirectXVideoSample}
    \clsdiagram{resources/Classes/Core/MORR/Data/Capture/Video/DirectXVideoSample.png}

    \clsdcl{public class DirectXVideoSample : VideoSample}
    
    \clsdsp{A single video capture sample in DirectX format.}

    \begin{attributes}
        \attribute{public IDirect3DSurface Surface \{ get; \}}{The surface containing the data for this sample.}
    \end{attributes}
\end{class}

\subsection*{MORR.Core.Data.Capture.Video.Exceptions}

\begin{class}{VideoCaptureException}
    \clsdiagram{resources/Classes/Core/MORR/Data/Capture/Video/Exceptions/VideoCaptureException.png}

    \clsdcl{public class VideoCaptureException : CaptureException}

    \clsdsp{An exception thrown if video sample capturing fails.}

    \begin{constructors}
        \begin{constructor}{public VideoCaptureException()}{Creates a new instance of a VideoCaptureException without any specific error message.}
        \end{constructor}
        \begin{constructor}{public Exception(string message)}{Creates a new instance of a VideoCaptureException with the specified error message.}
            \begin{parameters}
                \para{string message}{The error message to construct the instance with.}
            \end{parameters}
        \end{constructor}
        \begin{constructor}{public VideoCaptureException(string message, Exception innerException)}{Creates a new instance of a VideoCaptureException with the specified error message and inner exception.}
            \begin{parameters}
                \para{string message}{The error message to construct the instance with.}
                \para{Exception innerException}{The inner exception to construct the instance with.}
            \end{parameters}
        \end{constructor}
    \end{constructors}
\end{class}

\subsection*{MORR.Core.Data.Capture.Video.Desktop}

\begin{class}{DesktopCapture}
    \clsdiagram{resources/Classes/Core/MORR/Data/Capture/Video/Desktop/DesktopCapture.png}

    \clsdcl{public class DesktopCapture : IModule}

    \clsdsp{Capture video from the desktop using the Windows API.}

    \begin{attributes}
        \attribute{public bool IsActive \{ get; set; \}}{Indicates whether the module is active. True if it is active, false otherwise.}
        \attribute{public Guid Identifier \{ get; \}}{The identifier of the module.}
    \end{attributes}

    \begin{methods}
        \begin{method}{public void Initialize(bool isEnabled)}{Initializes the module.}
            \begin{parameters}
                \para{bool isEnabled}{Indicates whether the module should be enabled. True if it should be enabled, false otherwise.}
            \end{parameters}
        \end{method}
    \end{methods}
\end{class}

\begin{class}{DesktopCaptureConfiguration}
    \clsdiagram{resources/Classes/Core/MORR/Data/Capture/Video/Desktop/DesktopCaptureConfiguration.png}

    \clsdcl{public class DesktopCaptureConfiguration : IConfiguration}

    \clsdsp{Encapsulates all desktop capture configuration options.}

    \begin{attributes}
        \attribute{public Index MonitorIndex \{ get; \}}{The index of the monitor to capture.}
        \attribute{public bool PromptUserForMonitorSelection \{ get; \}}{Indicates whether the user should be prompted to manually select the monitor to capture. True if the user should be prompted, false otherwise.}
    \end{attributes}

    \begin{methods}
        \begin{method}{public void Parse(RawConfiguration configuration)}{Parses the configuration from the provided value.}
            \begin{parameters}
                \para{RawConfiguration configuration}{The configuration value to parse from.}
            \end{parameters}
        \end{method}
    \end{methods}
\end{class}

\begin{class}{VideoSampleProducer}
    \clsdiagram{resources/Classes/Core/MORR/Data/Capture/Video/Desktop/VideoSampleProducer.png}

    \clsdcl{public class VideoSampleProducer : DefaultEncodeableEventQueue<DirectXVideoSample>}

    \clsdsp{Provides the captured video samples.}
\end{class}
\subsection*{MORR.Core.Data.IntermediateFormat}

\begin{absclass}{IntermediateFormatSample}
    \clsdiagram{resources/Classes/Core/MORR/Data/IntermediateFormat/IntermediateFormatSample.png}

    \clsdcl{public abstract class IntermediateFormatSample : Event}

    \clsdsp{A sample in an intermediate format.}

    \begin{attributes}
        \attribute{public Type Type \{ get; set; \}}{The type of the event that is serialized.}
        \attribute{public byte[] Data \{ get; set; \}}{The data that is serialized.}
    \end{attributes}
\end{absclass}

\subsection*{MORR.Core.Data.IntermediateFormat.Json}

\begin{class}{JsonIntermediateFormatSample}
    \clsdiagram{resources/Classes/Core/MORR/Data/IntermediateFormat/Json/JsonIntermediateFormatSample.png}

    \clsdcl{public class JsonIntermediateFormatSample : IntermediateFormatSample}

    \clsdsp{A sample in JSON intermediate format.}

    \begin{attributes}
        \attribute{public JsonDocument JsonEncodedData \{ get; \}}{The data that is serialized in JSON-compatible format.}
        \attribute{public JsonEncodedText JsonEncodedType \{ get; \}}{The type of the event that is serialized in JSON-compatible format.}
    \end{attributes}
\end{class}

\begin{class}{JsonIntermediateFormatSerializer}
    \clsdiagram{resources/Classes/Core/MORR/Data/IntermediateFormat/Json/JsonIntermediateFormatSerializer.png}

    \clsdcl{public class JsonIntermediateFormatSerializer : DefaultEncodeableEventQueue<JsonIntermediateFormatSample>, IModule}

    \clsdsp{A serializer for converting typed events into their representation in the JSON-intermediate format.}

    \begin{attributes}
        \attribute{public bool IsActive \{ get; set; \}}{Indicates whether the module is active. True if it is active, false otherwise.}
        \attribute{public Guid Identifier \{ get; \}}{The identifier of the module.}
    \end{attributes}

    \begin{methods}
        \begin{method}{public bool Initialize(bool isEnabled)}{Initializes the module.}
            \begin{parameters}
                \para{bool isEnabled}{Indicates whether the module is enabled. True if it is enabled, false otherwise.}
            \end{parameters}
        \end{method}
    \end{methods}
\end{class}

\begin{class}{JsonIntermediateFormatDeserializer}
    \clsdiagram{resources/Classes/Core/MORR/Data/IntermediateFormat/Json/JsonIntermediateFormatDeserializer.png}

    \clsdcl{public class JsonIntermediateFormatDeserializer : IModule}

    \clsdsp{A deserializer for converting events in their JSON-intermediate format representation to typed events.}

    \begin{attributes}
        \attribute{public bool IsActive \{ get; set; \}}{Indicates whether the module is active. True if it is active, false otherwise.}
        \attribute{public Guid Identifier \{ get; \}}{The identifier of the module.}
    \end{attributes}

    \begin{methods}
        \begin{method}{public bool Initialize(bool isEnabled)}{Initializes the module.}
            \begin{parameters}
                \para{bool isEnabled}{Indicates whether the module is enabled. True if it is enabled, false otherwise.}
            \end{parameters}
        \end{method}
    \end{methods}
\end{class}
% IDecoder
% IEncoder
% DecodingException
% EncodingException

\input{chapter/Classes/Core/MORR/Data/Transcoding/Metadata/MetadataTranscoding}
\input{chapter/Classes/Core/MORR/Data/Transcoding/Video/VideoTranscoding}
\input{chapter/Classes/Core/MORR/Data/Transcoding/WinAPI/WinAPI_Transcoding}
\subsection{UI}

\begin{class}{App}
	\clsdiagram[scale = 0.7]{resources/Classes/Core/UI/App.png}

    \clsdcl{public partial class App : Application}

    \clsdsp{Auto-generated WPF application class}

\end{class}

\subsection*{Controls. NotifyIcon}
\begin{class}{NotifyIcon}
	\clsdiagram[scale = 0.7]{resources/Classes/Core/UI/NotifyIcon.png}

    \clsdcl{public class NotifyIcon : Control, IDisposable}

    \clsdsp{Manages a tray icon}

	\begin{attributes}
        \attribute{public string Tooltip}{The tooltip shown during hovering}
        \attribute{public ImageSource IconSource}{The icon shown in the tray}
        \attribute{public ICommand Command}{The command to execute on left click}
        \attribute{public object CommandParameter}{The parameter of the command to execute on left click}
        \attribute{public new ContextMenu ContextMenu}{The context menu to show on right click}
    \end{attributes}

   	\begin{constructors}
        \begin{constructor}{public NotifyIcon()}{Initializes a NotifyIcon}
        \end{constructor}
    \end{constructors}
    
    \begin{methods}
    	\begin{method}{public void Dispose()}{Frees all unmanaged resources}
        \end{method}
     \end{methods}

\end{class}

\subsection*{Dialogs}

\begin{class}{ErrorDialog}
	\clsdiagram[scale = 0.7]{resources/Classes/Core/UI/ErrorDialog.png}

    \clsdcl{public partial class ErrorDialog : Window}

    \clsdsp{Interaction logic for ErrorDialog}

    \begin{attributes}
        \attribute{public string ErrorMessage}{Error message of the error dialog}
    \end{attributes}
    
    \begin{constructors}
        \begin{constructor}{public ErrorDialog(string errorMessage)}{Initializes an ErrorDialog with errorMessage}
            \begin{parameters}
                \para{String errorMessage}{A string with an error message}
            \end{parameters}
        \end{constructor}
     \end{constructors}

\end{class}

\begin{class}{InfomationDialog}
	\clsdiagram[scale = 0.7]{resources/Classes/Core/UI/InformationDialog.png}

    \clsdcl{public partial class InformationDialog : Window }

    \clsdsp{Interaction logic for InformationDialog}

     \begin{constructors}
        \begin{constructor}{public InformationDialog()}{Initializes an InformationDialog}
        \end{constructor}
     \end{constructors}

\end{class}

\begin{class}{SaveDialog}
	\clsdiagram[scale = 0.7]{resources/Classes/Core/UI/SaveDialog.png}

    \clsdcl{public partial class SaveDialog : Window}

    \clsdsp{Interaction logic for SaveDialog}
    
    \begin{constructors}
        \begin{constructor}{public SaveDialog()}{Initializes a SaveDialog}
        \end{constructor}
     \end{constructors}

\end{class}

\subsection*{ViewModels}

\begin{class}{ApplicationViewModel}
	\clsdiagram[scale = 0.7]{resources/Classes/Core/UI/ApplicationViewModel.png}

    \clsdcl{public class ApplicationViewModel : DependencyObject}

    \clsdsp{}

	\begin{attributes}
        \attribute{public bool IsRecording}{Current state of the recording process}
    \end{attributes}    
    
    \begin{constructors}
        \begin{constructor}{public ApplicationViewModel()}{Initializes an ApplicationViewModel}
        \end{constructor}
     \end{constructors}
 

\end{class}

\subsection*{ViewModels.Utility}

\begin{class}{RelayCommand}
	\clsdiagram[scale = 0.7]{resources/Classes/Core/UI/RelayCommand.png}

    \clsdcl{public RelayCommand : ICommand }

    \clsdsp{A concrete command that implements ICommand interface}

	\begin{constructors}
        \begin{constructor}{public RelayCommand(Action<object> execute, Predicate<object>? canExecute = null)}{Creates a new RelayCommand}
        \begin{parameters}
                \para{Action<object> execute}{The execution logicq}
                \para{Predicate<object>? canExecute = null}{The execution status logic, set to null}
            \end{parameters}
        \end{constructor}
     \end{constructors}
    
    
    \begin{methods}
    	\begin{method}{public bool CanExecute(object parameter)}{Determines whether the RelayCommand can be executed in current state}
            \return{bool}{True if this RelayCommand can be executed; otherwise, false.}
            \begin{parameters}
                \para{Object parameter}{Data used by the RelayCommand}
            \end{parameters}
        \end{method}
        \begin{method}{public void Execute(object parameter)}{Executes the RelayCommand}
            \begin{parameters}
               \para{Object parameter}{Data used by the RelayCommand}
            \end{parameters}
        \end{method}
        
    \end{methods}

\end{class}




\newpage
\section{Modules}
\subsection{Clipboard}

\begin{class}{ClipboardModule} 
    \clsdiagram{resources/Classes/Modules/Clipboard/ClipboardModule.png}

    \clsdcl{public class ClipboardModule}

    \clsdsp{The ClipboardModule is responsible for recording all clipboard related user interactions.}
\end{class}

\begin{absclass}{ClipboardEvent} 
    \clsdiagram{resources/Classes/Modules/Clipboard/Events/ClipboardEvent.png}

    \clsdcl{public abstract class ClipboardEvent : Event}

    \clsdsp{A generic clipboard event which all specific ClipboardEvents inherit from.}
\end{absclass}

\begin{class}{} 
    \clsdiagram{resources/Classes/Modules/Clipboard/Events/ClipBoardInteractEvent.png}

    \clsdcl{public class ClipBoardInteractEvent : ClipboardEvent}

    \clsdsp{A clipboard user interaction}

    \begin{attributes}
        \attribute{public string Text}{The text in the clipboard}
        \attribute{public InteractionType Interaction}{The interaction type}
    \end{attributes}
\end{class}

\subsection{Keyboard}
\begin{class}{KeyboardModule} 
    \clsdiagram[scale = 0.7]{resources/Classes/Modules/Keyboard/KeyboardModule.png}

    \clsdcl{public class KeyboardModule : IModule}

    \clsdsp{The KeyboardModule is responsible for recording all keyboard related user interactions.}
\end{class}

\subsection*{Events}

\begin{absclass}{KeyboardEvent} 
    \clsdiagram{resources/Classes/Modules/Keyboard/Events/KeyboardEvent.png}

    \clsdcl{public abstract class KeyboardEvent : Event}

    \clsdsp{A generic keyboard event which all specific KeyboardEvents inherit from.}
\end{absclass}

\begin{class}{KeyboardInteractEvent} 
    \clsdiagram{resources/Classes/Modules/Keyboard/Events/KeyBoardInteractEvent.png}

    \clsdcl{public class KeyboardInteractEvent : KeyboardEvent}

    \clsdsp{A keyboard user interaction.}

    \begin{attributes}
        \attribute{public System.Windows.Input.Key PressedKey\_System\_Windows\_Input\_Key}{The key that was pressed.}
        \attribute{public string PressedKeyName}{The name of the key on the keyboard that is pressed.}
        \attribute{public System.Windows.Input.ModifierKeys ModifierKeys\_System\_Windows\_Input\_ModifierKeys}{The modifier keys to the key pressed.}
        \attribute{public string ModifierKeysName}{The name of the key on the keyboard that is pressed.}
        \attribute{public char MappedCharacter\_Unicode}{The actual user input according to the input locale.}
    \end{attributes}
\end{class}

\subsection*{Producers}

\begin{class}{KeyboardInteractEventProducer} 
	\clsdiagram[scale = 0.9]{resources/Classes/Modules/Keyboard/Producers/KeyboardInteractEventProducer.png}
    
    \clsdcl{public class KeyboardInteractEventProducer:\\\ DefaultEventQueue<KeyboardInteractEvent>}

    \clsdsp{Provides single-writer-multiple-reader queue for KeyboardInteractEvent}

    \begin{methods}
        \begin{method}{public void StartCapture(INativeKeyboard nativeK)}{Starts capturing keyboard interaction events}
        	\begin{parameters}
        		\para{INativeKeyboard nativeK}{Required for capturing KeyboardInteractEvent.}
        	\end{parameters}
        \end{method}
        \begin{method}{public void StopCapture()}{Stops capturing keyboard interaction events}
        \end{method}
    \end{methods}
\end{class}

\subsection{Mouse}

\begin{class}{MouseModule} 
    \clsdiagram[scale = 0.7]{resources/Classes/Modules/Mouse/MouseModule.png}

    \clsdcl{public class MouseModule : IModule}

    \clsdsp{The MouseModule is responsible for recording all mouse related user interactions.}
\end{class}

\subsection*{Configuration}

\begin{class}{MouseModuleConfiguration}
	\clsdiagram[scale = 0.7]{resources/Classes/Modules/Mouse/MouseModuleConfiguration.png}
	
	\clsdcl{public class MouseModuleConfiguration: IConfiguration}
	
	\clsdsp{Configuration for Mouse module}
	
	\begin{attributes}
		\attribute{public uint SamplingRateInHz}{The sampling rate of the mouse position capture, in Hz.}
		\attribute{public int Threshold}{ The minimal distance(computed with screen coordinates) a mouse move must reach in a period to be recorded.}
	\end{attributes}
\end{class}

\subsection*{Events}

\begin{absclass}{MouseEvent} 
    \clsdiagram{resources/Classes/Modules/Mouse/Events/MouseEvent.png}

    \clsdcl{public abstract class MouseEvent: Event}

    \clsdsp{A generic mouse event which all specific MouseEvents inherit from.}
    
    \begin{attributes}
    	\attribute{public System.Windows.Point MousePosition}{The current position of the mouse in screen coordinates}
    \end{attributes}
\end{absclass}

\begin{class}{MouseClickEvent} 
    \clsdiagram{resources/Classes/Modules/Mouse/Events/MouseClickEvent.png}

    \clsdcl{public class MouseClickEvent : MouseEvent}

    \clsdsp{ A mouse click user interaction.}

    \begin{attributes}
        \attribute{public System.Windows.Input.MouseAction MouseAction}{Specifies constants that define actions performed by the mouse.}
        \attribute{public string HWnd}{The handle of the window in which the mouse click occurred.}
    \end{attributes}
\end{class}

\begin{class}{MouseMoveEvent} 
    \clsdiagram{resources/Classes/Modules/Mouse/Events/MouseMoveEvent.png}

    \clsdcl{public class MouseMoveEvent : MouseEvent}

    \clsdsp{A mouse move user interaction.}
\end{class}

\begin{class}{MouseScrollEvent} 
    \clsdiagram{resources/Classes/Modules/Mouse/Events/MouseScrollEvent.png}

    \clsdcl{public class MouseScrollEvent : MouseEvent}

    \clsdsp{A mouse scroll user interaction.}

    \begin{attributes}
        \attribute{public short ScrollAmount}{The amount of the wheel being scrolled.}
        \attribute{public string HWnd}{The handle of the window in which the mouse scroll occurred.}
    \end{attributes}
\end{class}

\subsection*{Producers}

\begin{class}{MouseClickEventProducer} 
	\clsdiagram[scale = 0.7]{resources/Classes/Modules/Mouse/Producers/MouseClickEventProducer.png}
    
    \clsdcl{public class MouseClickEventProducer : DefaultEventQueue<MouseClickEvent>}

    \clsdsp{Provides single-writer-multiple-reader queue for MouseClickEvent}

    \begin{methods}
        \begin{method}{public void StartCapture()}{Starts capturing mouse click events}
        \end{method}
        \begin{method}{public void StopCapture()}{Stop capturing mouse click events}
        \end{method}
    \end{methods}
\end{class}

\begin{class}{MouseMoveEventProducer} 
	\clsdiagram[scale = 0.7]{resources/Classes/Modules/Mouse/Producers/MouseMoveEventProducer.png}
    
    \clsdcl{public class MouseMoveEventProducer : DefaultEventQueue<MouseMoveEvent>}

    \clsdsp{A single-writer-multiple-reader queue for MouseMoveEvent}

    \begin{methods}
        \begin{method}{public void StartCapture(INativeMouse nativeM)}{Starts capturing mouse move events}
        	\begin{parameters}
        		\para{INativeMouse nativeM}{Required for capturing MouseMoveEvent.}
        	\end{parameters}
        \end{method}
        \begin{method}{public void StopCapture()}{Stop capturing mouse move events}
        \end{method}
    \end{methods}
\end{class}

\begin{class}{MouseScrollEventProducer} 
	\clsdiagram[scale = 0.7]{resources/Classes/Modules/Mouse/Producers/MouseScrollEventProducer.png}
    
    \clsdcl{public class MouseScrollEventProducer : DefaultEventQueue<MouseScrollEvent>}

    \clsdsp{Provides single-writer-multiple-reader queue for MouseScrollEvent}

    \begin{methods}
        \begin{method}{public void StartCapture()}{Starts capturing mouse scroll events}
        \end{method}
        \begin{method}{public void StopCapture()}{Stop capturing mouse scroll events}
        \end{method}
    \end{methods}
\end{class}

\subsection{Webbrowser}

\begin{class}{WebBrowserModule} 
	\clsdiagram[scale = 0.7]{resources/Classes/Modules/WebBrowser/WebBrowserModule.png}
	
    \clsdcl{public class WebBrowserModule : ICollectingModule}

    \clsdsp{The WebBrowserModule is responsible for recording all browser related user interactions}

\end{class}

\subsection*{Events}

\begin{absclass}{WebBrowserEvent}
	\clsdiagram[scale = 0.7]{resources/Classes/Modules/WebBrowser/Events/WebBrowserEvent.png}

    \clsdcl{public abstract class WebBrowserEvent : Event}

    \clsdsp{A generic web browser event which all specific WebBrowserEvents inherit from}

    \begin{attributes}
        \attribute{public Guid TabID}{The identifier of the tab where the web browser event occured in}
        \attribute{public Uri CurrentURL}{The URL of the website where the web browser event occured in}
    \end{attributes}

\end{absclass}

\begin{class}{ButtonClickEvent : WebBrowserEvent} 
	\clsdiagram[scale = 0.7]{resources/Classes/Modules/WebBrowser/Events/ButtonClickEvent.png}
    
    \clsdcl{public class ButtonClickEvent}

    \clsdsp{A button click user interaction}

    \begin{attributes}
    	\attribute{Attributes inherited from WebbrowserEvent}
        \attribute{public string Button}{The title of the button item that was clicked on the website}
        \attribute{public Uri URL}{The URL of the website that the button is linked to}
    \end{attributes}
\end{class}

\begin{class}{CloseTabEvent} 
    \clsdiagram[scale = 0.7]{resources/Classes/Modules/WebBrowser/Events/CloseTabEvent.png}
    	
    \clsdcl{public class CloseTabEvent : WebBrowserEvent}

    \clsdsp{A close tab user interaction}

    \begin{attributes}
        \attribute{Only from WebBrowserEvent inherited attributes}
    \end{attributes}
\end{class}

\begin{class}{FileDownloadEvent} 
 	\clsdiagram[scale = 0.7]{resources/Classes/Modules/WebBrowser/Events/FileDownloadEvent.png}
 	   
    \clsdcl{public class FileDownloadEvent : WebBrowserEvent}

    \clsdsp{A file download user interaction}

    \begin{attributes}
    	\attribute{Attributes inherited from WebbrowserEvent}
        \attribute{public string MIMEType}{MIME type of the file that was downloaded}
        \attribute{public Uri FileURL}{The URL of the file that was downloaded}
    \end{attributes}
\end{class}

\begin{class}{HoverEvent} 
    \clsdiagram[scale = 0.7]{resources/Classes/Modules/WebBrowser/Events/HoverEvent.png}
    	
    \clsdcl{public class HoverEvent : WebBrowserEvent}

    \clsdsp{A hover user interaction}

    \begin{attributes}
    	\attribute{Attributes inherited from WebbrowserEvent}
        \attribute{public string HoverElement}{The element on the website that has been hovered}
    \end{attributes}
\end{class}

\begin{class}{NavigationEvent} 
    \clsdiagram[scale = 0.7]{resources/Classes/Modules/WebBrowser/Events/NavigationEvent.png}
    
    \clsdcl{public class NavigationEvent : WebBrowserEvent}

    \clsdsp{A navigation user interaction}

    \begin{attributes}
    	\attribute{Only from WebbrowserEvent inherited attributes}
    \end{attributes}
\end{class}

\begin{class}{OpenTabEvent} 
    \clsdiagram[scale = 0.7]{resources/Classes/Modules/WebBrowser/Events/OpenTabEvent.png}
    
    \clsdcl{public class OpenTabEvent : WebBrowserEvent}

    \clsdsp{An open tab user interaction}

    \begin{attributes}
    	\attribute{Only from WebbrowserEvent inherited attributes}
    \end{attributes}
\end{class}

\begin{class}{SwitchTabEvent} 
    \clsdiagram[scale = 0.7]{resources/Classes/Modules/WebBrowser/Events/SwitchTabEvent.png}
    
    \clsdcl{public class SwitchTabEvent : WebBrowserEvent}

    \clsdsp{A switch tab user interaction}

    \begin{attributes}
    	\attribute{Attributes inherited from WebbrowserEvent}
        \attribute{public Guid NewTabID}{The identifier of the tab that the user switched to}
    \end{attributes}
\end{class}

\begin{class}{TextInputEvent} 
    \clsdiagram[scale = 0.7]{resources/Classes/Modules/WebBrowser/Events/TextInputEvent.png}
    
    \clsdcl{public class TextInputEvent : WebBrowserEvent}

    \clsdsp{A text input user interaction}

    \begin{attributes}
    	\attribute{Attributes inherited from WebbrowserEvent}
        \attribute{public string InputtedText}{The text that was inputted by the user on the website}
        \attribute{public string Textbox}{The textbox where the text was inputted in}
    \end{attributes}
\end{class}

\begin{class}{TextSelectionEvent} 
    \clsdiagram[scale = 0.7]{resources/Classes/Modules/WebBrowser/Events/TextSelectionEvent.png}
    
    \clsdcl{public class TextSelectionEvent : WebBrowserEvent}

    \clsdsp{A text selection user interaction}

    \begin{attributes}
    	\attribute{Attributes inherited from WebbrowserEvent}
        \attribute{public string SelectedText}{The text that was selected on the website}
    \end{attributes}
\end{class}

\subsection*{Producers}

\begin{class}{ButtonClickEventProducer} 
	\clsdiagram[scale = 0.7]{resources/Classes/Modules/WebBrowser/Producers/ButtonClickEventProducer.png}
    
    \clsdcl{public class ButtonClickEventProducer}

    \clsdsp{A single-writer-multiple-reader queue for ButtonClickEvent}

    \begin{methods}
        \begin{method}{public override IAsyncEnumerable<ButtonClickEvent> GetEvents()}{Asynchronously gets all button click events}
          \return{IAsyncEnumerable<ButtonClickEvent>}{A stream of ButtonClickEvent}
        \end{method}
        \begin{method}{ protected override void Enqueue(ButtonClickEvent event)}{Asynchronously enqueues a new button click event}
            \begin{parameters}
                \para{ButtonClickEvent event}{ButtonClickEvent event}
            \end{parameters}
        \end{method}
    \end{methods}
\end{class}

\begin{class}{CloseTabEventProducer} 
	\clsdiagram[scale = 0.7]{resources/Classes/Modules/WebBrowser/Producers/CloseTabEventProducer.png}
    
    \clsdcl{public class CloseTabEventProducer}

    \clsdsp{A single-writer-multiple-reader queue for CloseTabEvent}

    \begin{methods}
        \begin{method}{public override IAsyncEnumerable<CloseTabEvent> GetEvents()}{Asynchronously gets all close tab events}
          \return{IAsyncEnumerable<CloseTabEvent>}{A stream of CloseTabEvent}
        \end{method}
        \begin{method}{ protected override void Enqueue(CloseTabEvent event)}{Asynchronously enqueues a new close tab event}
            \begin{parameters}
                \para{CloseTabEvent event}{CloseTabEvent event}
            \end{parameters}
        \end{method}
    \end{methods}
\end{class}

\begin{class}{FileDownloadEventProducer} 
	\clsdiagram[scale = 0.7]{resources/Classes/Modules/WebBrowser/Producers/FileDownloadEventProducer.png}
    
    \clsdcl{public class FileDownloadEventProducer}

    \clsdsp{A single-writer-multiple-reader queue for FileDownloadEvent}

    \begin{methods}
        \begin{method}{public override IAsyncEnumerable<FileDownloadEvent> GetEvents()}{Asynchronously gets all file download events}
          \return{IAsyncEnumerable<FileDownloadEvent>}{A stream of FileDownloadEvent}
        \end{method}
        \begin{method}{ protected override void Enqueue(FileDownloadEvent event)}{Asynchronously enqueues a new file download event}
            \begin{parameters}
                \para{FileDownloadEvent event}{FileDownloadEvent event}
            \end{parameters}
        \end{method}
    \end{methods}
\end{class}

\begin{class}{HoverEventProducer} 
	\clsdiagram[scale = 0.7]{resources/Classes/Modules/WebBrowser/Producers/HoverEventProducer.png}
    
    \clsdcl{public class HoverEventProducer}

    \clsdsp{A single-writer-multiple-reader queue for HoverEvent}

    \begin{methods}
        \begin{method}{public override IAsyncEnumerable<HoverEvent> GetEvents()}{Asynchronously gets all hover events}
          \return{IAsyncEnumerable<HoverEvent>}{A stream of HoverEvent}
        \end{method}
        \begin{method}{ protected override void Enqueue(HoverEvent event)}{Asynchronously enqueues a new hover event}
            \begin{parameters}
                \para{HoverEvent event}{HoverEvent event}
            \end{parameters}
        \end{method}
    \end{methods}
\end{class}

\begin{class}{NavigationEventProducer} 
	\clsdiagram[scale = 0.7]{resources/Classes/Modules/WebBrowser/Producers/NavigationEventProducer.png}
    
    \clsdcl{public class NavigationEventProducer}

    \clsdsp{A single-writer-multiple-reader queue for NavigationEvent}

    \begin{methods}
        \begin{method}{public override IAsyncEnumerable<NavigationEvent> GetEvents()}{Asynchronously gets all navigation events}
          \return{IAsyncEnumerable<NavigationEvent>}{A stream of NavigationEvent}
        \end{method}
        \begin{method}{ protected override void Enqueue(NavigationEvent event)}{Asynchronously enqueues a new navigation event}
            \begin{parameters}
                \para{NavigationEvent event}{NavigationEvent event}
            \end{parameters}
        \end{method}
    \end{methods}
\end{class}

\begin{class}{OpenTabEventProducer} 
	\clsdiagram[scale = 0.7]{resources/Classes/Modules/WebBrowser/Producers/OpenTabEventProducer.png}
    
    \clsdcl{public class OpenTabEventProducer}

    \clsdsp{A single-writer-multiple-reader queue for OpenTabEvent}

    \begin{methods}
        \begin{method}{public override IAsyncEnumerable<OpenTabEvent> GetEvents()}{Asynchronously gets all open tab events}
          \return{IAsyncEnumerable<OpenTabEvent>}{A stream of OpenTabEvent}
        \end{method}
        \begin{method}{ protected override void Enqueue(OpenTabEvent event)}{Asynchronously enqueues a new open tab event}
            \begin{parameters}
                \para{OpenTabEvent event}{OpenTabEvent event}
            \end{parameters}
        \end{method}
    \end{methods}
\end{class}

\begin{class}{SwitchTabEventProducer} 
	\clsdiagram[scale = 0.7]{resources/Classes/Modules/WebBrowser/Producers/SwitchTabEventProducer.png}
    
    \clsdcl{public class SwitchTabEventProducer}

    \clsdsp{A single-writer-multiple-reader queue for SwitchTabEvent}

    \begin{methods}
        \begin{method}{public override IAsyncEnumerable<SwitchTabEvent> GetEvents()}{Asynchronously gets all switch tab events}
          \return{IAsyncEnumerable<SwitchTabEvent>}{A stream of SwitchTabEvent}
        \end{method}
        \begin{method}{ protected override void Enqueue(SwitchTabEvent event)}{Asynchronously enqueues a new switch tab event}
            \begin{parameters}
                \para{SwitchTabEvent event}{SwitchTabEvent event}
            \end{parameters}
        \end{method}
    \end{methods}
\end{class}

\begin{class}{TextInputEventProducer} 
	\clsdiagram[scale = 0.7]{resources/Classes/Modules/WebBrowser/Producers/TextInputEventProducer.png}
    
    \clsdcl{public class TextInputEventProducer}

    \clsdsp{A single-writer-multiple-reader queue for TextInputEvent}

    \begin{methods}
        \begin{method}{public override IAsyncEnumerable<TextInputEvent> GetEvents()}{Asynchronously gets all text input events}
          \return{IAsyncEnumerable<TextInputEvent>}{A stream of TextInputEvent}
        \end{method}
        \begin{method}{ protected override void Enqueue(TextInputEvent event)}{Asynchronously enqueues a new text input event}
            \begin{parameters}
                \para{TextInputEvent event}{TextInputEvent event}
            \end{parameters}
        \end{method}
    \end{methods}
\end{class}

\begin{class}{TextSelectionEventProducer} 
	\clsdiagram[scale = 0.7]{resources/Classes/Modules/WebBrowser/Producers/TextSelectionEventProducer.png}
    
    \clsdcl{public class TextSelectionEventProducer}

    \clsdsp{A single-writer-multiple-reader queue for TextSelectionEvent}

    \begin{methods}
        \begin{method}{public override IAsyncEnumerable<TextSelectionEvent> GetEvents()}{Asynchronously gets all text selection events}
          \return{IAsyncEnumerable<TextSelectionEvent>}{A stream of TextSelectionEvent}
        \end{method}
        \begin{method}{ protected override void Enqueue(TextSelectionEvent event)}{Asynchronously enqueues a new text selection event}
            \begin{parameters}
                \para{TextSelectionEvent event}{TextSelectionEvent event}
            \end{parameters}
        \end{method}
    \end{methods}
\end{class}


\subsection{WindowManagement}

\begin{class}{WindowManagementModule} 
    \clsdiagram[scale = 0.7]{resources/Classes/Modules/WindowManagement/WindowManagementModule.png}

    \clsdcl{public class WindowManagementModule : IModule}

    \clsdsp{The WindowManagementModule is responsible for recording all window related user interactions.}
\end{class}

\subsection*{Events}

\begin{absclass}{WindowEvent} 
    \clsdiagram{resources/Classes/Modules/WindowManagement/Events/WindowEvent.png}

    \clsdcl{public abstract class WindowEvent: Event}

    \clsdsp{A window management event which all specific WindowEvents inherit from.}

    \begin{attributes}
        \attribute{public string Title}{The title of the interacted window.}
        \attribute{public string ProcessName}{The name of the process associated with the window}
    \end{attributes}
\end{absclass}

\begin{class}{WindowFocusEvent} 
    \clsdiagram{resources/Classes/Modules/WindowManagement/Events/WindowFocusEvent.png}

    \clsdcl{public class WindowFocusEvent : WindowEvent}

    \clsdsp{A window focus user interaction.}
\end{class}

\begin{class}{WindowMovementEvent} 
    \clsdiagram{resources/Classes/Modules/WindowManagement/Events/WindowMovementEvent.png}

    \clsdcl{public class WindowMovementEvent : WindowEvent}

    \clsdsp{ A window movement user interaction.}

    \begin{attributes}
        \attribute{public System.Windows.Point OldLocation}{The old location of the window.}
        \attribute{public System.Windows.Point NewLocation}{The new location of the window.}
    \end{attributes}
\end{class}

\begin{class}{WindowResizingEvent} 
    \clsdiagram{resources/Classes/Modules/WindowManagement/Events/WindowResizingEvent.png}

    \clsdcl{public class WindowResizingEvent : WindowEvent}

    \clsdsp{A window resizing user interaction.}

    \begin{attributes}
        \attribute{public System.Drawing.Size OldSize}{The old size of the window.}
        \attribute{public System.Drawing.Size NewSize}{The new size of the window.}
    \end{attributes}
\end{class}

\begin{class}{WindowStateChangedEvent} 
    \clsdiagram{resources/Classes/Modules/WindowManagement/Events/WindowStateChangedEvent.png}

    \clsdcl{public class WindowStateChangedEvent : WindowEvent}

    \clsdsp{An user interaction that changes the state of a window.}

    \begin{attributes}
        \attribute{public System.Windows.WindowState State}{The new state of the window.}
    \end{attributes}
\end{class}

\subsection*{Producers}

\begin{class}{WindowFocusEventProducer} 
	\clsdiagram[scale = 0.7]{resources/Classes/Modules/WindowManagement/Producers/WindowFocusEventProducer.png}
    
    \clsdcl{public class WindowFocusEventProducer : DefaultEventQueue<WindowFocusEvent>}

    \clsdsp{A single-writer-multiple-reader queue for WindowFocusEvent}

    \begin{methods}
        \begin{method}{public void StartCapture(INativeWindowManagement nativeWinManagement)}{Starts capturing window focus events}
        	\begin{parameters}
    			\para{INativeWindowManagement nativeWinManagement}{Required for capturing WindowFocusEvent.}
    		\end{parameters}
        \end{method}
        \begin{method}{public void StopCapture()}{Stop capturing window focus events}
    	\end{method}
    \end{methods}
\end{class}

\begin{class}{WindowMovementEventProducer} 
	\clsdiagram[scale = 0.7]{resources/Classes/Modules/WindowManagement/Producers/WindowMovementEventProducer.png}
    
    \clsdcl{public class WindowMovementEventProducer : DefaultEventQueue<WindowMovementEvent>}

    \clsdsp{A single-writer-multiple-reader queue for WindowMovementEvent}

    \begin{methods}
    	\begin{method}{public void StartCapture(INativeWindowManagement nativeWinManagement)}{Starts capturing window movement events}
    		\begin{parameters}
    			\para{INativeWindowManagement nativeWinManagement}{Required for capturing WindowMovementEvent.}
    		\end{parameters}
        \end{method}
        \begin{method}{public void StopCapture()}{Stop capturing window movement events}
    	\end{method}
    \end{methods}
\end{class}

\begin{class}{WindowResizingEventProducer} 
	\clsdiagram[scale = 0.7]{resources/Classes/Modules/WindowManagement/Producers/WindowResizingEventProducer.png}
    
    \clsdcl{public class WindowResizingEventProducer : DefaultEventQueue<WindowResizingEvent>>}

    \clsdsp{A single-writer-multiple-reader queue for WindowResizingEvent}

    \begin{methods}
        \begin{method}{public void StartCapture(INativeWindowManagement nativeWinManagement)}{Starts capturing window resizing events}
        	\begin{parameters}
    			\para{INativeWindowManagement nativeWinManagement}{Required for capturing WindowResizingEvent.}
    		\end{parameters}
        \end{method}
        \begin{method}{public void StopCapture()}{Stop capturing window resizing events}
    	\end{method}
    \end{methods}
\end{class}

\begin{class}{WindowStateChangedEventProducer} 
	\clsdiagram[scale = 0.7]{resources/Classes/Modules/WindowManagement/Producers/WindowStateChangedEventProducer.png}
    
    \clsdcl{public class WindowStateChangedEventProducer:\\\ DefaultEventQueue<WindowStateChangedEvent>}

    \clsdsp{A single-writer-multiple-reader queue for WindowStateChangedEvent}

    \begin{methods}
        \begin{method}{public void StartCapture(INativeWindowManagement nativeWinManagement)}{Starts capturing window state changed events}
        	\begin{parameters}
    			\para{INativeWindowManagement nativeWinManagement}{Required for capturing WindowStateChangedEvent.}
    		\end{parameters}
        \end{method}
        \begin{method}{public void StopCapture()}{Stop capturing window state changed events}
    	\end{method}
    \end{methods}
\end{class}

\newpage
\section{Common}
\subsection{Shared}
\issue{
Distinction between \class{ICollectingModule}, \class{ITransformingModule} and \class{IReceivingModule} has proven unnecessary.
}{
Do not differentiate between different modules and suggest use of general \class{IModule}.
}{
Remove interfaces: \class{ICollectingModule}, \class{ITransformingModule} and \class{IReceivingModule}.
}

\issue{
Design of different event queue strategies: \class{RingBufferStorageStrategy}, \class{RefCountedListStorageStrategy}, \class{KeepAllStorageStrategy} were not suitable for our implementation approach and use of the C\# .NET Core language library.
}{
Design new event queue storage strategies for different types of usage: \keyword{Bounded} or \keyword{Unbounded} and \keyword{SingleConsumer} or \keyword{MultiConsumer}.
}{
Add abstract storage strategies: \class{SingleConsumerChannelStrategy} and \class{MutliConsumerChannelStrategy}.\\
A \keyword{SingleConsumer} strategy allows only one consumer at a time to read events out of the queue which provides a performance boost in comparison to \keyword{MultiConsumer}.
A \keyword{MultiConsumer} strategy allows a finite and infinite amount of consumers to read simultanously. This provides multiplication of references to a specific \class{Event} and supports reference counting for each consumer.

Both strategy types have been additionally divided into \keyword{Bounded}, which acts as a RingBuffer and removes the oldest events if the queue is full, and \keyword{Unbounded}, which does not restrict the event count of the queue.
}

\issue{
Unable to use designed event queue structure in \keyword{MEF} correctly.
}{
Add new abstraction of event queues to support distinct imports in MEF and simplify usage for Modules.
}{
Add interfaces concerning distinct sections of the event pipeline and read-write access:\\
\class{IReadOnlyEventQueue}: To restrict usage to read only methods.\\
\class{ISupportDeserializationEventQueue}: Allows a strongly typed usage of events in the deserialization process that stores events serialized, which is necessary for any recording or processing session.\\
\class{IDecodableEventQueue}: Defines an strongly typed read only event queue which is used in the decoding process of the pipeline.\\
\class{IEncodableEventQueue}: Defines an strongly typed read only event queue which is used in the encoding process of the pipeline.\\

Additionally multiple default implementation using their correct storage strategies were introduced to simplify usage of queues in a Module.
}

\issue{
Unable to gather events from unowned processes and windows.
}{
Support our additional \package{HookLibrary} for use by our modules.
}{
To accomplish this and reduce the amount of multiple hooks per windows process, we had to add \class{GlobalHook} and \class{HookNativeMethods} to our shared project.
}

\issue{
Handling of directory- or filepaths using \class{string} has proven error-prone and not typesafe.
}{
Add shared support for directory- and filepaths using new classes.
}{
Add \class{DirectoryPath} and \class{FilePath} classes.
}

\issue{
Handling of to be serialized configurations using \class{string} has proven error-prone and not typesafe.
}{
Add shared support for strongly typed raw configurations.
}{
Add \class{RawConfiguration} as a wrapper to the unserialized configuration as a string at \member{RawConfiguration.RawValue}.
}

\newpage
\section{BrowserExtension}
\subsection{Shared}
\begin{interface}{IEvent}

\clsdcl{interface IEvent}

\clsdsp{Defines the same interface to use for recorded events as present in the MORR application.}

\begin{attributes}
\attribute{public timeStamp : Date}{The time at which the event was created.}
\attribute{public issuingModule : number}{The ID of the browser module. Is a fixed value in the context of the browser extension.}
\attribute{public type : EventType}{The type of the event.}
\end{attributes}
\begin{methods}
\begin{method}{public serialize() : string}{Serialize the event to a JSON string.}
\return{string}{The JSON string encoding the event.}
\end{method}
\end{methods}
\end{interface}

\begin{class}{BrowserEvent}

\clsdcl{class BrowserEvent implements IEvent}

\clsdsp{Implements the interface IEvent and extends it by browser specific, generic event data.}

\begin{attributes}
\attribute{public timeStamp : Date}{The time at which the event was created.}
\attribute{public issuingModule : number}{The ID of the browser module. Is a fixed value in the context of the browser extension.}
\attribute{public type : EventType}{The type of the event.}
\attribute{public windowID : number}{The ID of the browser window the event occured in. This does not need to be serialized in accordance with the functional specifications, as this information can be easily gathered it is stored for internal purposes mostly.}
\attribute{public tabID : number}{The ID of the tab the event occured in.}
\attribute{public url : URL}{The URL that was opened in the tab with ID tabID as the event occured.}
\end{attributes}
\begin{constructors}
\begin{constructor}{constructor(type : EventType, tabID : number, windowID: number, url : string)}{Create a new generic browser event.}
\begin{parameters}
\para{type : EventType}{The type to set for the event.}
\para{tabID : number}{The tabID to set for the event.}
\para{windowID : number}{The windowID to set for the event.}
\para{url : string}{The url to set for the event.}
\end{parameters}
\end{constructor}
\end{constructors}
\begin{methods}
\begin{method}{public serialize() : string}{Serialize the event to a JSON string.}
\return{string}{The JSON string encoding the event.}
\end{method}
\end{methods}
\end{class}

\begin{interface}{IListener}

\clsdcl{interface IListener}

\clsdsp{A listener is a class responsible for recording certain browser-specific events and sending them to the BackGroundScript.}

\begin{constructors}
\begin{constructor}{constructor(callback: (event : BrowserEvent) => void)}{Create a new listener.}
\begin{parameters}
\para{callback : BrowserEvent) => void}{The function to invoke on created events in order to send them back to the BackGroundScript.}
\end{parameters}
\end{constructor}
\end{constructors}
\begin{methods}
\begin{method}{public start() : void}{Start the listener. The listener will generate events and invoke the callback method on them until it is stopped.}
\end{method}
\begin{method}{public stop() : void}{Stop the listener.}
\end{method}
\end{methods}
\end{interface}

\subsection{Background}
\begin{class}{BackGroundScript}
\clsdcl{class BackGroundScript}

\clsdsp{The BackGroundScript class serves as the main class of the browser extension. It is responsible for creating and starting/stopping the ListenerManager and the CommunicationStrategy.}

\begin{constructors}
\begin{constructor}{public BackGroundScript()}{Initialize the BackGroundScript and thus the browser extension. After this function is completed, the browser extension is ready and awaits a signal from the MORR application.}
\end{constructor}
\end{constructors}
\begin{methods}
\begin{method}{public start() : void}{Start all components necessary for a session recording. To be called when a recording starts.}
\end{method}
\begin{method}{public stop() : void}{Stop all components which should only be active during a session recording. To be called when a recording stops.}
\end{method}
\begin{method}{public callback(BrowserEvent event) : void}{Callback to be handed over to all components which create events during a recording. The BackGroundScript class is responsible to forward these events to the ICommunicationStrategy for transmission to the MORR application.}
\begin{parameters}
\para{event : BrowserEvent}{The event to be forwarded to the MORR application.}
\end{parameters}
\end{method}
\end{methods}
\end{class}

\begin{class}{ListenerManager}
\clsdcl{class ListenerManager}

\clsdsp{The ListenerManager is responsible to create all IListeners and keep references to them. The ListenerManager provides methods to start/stop all attached listeners.}

\begin{constructors}
\begin{constructor}{public ListenerManager(configurationString : string)}{Initialize the ListenerManager and therefore all configured listeners.}
\begin{parameters}
\para{configurationString : string}{A valid JSON string containing optional configuration. The configuration determines whether specific event types shall be recorded or not.}
\end{parameters}
\end{constructor}
\end{constructors}
\begin{methods}
\begin{method}{public startAll() : void}{Start all listeners. To be called when a recording starts.}
\end{method}
\begin{method}{public stopAll() : void}{Stop all listeners. To be called when a recording stops.}
\end{method}
\end{methods}
\end{class}

\begin{interface}{ICommunicationStrategy}

\clsdcl{interface ICommunicationStrategy}

\clsdsp{A conrete implementation of ICommunicationStrategy provides means to communicate with the MORR application. This includes sending the generated BrowserEvents from the browser extension to the MORR application where they will be processed.}

\begin{methods}
\begin{method}{establishConnection(onSuccess : (response? : string) => void, onFail : (response? : string) => void) : void}{Asynchronously try to establish a connection to the MORR application.}
\begin{parameters}
\para{onSuccess : (response? : string) => void}{Callback. To be called when a connection has been established. Additional information may be passed as string by providing a value for response.}
\para{onFail : (response? : string) => void}{Callback. To be called when a connection could not be successfully established. Additional information may be passed as string by providing a value for response.}
\end{parameters}
\end{method}
\begin{method}{requestConfig(onSuccess : (response? : string) => void, onFail : (response? : string) => void) : void}{Request a configuration string from the MORR application.}
\begin{parameters}
\para{onSuccess : (response? : string) => void}{Callback. To be called when the MORR application positively replied to the request. In this case the configuration JSON string should be passed as the response parameter.}
\para{onFail : (response? : string) => void}{Callback. To be called when the configuration could not be successfully requested and received. Additional information may be passed as string by providing a value for response.}
\end{parameters}
\end{method}
\begin{method}{waitForStart(onStart : (response? : string) => void, onFail : (response? : string) => void) : void}{Await a start signal from the MORR application.}
\begin{parameters}
\para{onStart : (response? : string) => void}{Callback. To be called when a start signal was received. Additional information may be passed as string by providing a value for response.}
\para{onFail : (response? : string) => void}{Callback. To be called when an unexpected response was received or the connection terminated. Additional information may be passed as string by providing a value for response.}
\end{parameters}
\end{method}
\begin{method}{sendData(data : string, onSuccess : (response? : string) => void, onFail : (response? : string) => void) : void}{Send event data to the MORR application.}
\begin{parameters}
\para{data : string}{The serialized event data to send.}
\para{onSuccess : (response? : string) => void}{Callback. To be called when the data was sent successfully. Additional information may be passed as string by providing a value for response.}
\para{onFail : (response? : string) => void}{Callback. To be called when the data could not successfully be sent. Additional information may be passed as string by providing a value for response.}
\end{parameters}
\end{method}
\end{methods}
\end{interface}

\begin{class}{PostHTTPInterface}

\clsdcl{class PostHTTPInterface implements ICommunicationStrategy}

\clsdsp{Implements ICommunicationStrategy by sending HTTP POST-Requests to the MORR application.}

\begin{constructors}
\begin{constructor}{public PostHTTPInterface(url : string)}{Create a new PostHTTPInterface.}
\begin{parameters}
\para{url : string}{A string containg an URL with port number to send the HTTP requests to.}
\end{parameters}
\end{constructor}
\end{constructors}
\begin{methods}
\begin{method}{establishConnection(onSuccess : (response? : string) => void, onFail : (response? : string) => void) : void}{Asynchronously try to establish a connection to the MORR application.}
\begin{parameters}
\para{onSuccess : (response? : string) => void}{Callback. To be called when a connection has been established. Additional information may be passed as string by providing a value for response.}
\para{onFail : (response? : string) => void}{Callback. To be called when a connection could not be successfully established. Additional information may be passed as string by providing a value for response.}
\end{parameters}
\end{method}
\begin{method}{requestConfig(onSuccess : (response? : string) => void, onFail : (response? : string) => void) : void}{Request a configuration string from the MORR application.}
\begin{parameters}
\para{onSuccess : (response? : string) => void}{Callback. To be called when the MORR application positively replied to the request. In this case the configuration JSON string should be passed as the response parameter.}
\para{onFail : (response? : string) => void}{Callback. To be called when the configuration could not be successfully requested and received. Additional information may be passed as string by providing a value for response.}
\end{parameters}
\end{method}
\begin{method}{waitForStart(onStart : (response? : string) => void, onFail : (response? : string) => void) : void}{Await a start signal from the MORR application.}
\begin{parameters}
\para{onStart : (response? : string) => void}{Callback. To be called when a start signal was received. Additional information may be passed as string by providing a value for response.}
\para{onFail : (response? : string) => void}{Callback. To be called when an unexpected response was received or the connection terminated. Additional information may be passed as string by providing a value for response.}
\end{parameters}
\end{method}
\begin{method}{sendData(data : string, onSuccess : (response? : string) => void, onFail : (response? : string) => void) : void}{Send event data to the MORR application.}
\begin{parameters}
\para{data : string}{The serialized event data to send.}
\para{onSuccess : (response? : string) => void}{Callback. To be called when the data was sent successfully. Additional information may be passed as string by providing a value for response.}
\para{onFail : (response? : string) => void}{Callback. To be called when the data could not successfully be sent. Additional information may be passed as string by providing a value for response.}
\end{parameters}
\end{method}
\end{methods}
\end{class}
