\chapter{Classes}
\label{ch:class}
%%%%%%%%%%%%%%%%%%%%%%%%%%%%%%%%%% Notes about this chapter%%%%%%%%%%%%%%%%%%%%%%%%%%%%%%%%%%%%%
% Each namespace correspondes to a section
% Each project in a namespace correspondes to a subsection in its namespaces' section.
% Each class, abstract class, enumeration or interface in the project corresponds to a subsubsection in its project's subsection.
% Each class, abstract class, enumeration or interface has some of the followings: class diagram, description, declaration, constructor, attributes and Methods.
%
% namespace------section
%     |            |
%  project-----subsection
%     ||           |
%   class-------subsubsection
%
%  IMPORTANT: the structure above should corresponds to the directory structure chapter. \autoref{ch:dirstructure}
%
% At the start of each namespace \section{namespace name} should be called to start a new section
% At the start of each project \subsection{project name} should be called to start a new subsection
% At the start of each class, abstract class, enumeration or interface, a corresponding environment \begin{class},... will be called to start a new subsubsection
% At the start of class diagram, description, declaration, constructor, fields and Methods,
% a corresponding commad \cls----- or environment \begin{methods}... will be called to start a new paragraph.
%
%%%%%%%%%%%%%%%%%%%%%%%%%%%%%%%%%% Notes about this chapter%%%%%%%%%%%%%%%%%%%%%%%%%%%%%%%%%%%%%

%T he description
% para1: the description in text
\newcommand{\dsp}{\textbf{Description: }}

\newenvironment{class}[1]{\subsubsection{{\textbf{Class: #1}}}}{\newpage}
\newenvironment{absclass}[1]{\subsubsection{{\textbf{Abstract Class: #1}}}}{\newpage}
\newenvironment{interface}[1]{\subsubsection{{\textbf{Interface: #1}}}}{\newpage}
\newenvironment{enum}[1]{\subsubsection{{\textbf{Enumeration: #1}}}}{\newpage}
 
% Enum elements enumeration
\newenvironment{elements}{\textbf{Elements:}  \begin{enumerate}
}{\end{enumerate}}
% Eeum element
% para1: declaration para2: description
\newcommand{\element}[2]{\item \texttt{#1} \\ \dsp #2}

%include a class diagram
% optional para1: the optional argument of \includegraphics
% para2: the directory of the diagram
\newcommand{\clsdiagram}[2][]{\textbf{Class diagram:} \newline
\begin{center}
    \includegraphics[#1]{#2}
\end{center}}

% The declaration of a class, abstrat class or interface.
% para1: declaration in text
\newcommand{\clsdcl}[1]{\textbf{Declaration:} \newline \texttt{#1}} %para1: description 

% The description of a class, abstrat class or interface.
% para1: description in text
\newcommand{\clsdsp}[1]{\textbf{\dsp} \newline #1}

% Constructors enumeration with \constructor (constructor item)
\newenvironment{constructors}{\textbf{Constructors:}  \begin{enumerate}
}{\end{enumerate}}
% constructor
% para1: declaration para2: description
\newenvironment{constructor}[2]{\item \texttt{#1} \\ \dsp #2 \\}{}

% Attributes enumeration with attribute(attribute item)
\newenvironment{attributes}{\textbf{Attributes:}  \begin{enumerate}
}{\end{enumerate}}
% attribute
% para1: declaration para2: description
\newcommand{\attribute}[2]{\item \texttt{#1} \\ \dsp #2}

% Methods enumeration with method(method item)
\newenvironment{methods}{\textbf{Methods:}  \begin{enumerate}
}{\end{enumerate}}
% method
% para1: declaration para2: description
\newenvironment{method}[2]{\item \texttt{#1} \\ \dsp #2 \\}{}

% Parameters enumeration with \para (parameter item)
\newenvironment{parameters}{\textbf{Parameters:}  \begin{enumerate}
}{\end{enumerate}}
% parameter
% para1: declaration para2: description
\newcommand{\para}[2]{\item \texttt{#1} \\ \dsp #2}

% Exceptions enumeration with \excp (exception item)
\newenvironment{exceptions}{\textbf{Exceptions:} \begin{enumerate}
}{\end{enumerate}}
% exception
% para1: exception type para2: condition for throwing
\newcommand{\excp}[2]{\item \texttt{#1} \\ \dsp #2}

% return value
% para1: declaration para2: description
\newcommand{\return}[2]{\textbf{Return values:} \begin{itemize} \item \texttt{#1} \\ Description: #2 \end{itemize}}

\section{DummyNamespace}
%%%%%%%%%%%%%%%%%%%%%%%%%%%%%%%% EXAMPLE %%%%%%%%%%%%%%%%%%%%%%%%%%%%%%%
\subsection{DummyProject}
\begin{absclass}{Dummy} 
    \clsdiagram[1.2]{resources/Classes/DummyNamespace/dummy.png}

    \clsdcl{public abstract class Dummy}

    \clsdsp{This is a dummy class}

    \begin{attributes}
        \attribute{private String data}{The data of the dummy}
        \attribute{public Integer age}{The age of the dummy}
    \end{attributes}

    \begin{constructors}
        \begin{constructor}{public Dummy(Integer m)}{Initialize a dummy with a integer}
            \begin{parameters}
                \para{Integer m}{A integer m}
            \end{parameters}
        \end{constructor}

        \begin{constructor}{public Dummy(Integer m, List t)}{Initialize a dummy with a integer and a list}
            \begin{parameters}
                \para{Integer m}{A integer m}
                \para{List t}{A list t}
            \end{parameters}
        \end{constructor}
    \end{constructors}


    \begin{methods}
        \begin{method}{public void reproduce(Double x, String y)}{Reproduce}
            \begin{parameters}
               \para{Double x}{A double x}
                \para{String y}{A string y}
            \end{parameters}
        \end{method}
        \begin{method}{protected Float eat(Integer a, Integer b, Vektor2 c)}{Eat}
            \return{Float}{A floating number}
            \begin{parameters}
                \para{Integer a}{Integer a}
                \para{Integer b}{Integer b}
                \para{Vektor2 c}{A vektor c}
            \end{parameters}
        \end{method}
    \end{methods}
\end{absclass}

\begin{enum}{InteractionType}
    \clsdcl{public enum InteractionType}

    \clsdsp{An enum}

    \begin{elements}
        \element{Clear}{clear}
        \element{Copy}{copy}
        \element{Cut}{cut}
        \element{Paste}{paste}
    \end{elements}
\end{enum}
%%%%%%%%%%%%%%%%%%%%%%%%%%%%%%%% EXAMPLE %%%%%%%%%%%%%%%%%%%%%%%%%%%%%%%


\newpage
\section{Core}
\subsection{CLI}
\input{chapter/Classes/Core/CLI/Classes/Program}

\subsection{MORR}

\subsection*{MORR.Core}

\begin{interface}{IBootstrapper}
    \clsdiagram{resources/Classes/Core/MORR/IBootstrapper.png}

    \clsdcl{public interface IBootstrapper}

    \clsdsp{Responsible for bootstrapping the application and providing compositional facilities.}

    \begin{methods}
        \begin{method}{void ComposeImports(object @object)}{Composes the provided object.}
            \begin{parameters}
                \para{object @object}{The object to satisfy the imports on.}
            \end{parameters}
        \end{method}
    \end{methods}
\end{interface}

\begin{class}{Bootstrapper}
    \clsdiagram{resources/Classes/Core/MORR/Bootstrapper.png}
    
    \clsdcl{public class Bootstrapper : IBootstrapper}

    \clsdsp{Bootstraps the application using MEF.}

    \begin{methods}
        \begin{method}{public Bootstrapper()}{Creates a new instance of the Bootstrapper and loads all .MORR-Module.dll assemblies from the Modules subdirectory relative to the directory of the current executing assembly.}
        \end{method}
        \begin{method}{public void ComposeImports(object @object)}{Satisfies the imports on the provided object.}
            \begin{parameters}
                \para{object @object}{The object to satisfy the imports on.}
            \end{parameters}
        \end{method}
    \end{methods}
\end{class}

\begin{class}{BootstrapperConventions}
    \clsdiagram{resources/Classes/Core/MORR/BootstrapperConventions.png}

    \clsdcl{public static class BootstrapperConventions}

    \clsdsp{Provides conventions for composing objects with MEF.}

    \begin{methods}
        \begin{method}{public static RegistrationBuilder GetRegistrationBuilder()}{Gets a registration builder that contains all composition conventions.}
            \return{RegistrationBuilder}{A RegistrationBuilder containing the composition conventions.}
        \end{method}
    \end{methods}
\end{class}

\subsection*{MORR.Core.Configuration}

\begin{interface}{IConfigurationManager}
    \clsdiagram{resources/Classes/Core/MORR/Configuration/IConfigurationManager.png}

    \clsdcl{public interface IConfigurationManager}

    \clsdsp{Loads and manages the application's configuration.}

    \begin{methods}
        \begin{method}{void LoadConfiguration(FilePath path)}{Loads the configuration from the specified path.}
            \begin{parameters}
                \para{FilePath path}{The path to load the configuration from.}
            \end{parameters}
            \begin{exceptions}
                \excp{InvalidConfigurationException}{Exception thrown if the specified configuration is invalid.}
            \end{exceptions}
        \end{method}
    \end{methods}
\end{interface}

\begin{class}{ConfigurationManager}
    \clsdiagram{resources/Classes/Core/MORR/Configuration/ConfigurationManager.png}

    \clsdcl{public class ConfigurationManager : IConfigurationManager}

    \clsdsp{Manages the application's configuration.}

    \begin{methods}
        \begin{method}{public void LoadConfiguration(FilePath path)}{Loads the configuration from the specified path.}
            \begin{parameters}
                \para{FilePath path}{The path to load the configuration from.}
            \end{parameters}
            \begin{exceptions}
                \excp{InvalidConfigurationException}{Exception thrown if the specified configuration is invalid.}
            \end{exceptions}
        \end{method}
    \end{methods}
\end{class}

\begin{class}{InvalidConfigurationException}
    \clsdiagram{resources/Classes/Core/MORR/Configuration/InvalidConfigurationException.png}

    \clsdcl{public class InvalidConfigurationException : Exception}

    \clsdsp{An exception thrown when the configuration is invalid.}

    \begin{constructors}
        \begin{constructor}{public InvalidConfigurationException()}{Creates a new instance of an InvalidConfigurationException without any specific error message.}
        \end{constructor}
        \begin{constructor}{public InvalidConfigurationException(string message)}{Creates a new instance of an InvalidConfigurationException with the specified error message.}
            \begin{parameters}
                \para{string message}{The error message to construct the instance with.}
            \end{parameters}
        \end{constructor}
        \begin{constructor}{public InvalidConfigurationException(string message, Exception innerException)}{Creates a new instance of an InvalidConfigurationException with the specified error message and inner exception.}
            \begin{parameters}
                \para{string message}{The error message to construct the instance with.}
                \para{Exception innerException}{The inner exception to construct the instance with.}
            \end{parameters}
        \end{constructor}
    \end{constructors}
\end{class}

\subsection*{MORR.Core.Session}

\begin{interface}{ISessionManager}
    \clsdiagram{resources/Classes/Core/MORR/Session/ISessionManager.png}

    \clsdcl{public interface ISessionManager}

    \clsdsp{A manager responsible for all aspects of recording and processing.}

    \begin{attributes}
        \attribute{DirectoryPath? CurrentRecordingDirectory \{ get; \}}{The path to the directory containing the most recent recording or null if no recording has been created yet.}
        \attribute{DirectoryPath? RecordingsFolder \{ get; \}}{The path to the top-level folder containing the recording subdirectories.}
    \end{attributes}

    \begin{methods}
        \begin{method}{public void StartRecording()}{Starts a recording if no session is currently being recorded.}
            \begin{exceptions}
                \excp{AlreadyRecordingException}{Thrown if a recording is started while another recording is already running.}
            \end{exceptions}
        \end{method}
        \begin{method}{public void StopRecording()}{Stops a recording if a session is currently being recorded.}
            \begin{exceptions}
                \excp{NotRecordingException}{Thrown if a recording is stopped while no recording is currently running.}
            \end{exceptions}
        \end{method}
        \begin{method}{void Process(IEnumerable<DirectoryPath> recordings)}{Processes the specified recordings.}
            \begin{parameters}
                \para{IEnumerable<DirectoryPath> recordings}{The recordings to process.}
            \end{parameters}
        \end{method}
    \end{methods}
\end{interface}

\begin{class}{SessionManager}
    \clsdiagram[width=\textwidth]{resources/Classes/Core/MORR/Session/SessionManager.png}

    \clsdcl{public class SessionManager : ISessionManager}

    \clsdsp{A manager responsible for all aspects of recording and processing.}

    \begin{attributes}
        \attribute{DirectoryPath? CurrentRecordingDirectory \{ get; \}}{The path to the directory containt the most recent recording or null if no recording has been created yet.}
        \attribute{DirectoryPath? RecordingsFolder \{ get; \}}{The path to the top-level folder containing the recording subdirectories.}
    \end{attributes}

    \begin{constructors}
        \begin{constructor}{public SessionManager(FilePath configurationPath)}{Creates a new instance of a SessionManager with the specified configuration path and a default Bootstrapper, ConfigurationManager and ModuleManager}
            \begin{parameters}
                \para{FilePath configurationPath}{The configuration path to construct the instance with.}
            \end{parameters}
        \end{constructor}
        \begin{constructor}{public SessionManager(FilePath configurationPath, IBootstrapper bootstrapper, IConfigurationManager configurationManager, IModuleManager moduleManager)}{Creates a new instance of a SessionManager with the specified configuration path, bootstrapper, configuration manager and module manager.}
            \begin{parameters}
                \para{FilePath configurationPath}{The configuration path to construct the instance with.}
                \para{IBootstrapper bootstrapper}{The bootstrapper to construct the instance with.}
                \para{IConfigurationManager configurationManager}{The configuration manager to construct the instance with.}
                \para{IModuleManager moduleManager}{The module manager to construct the instance with.}
            \end{parameters}
        \end{constructor}
    \end{constructors}

    \begin{methods}
        \begin{method}{public void StartRecording()}{Starts a recording if no session is currently being recorded.}
            \begin{exceptions}
                \excp{AlreadyRecordingException}{Thrown if a recording is started while another recording is already running.}
            \end{exceptions}
        \end{method}
        \begin{method}{public void StopRecording()}{Stops a recording if a session is currently being recorded.}
            \begin{exceptions}
                \excp{NotRecordingException}{Thrown if a recording is stopped while no recording is currently running.}
            \end{exceptions}
        \end{method}
        \begin{method}{void Process(IEnumerable<DirectoryPath> recordings)}{Processes the specified recordings.}
            \begin{parameters}
                \para{IEnumerable<DirectoryPath> recordings}{The recordings to process.}
            \end{parameters}
        \end{method}
    \end{methods}
\end{class}

\subsection*{MORR.Core.Session.Exceptions}

\begin{class}{RecordingException}
    \clsdiagram{resources/Classes/Core/MORR/Session/Exceptions/RecordingException.png}

    \clsdcl{public class RecordingException : Exception}

    \clsdsp{A generic recording exception that all specialized recording exceptions derive from.}
\end{class}

\begin{class}{AlreadyRecordingException}
    \clsdiagram{resources/Classes/Core/MORR/Session/Exceptions/AlreadyRecordingException.png}

    \clsdcl{public class AlreadyRecordingException : RecordingException}

    \clsdsp{An exception thrown if a recording is started while another recording is already running.}
\end{class}

\begin{class}{NotRecordingException}
    \clsdiagram{resources/Classes/Core/MORR/Session/Exceptions/NotRecordingException.png}

    \clsdcl{public class NotRecordingException : RecordingException}

    \clsdsp{An exception thrown if a recording is stopped while no recording is currently running.}
\end{class}

\subsection*{MORR.Core.Modules}

\begin{interface}{IModuleManager}
    \clsdiagram{resources/Classes/Core/MORR/Modules/IModuleManager.png}

    \clsdcl{public interface IModuleManager}

    \clsdsp{Initializes and manages all modules.}

    \begin{methods}
        \begin{method}{void InitializeModules()}{Initializes all modules.}      
        \end{method}
        \begin{method}{void NotifyModulesOnSessionStart()}{Notifies all modules when a session starts.}
        \end{method}
        \begin{method}{void NotifyModulesOnSessionStop()}{Notifies all modules when a session stops.}
        \end{method}
    \end{methods}
\end{interface}

\begin{class}{ModuleManager}
    \clsdiagram{resources/Classes/Core/MORR/Modules/ModuleManager.png}

    \clsdcl{public class ModuleManager : IModuleManager}

    \clsdsp{Initializes and manages all modules..}

    \begin{methods}
        \begin{method}{void InitializeModules()}{Initializes all modules.}      
        \end{method}
        \begin{method}{void NotifyModulesOnSessionStart()}{Notifies all modules when a session starts.}
        \end{method}
        \begin{method}{void NotifyModulesOnSessionStop()}{Notifies all modules when a session stops.}
        \end{method}
    \end{methods}
\end{class}

\begin{class}{GlobalModuleConfiguration}
    \clsdiagram{resources/Classes/Core/MORR/Modules/GlobalModuleConfiguration.png}

    \clsdcl{public class GlobalModuleConfiguration : IConfiguration}

    \clsdsp{Encapsulates all global module configuration options.}

    \begin{attributes}
        \attribute{public IEnumerable<Type> EnabledModules \{ get; \}}{The types of all IModule instances that should be enabled.}
    \end{attributes}
    
    \begin{methods}
        \begin{method}{public void Parse(RawConfiguration configuration)}{Parses the configuration from the provided value.}
            \begin{parameters}
                \para{RawConfiguration configuration}{The configuration value to parse from.}
            \end{parameters}
        \end{method}
    \end{methods}
\end{class}

\subsection*{MORR.Core.Data.Capture}

\begin{class}{CaptureException}
    \clsdiagram[width=0.3\textwidth]{resources/Classes/Core/MORR/Capture/CaptureException.png}

    \clsdcl{public class CaptureException : Exception}

    \clsdsp{A generic capture exception that all specialized capture exceptions derive from.}
\end{class}

\subsection*{MORR.Core.Data.Capture.Video.Exceptions}

\begin{class}{VideoCaptureException}
    \clsdiagram[width=0.3\textwidth]{resources/Classes/Core/MORR/Capture/VideoCaptureException.png}

    \clsdcl{public class VideoCaptureException : CaptureException}

    \clsdsp{An exception thrown if video sample capturing fails.}
\end{class}

\subsection*{MORR.Core.Data.Capture.Video}

\begin{interface}{IVideoCapture}
    \clsdiagram[width=0.35\textwidth]{resources/Classes/Core/MORR/Capture/IVideoCapture.png}

    \clsdcl{public interface IVideoCapture}

    \clsdsp{Captures video output and provides it on a per-sample basis.}

    \begin{methods}
        \begin{method}{VideoSample NextSample()}{Gets the next VideoSample from the capture.}
            \return{VideoSample}{The next VideoSample from the capture.}
        \end{method}
    \end{methods}
\end{interface}

\subsection*{MORR.Core.Data.Capture.Video.WinAPI}

\begin{interface}{DesktopCapture}
    \clsdiagram[width=0.6\textwidth]{resources/Classes/Core/MORR/Capture/DesktopCapture.png}

    \clsdcl{public class DesktopCapture : IVideoCapture}

    \clsdsp{Captures the desktop using the Windows API and provides the capture samples.}

    \begin{attributes}
        \attribute{[Import] DesktopCaptureConfiguration CaptureConfiguration \{ get; set; \}}{The IConfiguration instance specifying configuration options regarding desktop capture.}
    \end{attributes}

    \begin{methods}
        \begin{method}{VideoSample NextSample()}{Gets the next VideoSample from the capture.}
            \return{VideoSample}{The next VideoSample from the capture.}
        \end{method}
    \end{methods}   
\end{interface}

\begin{class}{DesktopCaptureConfiguration}
    \clsdiagram[width=0.35\textwidth]{resources/Classes/Core/MORR/Capture/DesktopCaptureConfiguration.png}

    \clsdcl{public class DesktopCaptureConfiguration : IConfiguration}

    \clsdsp{Encapsulates all DesktopCapture configuration options.}

    \begin{attributes}
        \attribute{string DeviceToRecord \{ get; set; \}}{The device name of the screen to record.}
    \end{attributes}

    \begin{methods}
        \begin{method}{public void Parse(string configuration)}{Parses the configuration from the provided value.}
            \begin{parameters}
                \para{string configuration}{The configuration value to parse from.}
            \end{parameters}
        \end{method}
    \end{methods}
\end{class}

\subsection*{MORR.Core.Data.Capture.Video.WinAPI.Utility}

\begin{class}{CaptureHelper}
    \clsdiagram[width=0.55\textwidth]{resources/Classes/Core/MORR/Capture/CaptureHelper.png}

    \clsdcl{internal static class CaptureHelper}

    \clsdsp{Provides utility methods for working with Windows' GraphicsCaptureItems.}

    \begin{attributes}
        \attribute{static bool CanCreateItemWithoutPicker \{ get; \}}{Indicates whether the creation of a GraphicsCaptureItem instance requires user interaction. True if GraphicsCaptureItems can be created without user interaction, false otherwise.}
    \end{attributes}
    \begin{methods}
        \begin{method}{public static GraphicsCaptureItem CreateItemForMonitor(IntPtr hMon)}{Creates a GraphicsCaptureItem for the specified monitor.}
            \begin{parameters}
                \para{IntPtr hMon}{The handle of the monitor to create the GraphicsCaptureItem for.}
            \end{parameters}
        \end{method}
    \end{methods}
\end{class}

\begin{class}{Direct3D11Helper}
    \clsdiagram[width=0.8\textwidth]{resources/Classes/Core/MORR/Capture/Direct3D11Helper.png}

    \clsdcl{internal static class Direct3D11Helper}

    \clsdsp{Provides utility methods for using Direct3D11 objects.}

    \begin{methods}
        \begin{method}{public static IDirect3DDevice CreateDevice(bool useWARP = false)}{Creates a new IDirect3DDevice device using the specified driver type.}
            \begin{parameters}
                \para{bool useWARP}{Indicates whether the created device should utilize hardware support or software emulation. True if software emulation should be used, false otherwise.}
            \end{parameters}
            \return{IDirect3DDevice}{The created device.}
        \end{method}
        \begin{method}{public static IDirect3DSurface \\CreateDirect3DSurfaceFromSharpDXTexture(Texture2D texture)}{Creates a IDirect3DSurface from a SharpDX texture.}
            \begin{parameters}
                \para{Texture2D texture}{The SharpDX texture to convert.}
            \end{parameters}
            \return{IDirect3DSurface}{The created Direct3D surface.}
        \end{method}
        \begin{method}{public static Device CreateSharpDXDevice(IDirect3DDevice device)}{Create a SharpDX device from the specified Direct3D device.}
            \begin{parameters}
                \para{IDirect3DDevice device}{The device to create a SharpDX device from.}
            \end{parameters}
            \return{Device}{The created SharpDX device.}
        \end{method}
        \begin{method}{public static Texture2D CreateSharpDXTexture2D(IDirect3DSurface surface)}{Creates a SharpDX texture from the specified Direct3D surface.}
            \begin{parameters}
                \para{IDirect3DSurface surface}{The surface to convert.}
            \end{parameters}
            \return{Texture2D}{The created SharpDX texture.}
        \end{method}
    \end{methods}
\end{class}

\begin{class}{MonitorEnumerationHelper}
    \clsdiagram[width=0.4\textwidth]{resources/Classes/Core/MORR/Capture/MonitorEnumerationHelper.png}

    \clsdcl{internal static class MonitorEnumerationHelper}

    \clsdsp{Provides utility methods for enumerating monitors.}

    \begin{methods}
        \begin{method}{public static IEnumerable<MonitorInfo> GetMonitors()}{Gets MonitorInfo instances containing information about all connected monitors.}
            \return{IEnumerable<MonitorInfo>}{A list of MonitorInfo instances containing information about all connected monitors.}
        \end{method}
    \end{methods}
\end{class}

\begin{class}{MonitorInfo}
    \clsdiagram[width=0.35\textwidth]{resources/Classes/Core/MORR/Capture/MonitorInfo.png}

    \clsdcl{internal class MonitorInfo}

    \clsdsp{Represents information about a connected monitor.}

    \begin{attributes}
        \attribute{public bool IsPrimary \{ get; private set; \}}{Indicates whether the describe monitor is the primary monitor. True if the monitor is the primary monitor, false otherwise.}
        \attribute{public Vector2 ScreenSize \{ get; private set; \}}{The size of the screen as (width, height) vector.}
        \attribute{public Rect MonitorArea \{ get; private set; \}}{The size of the monitor.}
        \attribute{public Rect WorkArea \{ get; private set;\}}{The size of the working area.}
        \attribute{public string DeviceName \{ get; private set;\}}{The name of the device.}
        \attribute{public IntPtr HMon \{ get; private set;\}}{The handle of the monitor.}
    \end{attributes}
\end{class}
\subsection*{MORR.Core.Data.IntermediateFormat}

\begin{absclass}{IntermediateFormatSample}
    \clsdiagram{resources/Classes/Core/MORR/Data/IntermediateFormat/IntermediateFormatSample.png}

    \clsdcl{public abstract class IntermediateFormatSample : Event}

    \clsdsp{A sample in an intermediate format.}

    \begin{attributes}
        \attribute{public Type Type \{ get; set; \}}{The type of the event that is serialized.}
        \attribute{public byte[] Data \{ get; set; \}}{The data that is serialized.}
    \end{attributes}
\end{absclass}

\subsection*{MORR.Core.Data.IntermediateFormat.Json}

\begin{class}{JsonIntermediateFormatSample}
    \clsdiagram{resources/Classes/Core/MORR/Data/IntermediateFormat/Json/JsonIntermediateFormatSample.png}

    \clsdcl{public class JsonIntermediateFormatSample : IntermediateFormatSample}

    \clsdsp{A sample in JSON intermediate format.}

    \begin{attributes}
        \attribute{public JsonDocument JsonEncodedData \{ get; \}}{The data that is serialized in JSON-compatible format.}
        \attribute{public JsonEncodedText JsonEncodedType \{ get; \}}{The type of the event that is serialized in JSON-compatible format.}
    \end{attributes}
\end{class}

\begin{class}{JsonIntermediateFormatSerializer}
    \clsdiagram{resources/Classes/Core/MORR/Data/IntermediateFormat/Json/JsonIntermediateFormatSerializer.png}

    \clsdcl{public class JsonIntermediateFormatSerializer : DefaultEncodeableEventQueue<JsonIntermediateFormatSample>, IModule}

    \clsdsp{A serializer for converting typed events into their representation in the JSON-intermediate format.}

    \begin{attributes}
        \attribute{public bool IsActive \{ get; set; \}}{Indicates whether the module is active. True if it is active, false otherwise.}
        \attribute{public Guid Identifier \{ get; \}}{The identifier of the module.}
    \end{attributes}

    \begin{methods}
        \begin{method}{public bool Initialize(bool isEnabled)}{Initializes the module.}
            \begin{parameters}
                \para{bool isEnabled}{Indicates whether the module is enabled. True if it is enabled, false otherwise.}
            \end{parameters}
        \end{method}
    \end{methods}
\end{class}

\begin{class}{JsonIntermediateFormatDeserializer}
    \clsdiagram{resources/Classes/Core/MORR/Data/IntermediateFormat/Json/JsonIntermediateFormatDeserializer.png}

    \clsdcl{public class JsonIntermediateFormatDeserializer : IModule}

    \clsdsp{A deserializer for converting events in their JSON-intermediate format representation to typed events.}

    \begin{attributes}
        \attribute{public bool IsActive \{ get; set; \}}{Indicates whether the module is active. True if it is active, false otherwise.}
        \attribute{public Guid Identifier \{ get; \}}{The identifier of the module.}
    \end{attributes}

    \begin{methods}
        \begin{method}{public bool Initialize(bool isEnabled)}{Initializes the module.}
            \begin{parameters}
                \para{bool isEnabled}{Indicates whether the module is enabled. True if it is enabled, false otherwise.}
            \end{parameters}
        \end{method}
    \end{methods}
\end{class}
\input{chapter/Classes/Core/MORR/Data/Transcoding/Metadata/MetadataTranscoding}
\input{chapter/Classes/Core/MORR/Data/Transcoding/Video/VideoTranscoding}
\input{chapter/Classes/Core/MORR/Data/Transcoding/WinAPI/WinAPI_Transcoding}

\begin{interface}{IEncoder}
    \clsdcl{public interface IEncoder}

    \clsdsp{Encodes the provided samples into a file at a path.}

    \begin{attributes}
        \attribute{event VideoSampleRequestedEventHandler VideoSampleRequested}{Event raised when the next VideoSample may be added for encoding.}
        \attribute{event MetadataSampleRequestedEventHandler MetadataSampleRequested}{Event raised when the next MetadataSample may be added for encoding.}
    \end{attributes}

    \begin{methods}
        \begin{method}{void EncodeTo(string path)}{Encodes all provided sample to a file at the specified location.}
            \begin{parameters}
                \para{string path}{The path to encode to.}
            \end{parameters}
            \begin{exceptions}
                \excp{EncodingException}{Exception thrown if the encoding fails. This may be a MetadataEncodingException or a VideoEncodingException.}
            \end{exceptions}
        \end{method}
    \end{methods}
\end{interface}

\begin{interface}{IDecoder}
    \clsdcl{public interface IDecoder}

    \clsdsp{Decodes samples stored in a file at a path and provides the decoded samples.}

    \begin{attributes}
        \attribute{event VideoSampleDecodedEventHandler VideoSampleDecoded}{Event raised when a VideoSample gets decoded.}
        \attribute{event MetadataSampleDecodedEventHandler MetadataSampleDecoded}{Event raised when a MetadataSample gets decoded.}
    \end{attributes}

    \begin{methods}
        \begin{method}{void DecodeFrom(string path)}{Decodes samples contained in a file at a specified location.}
            \begin{parameters}
                \para{string path}{The path to decode from.}
            \end{parameters}
            \begin{exceptions}
                \excp{DecodingException}{Exception thrown if decoding fails. This may be a MetadataDecodingException or a VideoDecodingException.}
            \end{exceptions}
        \end{method}
    \end{methods}
\end{interface}

\begin{class}{EncodingException}
    \clsdcl{public class EncodingException : Exception}

    \clsdsp{A generic encoding exception that all specialized encoding exceptions derive from.}
\end{class}

\begin{class}{DecodingException}
    \clsdcl{public class DecodingException : Exception}

    \clsdsp{A generic decoding exception that all specialized decoding exceptions derive from.}
\end{class}
\subsection{UI}

\begin{class}{App}
	\clsdiagram[scale = 0.7]{resources/Classes/Core/UI/App.png}

    \clsdcl{public partial class App : Application}

    \clsdsp{Auto-generated WPF application class}

\end{class}

\subsection*{Controls. NotifyIcon}
\begin{class}{NotifyIcon}
	\clsdiagram[scale = 0.7]{resources/Classes/Core/UI/NotifyIcon.png}

    \clsdcl{public class NotifyIcon : Control, IDisposable}

    \clsdsp{Manages a tray icon}

	\begin{attributes}
        \attribute{public string Tooltip}{The tooltip shown during hovering}
        \attribute{public ImageSource IconSource}{The icon shown in the tray}
        \attribute{public ICommand Command}{The command to execute on left click}
        \attribute{public object CommandParameter}{The parameter of the command to execute on left click}
        \attribute{public new ContextMenu ContextMenu}{The context menu to show on right click}
    \end{attributes}

   	\begin{constructors}
        \begin{constructor}{public NotifyIcon()}{Initializes a NotifyIcon}
        \end{constructor}
    \end{constructors}
    
    \begin{methods}
    	\begin{method}{public void Dispose()}{Frees all unmanaged resources}
        \end{method}
     \end{methods}

\end{class}

\subsection*{Dialogs}

\begin{class}{ErrorDialog}
	\clsdiagram[scale = 0.7]{resources/Classes/Core/UI/ErrorDialog.png}

    \clsdcl{public partial class ErrorDialog : Window}

    \clsdsp{A dialog shown when an error occured}

    \begin{attributes}
        \attribute{public string ErrorMessage}{Error message of the error dialog}
    \end{attributes}
    
    \begin{constructors}
        \begin{constructor}{public ErrorDialog(string errorMessage)}{Initializes an ErrorDialog with errorMessage}
            \begin{parameters}
                \para{String errorMessage}{A string with an error message}
            \end{parameters}
        \end{constructor}
     \end{constructors}

\end{class}

\begin{class}{InfomationDialog}
	\clsdiagram[scale = 0.7]{resources/Classes/Core/UI/InformationDialog.png}

    \clsdcl{public partial class InformationDialog : Window }

    \clsdsp{A dialog shown before the start of the recording}

     \begin{constructors}
        \begin{constructor}{public InformationDialog()}{Initializes an InformationDialog}
        \end{constructor}
     \end{constructors}

\end{class}

\begin{class}{SaveDialog}
	\clsdiagram[scale = 0.7]{resources/Classes/Core/UI/SaveDialog.png}

    \clsdcl{public partial class SaveDialog : Window}

    \clsdsp{A dialog shown after the end of the recording}
    
    \begin{constructors}
        \begin{constructor}{public SaveDialog()}{Initializes a SaveDialog}
        \end{constructor}
     \end{constructors}

\end{class}

\subsection*{ViewModels}

\begin{class}{ApplicationViewModel}
	\clsdiagram[scale = 0.7]{resources/Classes/Core/UI/ApplicationViewModel.png}

    \clsdcl{public class ApplicationViewModel : DependencyObject}

    \clsdsp{}

	\begin{attributes}
        \attribute{public bool IsRecording}{Current state of the recording process}
    \end{attributes}    
    
    \begin{constructors}
        \begin{constructor}{public ApplicationViewModel()}{Initializes an ApplicationViewModel}
        \end{constructor}
     \end{constructors}
 

\end{class}

\subsection*{ViewModels.Utility}

\begin{class}{RelayCommand}
	\clsdiagram[scale = 0.7]{resources/Classes/Core/UI/RelayCommand.png}

    \clsdcl{public RelayCommand : ICommand }

    \clsdsp{A concrete command that implements ICommand interface}

	\begin{constructors}
        \begin{constructor}{public RelayCommand(Action<object> execute, Predicate<object>? canExecute = null)}{Creates a new RelayCommand}
        \begin{parameters}
                \para{Action<object> execute}{The execution logic}
                \para{Predicate<object>? canExecute = null}{The execution status logic, set to null}
            \end{parameters}
        \end{constructor}
     \end{constructors}
    
    
    \begin{methods}
    	\begin{method}{public bool CanExecute(object parameter)}{Determines whether the RelayCommand can be executed in current state}
            \return{bool}{True if this RelayCommand can be executed; otherwise, false.}
            \begin{parameters}
                \para{object parameter}{Data used by the RelayCommand}
            \end{parameters}
        \end{method}
        \begin{method}{public void Execute(object parameter)}{Executes the RelayCommand}
            \begin{parameters}
               \para{object parameter}{Data used by the RelayCommand}
            \end{parameters}
        \end{method}
        
    \end{methods}

\end{class}




\newpage
\subsection*{MORR.Shared.Modules}

\begin{interface}{IModule}
    \clsdiagram[width=0.3\textwidth]{resources/Classes/Common/Shared/Modules/IModule.png}

    \clsdcl{public interface IModule}

    \clsdsp{Defines a module.}

    \begin{attributes}
        \attribute{bool IsEnabled \{ get; set; \}}{Indicates whether the module is currently enabled. True if the module is enabled, false otherwise.}
        \attribute{Guid Identifier \{ get; \}}{The identifier of the module.}
    \end{attributes}
    \begin{methods}
        \begin{method}{void Initialize()}{Initializes the module.}
        \end{method}
    \end{methods}
\end{interface}

\begin{interface}{ICollectingModule}
    \clsdiagram[width=0.25\textwidth]{resources/Classes/Common/Shared/Modules/ICollectingModule.png}

    \clsdcl{public interface ICollectingModule : IModule}

    \clsdsp{A module that accepts no events as an input and generates one or more events s an output.}
\end{interface}

\begin{interface}{ITransformingModule}
    \clsdiagram[width=0.28\textwidth]{resources/Classes/Common/Shared/Modules/ITransformingModule.png}

    \clsdcl{public interface ITransformingModule : IModule}

    \clsdsp{A module that accepts one or more events as an input and generates one or more events as an output.}
\end{interface}

\begin{interface}{IReceivingModule}
    \clsdiagram[width=0.25\textwidth]{resources/Classes/Common/Shared/Modules/IReceivingModule.png}

    \clsdcl{public interface IReceivingModule : IModule}

    \clsdsp{A module that accepts one or more events as an input and generates no events as an output.}
\end{interface}
\newpage
\section{Common}
\subsection*{MORR.Shared.Configuration}

\begin{interface}{IConfiguration}
    \clsdiagram[scale = 0.3]{resources/Classes/Common/Shared/Configuration/IConfiguration.png}

    \clsdcl{public interface IConfiguration}

    \clsdsp{Encapsulates a self-contained unit of configuration.}
    
\end{interface}

\begin{class}{RawConfiguration}
	\clsdiagram[scale = 1]{resources/Classes/Common/Shared/Configuration/RawConfiguration.png}
	
	\clsdcl{public class RawConfiguration}
	
	\clsdsp{}
	
	\begin{attributes}
		\attribute{public string RawValue \{ get; \}}{Value of the configuration}
	\end{attributes}
	
\end{class}
		
\subsection*{MORR.Shared.Events}

\begin{absclass}{Event}
    \clsdiagram[width=0.4\textwidth]{resources/Classes/Common/Shared/Events/Event.png}

    \clsdcl{public abstract class Event}

    \clsdsp{Provides attributes shared between every user interaction event.}

    \begin{attributes}
        \attribute{public DateTime Timestamp \{ get; set; \}}{The timestamp at which the event occured.}
        \attribute{public Guid IssuingModule \{ get; set; \}}{The identifier of the module that issued the event.}
    \end{attributes}
\end{absclass}

\subsection*{MORR.Shared.Events.Queue}

\begin{absclass}{DefaultEventQueue<T>}
	\clsdiagram[width=0.6\textwidth]{resources/Classes/Common/Shared/Events/Queue/DefaultEventQueue.png}
	
	\clsdcl{public abstract class DefaultEventQueue<T> : SupportDeserializationEventQueue<T> where T : Event}
	
	\clsdsp{Provides an event queue for the most common scenarios that supports deserialization.}
\end{absclass}

\begin{absclass}{DefaultEncodableEventQueue<T>}
	\clsdiagram[width=0.6\textwidth]{resources/Classes/Common/Shared/Events/Queue/DefaultEncodableEventQueue.png}
	
	\clsdcl{public abstract class DefaultEncodeableEventQueue<T> : BaseEventQueue<T>, IEncodableEventQueue<T> where T : Event}
	
	\clsdsp{Provides an event queue for events intended for encoding.}
\end{absclass}

\begin{absclass}{DefaultDecodableEventQueue<T>}
	\clsdiagram[width=0.6\textwidth]{resources/Classes/Common/Shared/Events/Queue/DefaultDecodableEventQueue.png}
	
	\clsdcl{public abstract class DefaultDecodableEventQueue<T> : BaseEventQueue<T>, IDecodableEventQueue<T> where T : Event}
	
	\clsdsp{Provides an event queue for events intended for decoding.}
\end{absclass}

\begin{absclass}{NonDeserializableEventQueue<T>}
	\clsdiagram[width=0.6\textwidth]{resources/Classes/Common/Shared/Events/Queue/NonDeserializableEventQueue.png}
	
	\clsdcl{public abstract class NonDeserializableEventQueue<T> : BaseEventQueue<T>, 
	IReadOnlyEventQueue<T> where T : Event}
	
	\clsdsp{Provides an event queue for events that do not support deserialization.}
\end{absclass}
	
\begin{absclass}{BaseEventQueue<TEvent>}
	\clsdiagram[width=0.5\textwidth]{resources/Classes/Common/Shared/Events/Queue/BaseEventQueue.png}
	
	\clsdcl{public abstract class BaseEventQueue<TEvent> where TEvent : Event}
	
	\clsdsp{Provides the basic structure of an EventQueue.}
	
	\begin{attributes}
		\attribute{public bool IsClosed}{Describes whether the queue is currently enabled to queue new events or not.}
	\end{attributes}

    \begin{methods}
        \begin{method}{public IAsyncEnumerable<TEvent> GetEvents()}{Asynchronously gets all events as concrete type TEvent.}
            \return{IAsyncEnumerable<TEvent>}{A stream of events of type T.}
        \end{method}
        \begin{method}{public void Enqueue(TEvent @event)}{Asynchronously enqueues a new event.}
        	\begin{parameters}
        		\para{TEvent @event}{The event to enqueue}
        	\end{parameters}
        \end{method}
        \begin{method}{public void Open()}{Opens the EventQueue so new events can be queued.}
        \end{method}
        \begin{method}{public void Close()}{Closes the EventQueue so no new event can be queued.}
        \end{method}
    \end{methods}
\end{absclass}

\begin{interface}{IDecodableEventQueue<out T>}
	\clsdiagram[width=0.5\textwidth]{resources/Classes/Common/Shared/Events/Queue/IDecodableEventQueue.png}

	\clsdcl{public interface IDecodableEventQueue<out T> where T : Event}
	
	\clsdsp{Provides a read-only queue for event types intended for decoding.}
	
	\begin{attributes}
		\attribute{public bool IsClosed \{ get; \}}{Describes whether the queue is currently enabled to queue new events or not.}
	\end{attributes}
	
\end{interface}

\begin{interface}{IEncodableEventQueue<out T>}
	\clsdiagram[width=0.5\textwidth]{resources/Classes/Common/Shared/Events/Queue/IEncodableEventQueue.png}

	\clsdcl{public interface IEncodableEventQueue<out T> where T : Event}
	
	\clsdsp{Provides a read-only queue for event types intended for encoding.}
	
	\begin{attributes}
		\attribute{public bool IsClosed \{ get; \}}{Describes whether the queue is currently enabled to queue new events or not.}
	\end{attributes}
	
\end{interface}

\begin{interface}{IReadOnlyEventQueue<out T>}
	\clsdiagram[width=0.5\textwidth]{resources/Classes/Common/Shared/Events/Queue/IReadOnlyEventQueue.png}

	\clsdcl{public interface IReadOnlyEventQueue<out T> where T : Event}
	
	\clsdsp{Provides a read-only access to a queue for events as concrete type T.}
	
	\begin{attributes}
		\attribute{public bool IsClosed \{ get; \}}{Describes whether the queue is currently enabled to queue new events or not.}
	\end{attributes}
	
\end{interface}

\begin{interface}{ISupportDeserializationEventQueue<out T>}
	\clsdiagram[width=0.6\textwidth]{resources/Classes/Common/Shared/Events/Queue/ISupportDeserializationEventQueue.png}

	\clsdcl{public interface ISupportDeserializationEventQueue<out T> where T : Event}
	
	\clsdsp{Provides a single-writer-multiple-reader queue for event types with support for deserialization.}
	
	\begin{methods}
		\begin{method}{public void Open()}{Opens the EventQueue so new events can be queued.}
        \end{method}
        \begin{method}{public void Close()}{Closes the EventQueue so no new event can be queued.}
        \end{method}
    \end{methods}

\end{interface}

\begin{absclass}{SupportDeserializationEventQueue<T>}
	\clsdiagram[width=0.8\textwidth]{resources/Classes/Common/Shared/Events/Queue/SupportDeserializationEventQueue.png}
	
	\clsdcl{public abstract class SupportDeserializationEventQueue<T> : BaseEventQueue<T>, IReadOnlyEventQueue<T>, ISupportDeserializationEventQueue<T> where T : Event}
	
	\clsdsp{Provides a single-writer-multiple-reader event queue for event types which may be deserialized for processing.}
	
	\begin{methods}
		\begin{method}{public void Enqueue(object @event)}{Asynchronously enqueues a new event of type T.}
            \begin{parameters}
                \para{T @event}{The event to enqueue.}
            \end{parameters}
            \begin{exceptions}
            	\excp{ArgumentException}{Exception is thrown when event type is not T}
            \end{exceptions}
        \end{method}
    \end{methods}
\end{absclass}
	
\subsection*{MORR.Shared.Events.Queue.Strategy}

\begin{class}{ChannelConsumingException}
	\clsdiagram[width=0.4\textwidth]{resources/Classes/Common/Shared/Events/Queue/Strategy/ChannelConsumingException.png}
	
	\clsdcl{public class ChannelConsumingException : Exception}	
	
	\clsdsp{A simple Exception that encapsulates errors occuring in ChannelStrategies.}
	
\end{class}
	
\begin{interface}{IEventQueueStorageStrategy}
    \clsdiagram[width=0.5\textwidth]{resources/Classes/Common/Shared/Events/Queue/Strategy/IEventQueueStorageStrategy.png}

    \clsdcl{public interface IEventQueueStorageStrategy<T> where T : Event}

    \clsdsp{Provides the backing store of an EventQueue with a specific storage strategy.}

\end{interface}

\subsubsection*{MORR.Shared.Event.Queue.Strategy.MultiConsumer}

\begin{class}{BoundedMultiConsumerChannelStrategy<TEvent>}
	\clsdiagram[width=0.5\textwidth]{resources/Classes/Common/Shared/Events/Queue/Strategy/MultiConsumer/BoundedMultiConsumerChannelStrategy.png}
	
	\clsdcl{public class BoundedMultiConsumerChannelStrategy<TEvent> : \\\
	MultiConsumerChannelStrategy<TEvent> where TEvent : Event}
	
	\clsdsp{A BoundedMultiConsumerChannelStrategy is a Distributive-FIFO Queue which allows multiple producers and multiple consumers. However every event is propagated to each consumer once. This queue is bounded by a buffer capacity. To be more detailed each channel is bound by the bufferCapacity. Is the capacity reached for a specific channel it will drop the oldest event and write the new event.}
\end{class}

\begin{absclass}{MultiConsumerChannelStrategy<TEvent>} 
	\clsdiagram[width=0.5\textwidth]{resources/Classes/Common/Shared/Events/Queue/Strategy/MultiConsumer/MultiConsumerChannelStrategy.png}
	
	\clsdcl{public abstract class MultiConsumerChannelStrategy<TEvent> : \\\
	IEventQueueStorageStrategy<TEvent> where TEvent : Event}
	
	\clsdsp{A MultiConsumerChannelStrategies is a Distributive-FIFO Queue which allows multiple producers and multiple consumers. However every event is propagated to each consumer once.}
	
	\begin{attributes}
		\attribute{public bool IsClosed { get; private set; }}{Describes whether the queue is currently enabled to queue new events or not}
	\end{attributes}
	
	\begin{methods}
		\begin{method}{public IAsyncEnumerable<TEvent> GetEvents(CancellationToken token = default)}{Asynchronously gets all events as concrete type T}
			\begin{parameters}
				\para{CancellationToken token = default}
			\end{parameters}
			\begin{exceptions}
				\excp{ChannelConsumingException}{Exception is thrown when max number of consumers is reached}
			\end{exceptions}
			\return{IAsyncEnumerable<TEvent>}{A stream of events type T}
		\end{method}
		\begin{method}{public async void Enqueue(TEvent @event)}{Asynchronously enqueues a new event}
			\begin{parameters}
				\para{TEvent @event}{The event to enqueue}
			\end{parameters}
		\end{method}
		\begin{method}{public void Open()}{Notifies the event queue that it is free to accept input.}
        \end{method}
        \begin{method}{public void Close()}{Notifies the event queue that no more events will be queued.}
        \end{method}
	\end{methods}
\end{absclass}

\begin{class}{UnboundedMultiConsumerChannelStrategy<TEvent>}
	\clsdiagram[width=0.5\textwidth]{resources/Classes/Common/Shared/Events/Queue/Strategy/MultiConsumer/UnboundedMultiConsumerChannelStrategy.png}
	
	\clsdcl{public class UnboundedMultiConsumerChannelStrategy<TEvent> : \\\
	MultiConsumerChannelStrategy<TEvent> where TEvent : Event}
	
	\clsdsp{An UnboundedMultiConsumerChannelStrategy is a Distributive-FIFO Queue which allows multiple producers and multiple consumers. However every event is propagated to each consumer once. This queue is unbound, which means that an event is never dismissed. Make sure to only use this queue if you are sure it is bound by a maximum number of events.}
\end{class}

\subsubsection*{SingleConsumer}

\begin{class}{BoundedSingleConsumerChannelStrategy<TEvent>}
	\clsdiagram[width=0.5\textwidth]{resources/Classes/Common/Shared/Events/Queue/Strategy/SingleConsumer/BoundedSingleConsumerChannelStrategy.png}
	
	\clsdcl{public class BoundedSingleConsumerChannelStrategy<TEvent> : \\\
	 SingleConsumerChannelStrategy<TEvent> where TEvent : Event}
	
	\clsdsp{A BoundedSingleConsumerChannelStrategy is a Distributive-FIFO Queue which allows multiple producers and a single consumer. This queue is performance optimized for a single consumer and should be preferred in this case. This queue is bounded by a buffer capacity. If capacity is reached it will dismiss the oldest event and queue the new event.}
	
\end{class}

\begin{absclass}{SingleConsumerChannelStrategy<TEvent>} 
	\clsdiagram[width=0.5\textwidth]{resources/Classes/Common/Shared/Events/Queue/Strategy/SingleConsumer/SingleConsumerChannelStrategy.png}
	
	\clsdcl{public abstract class SingleConsumerChannelStrategy<TEvent> : \\\
	IEventQueueStorageStrategy<TEvent> where TEvent : Event}
	
	\clsdsp{A SingleConsumerChannelStrategies is a Distributive-FIFO Queue which allows multiple producers and singer consumers. This queue is performance optimized for a single consumer and should be preferred in this case.}
	
	\begin{attributes}
		\attribute{public bool IsClosed { get; private set; }}{Describes whether the queue is currently enabled to queue new events or not}
	\end{attributes}
	
	\begin{methods}
		\begin{method}{public IAsyncEnumerable<TEvent> GetEvents(CancellationToken token = default)}{Asynchronously gets all events as concrete type T}
			\begin{parameters}
				\para{CancellationToken token = default}
			\end{parameters}
			\begin{exceptions}
				\excp{ChannelConsumingException}{Exception is thrown when max number of consumers is reached}
			\end{exceptions}
			\return{IAsyncEnumerable<TEvent>}{A stream of events type T}
		\end{method}
		\begin{method}{public async void Enqueue(TEvent @event)}{Asynchronously enqueues a new event}
			\begin{parameters}
				\para{TEvent @event}{The event to enqueue}
			\end{parameters}
		\end{method}
		\begin{method}{public void Open()}{Notifies the event queue that it is free to accept input.}
        \end{method}
        \begin{method}{public void Close()}{Notifies the event queue that no more events will be queued.}
        \end{method}
	\end{methods}
\end{absclass}

\begin{class}{UnboundedSingleConsumerChannelStrategy<TEvent>}
	\clsdiagram[width=0.5\textwidth]{resources/Classes/Common/Shared/Events/Queue/Strategy/SingleConsumer/UnboundedSingleConsumerChannelStrategy.png}
	
	\clsdcl{public class UnboundedSingleConsumerChannelStrategy<TEvent> :\\\
	SingleConsumerChannelStrategy<TEvent> where TEvent : Event}
	
	\clsdsp{A UnboundedSingleConsumerChannelStrategy is a Distributive-FIFO Queue which allows multiple producers and a single consumer. This queue is performance optimized for a single consumer and should be preferred in this case.  This queue is unbounded and will therefore don't dismiss any events.}
\end{class}


\subsection*{MORR.Shared.Modules}

\begin{interface}{IModule}
    \clsdiagram[width=0.3\textwidth]{resources/Classes/Common/Shared/Modules/IModule.png}

    \clsdcl{public interface IModule}

    \clsdsp{Defines a module.}

    \begin{attributes}
        \attribute{bool IsEnabled \{ get; set; \}}{Indicates whether the module is currently enabled. True if the module is enabled, false otherwise.}
        \attribute{Guid Identifier \{ get; \}}{The identifier of the module.}
    \end{attributes}
    \begin{methods}
        \begin{method}{void Initialize()}{Initializes the module.}
        \end{method}
    \end{methods}
\end{interface}

\begin{interface}{ICollectingModule}
    \clsdiagram[width=0.25\textwidth]{resources/Classes/Common/Shared/Modules/ICollectingModule.png}

    \clsdcl{public interface ICollectingModule : IModule}

    \clsdsp{A module that accepts no events as an input and generates one or more events s an output.}
\end{interface}

\begin{interface}{ITransformingModule}
    \clsdiagram[width=0.28\textwidth]{resources/Classes/Common/Shared/Modules/ITransformingModule.png}

    \clsdcl{public interface ITransformingModule : IModule}

    \clsdsp{A module that accepts one or more events as an input and generates one or more events as an output.}
\end{interface}

\begin{interface}{IReceivingModule}
    \clsdiagram[width=0.25\textwidth]{resources/Classes/Common/Shared/Modules/IReceivingModule.png}

    \clsdcl{public interface IReceivingModule : IModule}

    \clsdsp{A module that accepts one or more events as an input and generates no events as an output.}
\end{interface}
\newpage
\section{Browser Extension}
\issue{The public \member{start} and \member{stop} functions of \class{BackgroundScript} will not be invoked anywhere as
\class{BackgroundScript} is on top of the class hierarchy.}
{Let the \class{BackgroundScript} manage its own state.}
{Make \member{start} and \member{stop} functions private and provide public \member{run} function.}

\issue{Current semantics of the \class{ICommunicationInterface} functions do not make it clear how a "Stop" signal should
be transferred to the \class{BackgroundScript}.}{Provide additional function in \class{ICommunicationInterface}.}
{Add \member{addOnStopListener} function to \class{ICommunicationInterface}, which will invoke the given 
callback function when a "Stop" signal as received.}

\issue{The \class{DOMEventRecorder} is mostly unaware if a given DOMEvent should be converted into a \class{BrowserEvent},
it however always expects the \class{DOMEventFactory} to create a \class{BrowserEvent} when passing
the DOMEvent as parameter.}
{Allow the \member{createEvent} function of \class{DOMEventFactory} to not return an event.}
{Changed return type of \member{createEvent} to "BrowserEvent | undefined". This approach
was preferred over throwing an exception, as this condition is regularly met due to the distributation of responsibilites
between the \class{DOMEventRecorder} and \class{DOMEventFactory} classes.}

\issue{The \class{DOMEventRecorder} needs to use asynchronous API functions for certain tasks.}
{Made the \member{createEvent} function asynchronous itself by using a Promise return type.}
{Changed return type of \member{createEvent} to "Promise<BrowserEvent | undefined>".}

\issue{The opened URLs are not known for all tab event types.}
{Adjust defintions to allow for an "url:unknown" magic value.}
{\class{OpenTabEvent} and \class{CloseTabEvent} have their \member{URL} value set to "url:unknown" when no
URL is passed to their constructors.}

\issue{Typescript does not allow for parameter-type based function overloads which are available in most common OOP languages.}
{Change function names of affected overloaded function.}
{Changed names of member functions of \class{TabEventFactory} to represent their specific return types.}

\issue{DOM based events should allow to be deserialized when retrieved from the contentscript.}
{Add deserializer functions.}
{Added static \member{deserialize} functions to \class{TextSelectionEvent}, \class{TextInputEvent},
\class{ButtonClickEvent} and \class{HoverEvent}.}

\issue{\class{DownloadEventFactory} needs to use an asynchronous API function to add the tab ID when creating a \class{DownloadEvent}.}
{Make \member{createEvent} function of \class{DownloadEventFactory} asynchronous.}
{Changed return type of \member{createEvent} function to "Promise<DownloadEvent>".}

\issue{Depending on context, a serialized \class{BrowserEvent} should have its member names match the browser extension style (ES6),
or the style used in the MORR main application.}
{Allow users of the \member{serialize} function to choose whether the leading underscores (ES6) should be omitted on serialization.}
{Changed function signature to \member{serialize(noUnderScore : boolean) : string}.}
