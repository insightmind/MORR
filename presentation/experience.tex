 \section{Experience - Applying Software Design principles}
 \subsection{Waterfall Model}
\begin{frame}{Using the Waterfall model}
Requirements phase
\begin{itemize}
\item Assessment of implementation complexity difficult without prior knowledge of the existing APIs and libraries
\item Some easy-to-implement features are overlooked and thus not specified
\end{itemize}
Design phase
\begin{itemize}
\item Some packages could not be realistically designed without knowledge about the implementation
\item UML diagrams quickly become confusing when exceeding a certain size
\end{itemize}

\end{frame}
\begin{frame}{Using the Waterfall model}
Implementation phase
\begin{itemize}
\item Software design clashes with restrictions of the programming language or API (certain methods had to be run in specific threads)
\item Many changes to design for practical reasons
\item Ambiguous library documentation required replanning of event encoding feature
\end{itemize}

Testing phase
\begin{itemize}
\item Testing often required class redesign to allow mocking of native methods
\item Testing of multithreaded/asynchronous code often difficult and error prone
\item Had to employ many abstractions as many classes operate "close to the operating system" which is not suitable for automated testing
\end{itemize}
\end{frame}

\begin{frame}{Using the Waterfall model}
\begin{block}{Verdict}
Pure waterfall model would not have worked, having the feedback option was essential.
\end{block}
\end{frame}